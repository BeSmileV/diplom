\subsection*{Технологии разработки клиентской части приложения}
\addcontentsline{toc}{subsection}{Технологии разработки клиентской части приложения}

Для реализации клиентской части платформы выбраны современные инструменты, обеспечивающие модульность, производительность, типизацию и масштабируемость интерфейса.

\subsubsection*{TypeScript}
\addcontentsline{toc}{subsubsection}{TypeScript}

JavaScript является одним из самых популярных языков программирования для веб-разработки. Он широко используется для создания динамичных веб-страниц и приложений, поскольку позволяет работать с элементами DOM, асинхронно загружать данные и обеспечивать интерактивность пользовательских интерфейсов. Однако JavaScript имеет важный недостаток — отсутствие статической типизации. Это означает, что переменные и функции не привязываются к определённым типам данных, что может привести к ошибкам на этапе выполнения, которые трудно обнаружить в процессе разработки. Особенно это может быть проблемой в крупных приложениях, где сложно отслеживать все возможные типы данных и их изменения.

Для устранения этих проблем был разработан язык TypeScript, являющийся надмножеством JavaScript. TypeScript добавляет в JavaScript статическую типизацию, что позволяет разработчикам явно указывать типы данных для переменных и функций. Это значительно снижает вероятность ошибок и улучшает поддержку кода в будущем. Благодаря строгой типизации TypeScript помогает предотвращать баги, связанные с динамическими типами в JavaScript, и улучшает автозаполнение в редакторах кода. TypeScript распространяется как библиотека, которую можно интегрировать в проекты на JavaScript, обеспечивая совместимость с существующим кодом и позволяя постепенно внедрять типизацию без необходимости переписывать весь проект. Это особенно важно в крупных и масштабируемых приложениях, где несколько разработчиков работают с общими компонентами, и типизация помогает поддерживать консистентность кода на протяжении всего проекта.


\subsubsection*{React}
\addcontentsline{toc}{subsubsection}{React}
React используется как библиотека для построения пользовательских интерфейсов. Она позволяет эффективно обновлять и рендерить компоненты при изменении данных, что делает интерфейс приложения быстрым и отзывчивым. React обеспечивает декларативный подход к построению UI, где разработчик описывает, как должен выглядеть интерфейс при разных состояниях приложения, а React самостоятельно управляет его обновлением. Это позволяет избежать многих ошибок, связанных с ручным управлением DOM, и значительно ускоряет процесс разработки.

\subsubsection*{Next.js}
\addcontentsline{toc}{subsubsection}{Next.js}

Next.js — это популярный фреймворк для React, который значительно расширяет его возможности, предоставляя разработчикам мощные инструменты для создания высокопроизводительных веб-приложений. Одной из ключевых особенностей Next.js является поддержка рендеринга на сервере (SSR, Server-Side Rendering) и статической генерации контента (SSG, Static Site Generation). Эти подходы позволяют улучшить производительность приложений, поскольку они обеспечивают быструю загрузку страниц, оптимизированную для поисковых систем и пользователей.

С серверным рендерингом Next.js позволяет генерировать HTML на сервере для каждой страницы перед её отправкой клиенту, что обеспечивает быстрое отображение контента. Это особенно полезно для SEO, поскольку поисковые системы могут индексировать контент сразу после его загрузки. Такой подход значительно улучшает видимость веб-приложений в поисковых системах и способствует их более высокому ранжированию.

Одним из ключевых преимуществ Next.js является автоматическая разбивка кода (code splitting). Это означает, что Next.js разделяет приложение на небольшие части, которые загружаются только по мере необходимости, что помогает сократить время загрузки страниц и улучшить пользовательский опыт. Таким образом, браузер загружает только тот код, который необходим для отображения текущей страницы, а не весь код приложения.

Кроме того, Next.js поддерживает гибкие методы рендеринга, что дает разработчикам возможность выбирать наиболее подходящий способ для каждой страницы. Статическая генерация (SSG) идеально подходит для страниц, которые не изменяются часто и могут быть сгенерированы заранее, например, блоговые записи или страницы с информацией о компании. В то время как для динамических страниц, которые требуют актуализации данных на сервере при каждом запросе, можно использовать серверный рендеринг.

Next.js также упрощает настройку маршрутизации и управление данными. Встроенная система маршрутизации автоматически генерирует страницы на основе файловой структуры, что делает создание новых страниц и маршрутов простым и интуитивно понятным. Кроме того, Next.js предоставляет инструменты для работы с API, что позволяет без труда интегрировать серверную логику в приложение.

Еще одной значимой особенностью является поддержка типизации с помощью TypeScript, что делает разработку в Next.js ещё более удобной и безопасной. Комбинация TypeScript и Next.js позволяет создавать стабильные и хорошо структурированные приложения, минимизируя количество ошибок на этапе разработки.

Важно отметить, что реализацию приложения можно было бы построить и на чистом React, однако в этом случае значительная часть функциональности, такой как маршрутизация, SSR, SSG и работа с API, потребовала бы ручной настройки и подключения дополнительных библиотек. Использование Next.js избавляет от необходимости собирать всё вручную и предоставляет готовую, хорошо спроектированную архитектуру. Таким образом, Next.js становится де-факто стандартом разработки современных React-приложений. Это не просто библиотека, а фреймворк — а значит, он предлагает определённую «протоптанную дорожку», следование которой позволяет создавать более надёжные и поддерживаемые решения.

\subsubsection*{Auth.js}
\addcontentsline{toc}{subsubsection}{Auth.js}
Auth.js --- это библиотека для реализации аутентификации и авторизации в веб-приложениях. Она является официальным решением, рекомендуемым и поддерживаемым фреймворком Next.js, что гарантирует хорошую интеграцию и поддержку всех необходимых функций. Библиотека позволяет легко подключать сторонние провайдеры аутентификации, такие как Google, Facebook и другие, а также реализовывать собственную аутентификацию с использованием базы данных. Auth.js обеспечивает надежную защиту пользовательских данных, управление сессиями, работу с токенами и предоставляет удобные API для быстрой настройки. Это решение упрощает реализацию всех ключевых механизмов безопасности, освобождая разработчиков от необходимости погружаться в тонкости реализации.

\subsubsection*{Redux}
\addcontentsline{toc}{subsubsection}{Redux}
Redux --- это библиотека для управления состоянием в приложениях, основанных на React. Она используется для централизованного хранения состояния приложения, что облегчает обмен данными между компонентами и упрощает их взаимодействие. Redux помогает избежать "проблемы пропс-дерева" в больших приложениях, когда передача данных через множество вложенных компонентов становится сложной. Хотя Redux часто используется в более сложных приложениях, в данном проекте его роль заключается в том, чтобы сделать взаимодействие между компонентами более организованным и улучшить предсказуемость состояния приложения.



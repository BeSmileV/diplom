\subsection{Безопасность данных}
\subsubsection{Авторизация с использованием JWT и Auth.js}
Механизм авторизации в приложении реализован на основе JSON Web Token (JWT) и библиотеки Auth.js, которая обеспечивает безопасную и гибкую аутентификацию пользователей на стороне front-end. Процесс состоит из следующих этапов:

\begin{enumerate}
    \item \textbf{Аутентификация пользователя}:
    \begin{itemize}
        \item Пользователь выполняет вход с помощью логина/пароля или через одного из OAuth-провайдеров (например, Google).
        \item Auth.js инициирует процесс аутентификации и, при успешной проверке, получает JWT.
    \end{itemize}

    \item \textbf{Работа с токеном}:
    \begin{itemize}
        \item Полученный JWT содержит минимальный необходимый payload (например, идентификатор пользователя, роль и срок действия).
        \item Токен сохраняется в \texttt{HTTP-only} cookie с флагами \texttt{Secure} и \texttt{SameSite=Strict}, что предотвращает XSS- и CSRF-атаки.
    \end{itemize}

    \item \textbf{Доступ к защищённым ресурсам}:
    \begin{itemize}
        \item При обращении к API, front-end автоматически прикрепляет токен к запросу.
        \item При недействительном или истёкшем токене, Auth.js может автоматически обновить его через refresh-токен, если он присутствует.
    \end{itemize}
\end{enumerate}

\subsubsection{Роль Auth.js}
Библиотека Auth.js упрощает реализацию безопасной авторизации за счёт следующих возможностей:

\begin{enumerate}
    \item \textbf{Инкапсуляция логики}:
    \begin{itemize}
        \item Обрабатывает все основные сценарии авторизации (вход, выход, обновление токена).
        \item Управляет хранением и безопасной передачей токенов.
    \end{itemize}

    \item \textbf{Поддержка современных стандартов}:
    \begin{itemize}
        \item Генерирует CSRF-токены.
        \item Поддерживает стратегию Single Sign-On (SSO).
        \item Позволяет легко интегрировать внешние OAuth-провайдеры.
    \end{itemize}
\end{enumerate}

\subsubsection{Меры защиты}
Для повышения безопасности механизма авторизации реализованы дополнительные меры:

\begin{enumerate}
    \item \textbf{JWT}:
    \begin{itemize}
        \item Access-токен действует ограниченное время (например, 1 час), а refresh-токен — до 7 дней.
        \item Периодически обновляются ключи подписи токенов.
    \end{itemize}

    \item \textbf{Auth.js}:
    \begin{itemize}
        \item Валидирует параметры входа, включая redirect URI.
        \item Блокирует злоумышленников по IP при множественных неудачных попытках входа.
    \end{itemize}
\end{enumerate}

Таким образом, связка JWT и Auth.js позволяет реализовать надёжный и гибкий механизм авторизации на стороне front-end с минимальной утечкой чувствительных данных.

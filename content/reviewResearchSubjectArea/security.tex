\subsection{Безопасность данных}

В данном разделе приводится обзор ключевых принципов и практик, обеспечивающих конфиденциальность, целостность и доступность данных в приложении.

\subsubsection{Авторизация с использованием JWT и Auth.js}
Механизм авторизации в приложении реализован на основе JSON Web Token (JWT) и библиотеки Auth.js, которая обеспечивает безопасную и гибкую аутентификацию пользователей на клиентской стороне. Процесс состоит из следующих этапов:
\begin{enumerate}
  \item Аутентификация пользователя:
    \begin{itemize}
      \item Пользователь выполняет вход с помощью логина/пароля или через одного из OAuth-провайдеров (например, Google);
      \item Auth.js инициирует процесс аутентификации и, при успешной проверке, получает JWT.
    \end{itemize}
  \item Работа с токеном:
    \begin{itemize}
      \item Полученный JWT содержит минимальные необходимые полезные данные  (например, идентификатор пользователя, роль и срок действия);
      \item Токен сохраняется в \textit{HTTP-only} cookie с флагами \textit{Secure} и \textit{SameSite=Strict}, что предотвращает XSS- и CSRF-атаки.
    \end{itemize}
  \item Доступ к защищённым ресурсам:
    \begin{itemize}
      \item При обращении к серверной части автоматически прикрепляет токен к запросу;
      \item При недействительном или истёкшем токене Auth.js может автоматически обновить его через токен обновления, если он присутствует.
    \end{itemize}
\end{enumerate}

\subsubsection{Роль Auth.js}
Библиотека Auth.js упрощает реализацию безопасной авторизации за счёт следующих возможностей:
\begin{enumerate}
  \item Инкапсуляция логики:
    \begin{itemize}
      \item Обрабатывает все основные сценарии авторизации (вход, выход, обновление токена);
      \item Управляет хранением и безопасной передачей токенов.
    \end{itemize}
  \item Поддержка современных стандартов:
    \begin{itemize}
      \item Генерирует CSRF-токены;
      \item Поддерживает стратегию Single Sign-On (SSO).
    \end{itemize}
\end{enumerate}

\subsubsection{Меры защиты}
Для повышения безопасности механизма авторизации реализованы дополнительные меры:

\begin{enumerate}
  \item JWT: токен доступа действует ограниченное время (например, 5 минут), а токен обновления — 30 дней;
  \item Auth.js: Валидирует параметры входа.
\end{enumerate}

Таким образом, связка JWT и Auth.js позволяет реализовать надёжный и гибкий механизм авторизации на клиентской стороне с минимальной утечкой чувствительных данных.
\subsection{Система создания заданий}

В данном разделе описаны основные функциональные возможности анализа кода на основе искусственного интеллекта для преподавателей и студентов.

\subsubsection{Общее описание модуля}
Разработанная система позволяет преподавателям создавать виртуальные классрумы, выдавать задания и автоматически анализировать решения студентов с использованием инструментов искусственного интеллекта. Это аналог образовательной платформы (например, Google Classroom), ориентированный на технические дисциплины с программированием.

Основные функции:
\begin{itemize}
  \item Создание виртуального класса преподавателем;
  \item Назначение заданий с параметрами оценки;
  \item Загрузка решений студентами;
  \item Получение отчётов об автоматическом анализе кода на основе искусственного интеллекта;
  \item Просмотр статистики и аналитики преподавателем.
\end{itemize}

\subsubsection{Функциональные возможности}

Для преподавателей представлены следующие функциональные возможности: 

\begin{itemize}
  \item Создание и управление классами;
  \item Назначение заданий;
  \item Просмотр отчётов искусственного интеллекта по каждому студенту;
  \item Сводная статистика по группе.
\end{itemize}

У студентов возможностей меньше, а именно:

\begin{itemize}
  \item Просмотр активных заданий и дедлайнов;
  \item Загрузка решения.
\end{itemize}

\subsubsection{Анализ кода на основе искусственного интеллекта}
После загрузки решения студентом система автоматически выполняет его анализ с использованием инструментов искусственного интеллекта. Проверка охватывает корректность, читаемость, соответствие заданию и уровень оригинальности кода.

На основе заданных преподавателем критериев формируется интерактивный отчёт, включающий:
\begin{itemize}
  \item Комментарии и замечания по структуре и стилю кода;
  \item Оценку соответствия решению поставленным требованиям.
\end{itemize}

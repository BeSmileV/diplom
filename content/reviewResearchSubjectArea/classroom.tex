\subsection{Система создания заданий с AI-анализом кода}

В данном разделе описаны основные функциональные возможности AI-анализа кода для преподавателей и студентов.

\subsubsection{Общее описание модуля}
Разработанная система позволяет преподавателям создавать виртуальные классрумы, выдавать задания и автоматически анализировать решения студентов с использованием AI-инструментов. Это аналог образовательной платформы (например, Google Classroom), ориентированный на технические дисциплины с программированием.

Основные функции:
\begin{itemize}
  \item Создание виртуального класса преподавателем;
  \item Назначение заданий с параметрами оценки;
  \item Загрузка решений студентами;
  \item Получение отчётов об автоматическом AI-анализе кода;
  \item Просмотр статистики и аналитики преподавателем.
\end{itemize}

\subsubsection{Функциональные возможности}

\paragraph{Для преподавателей:}
\begin{itemize}
  \item Создание и управление классами;
  \item Назначение заданий;
  \item Просмотр AI-отчётов по каждому студенту;
  \item Сводная статистика по группе.
\end{itemize}

\paragraph{Для студентов:}
\begin{itemize}
  \item Просмотр активных заданий и дедлайнов;
  \item Загрузка решения.
\end{itemize}

\subsubsection{AI-анализ кода}
После загрузки решения студентом система автоматически выполняет его анализ с использованием инструментов искусственного интеллекта. Проверка охватывает корректность, читаемость, соответствие заданию и уровень оригинальности кода.

На основе заданных преподавателем критериев формируется интерактивный отчёт, включающий:
\begin{itemize}
  \item Комментарии и замечания по структуре и стилю кода;
  \item Оценку соответствия решению поставленным требованиям.
\end{itemize}

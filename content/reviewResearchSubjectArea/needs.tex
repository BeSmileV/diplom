\subsection{Потребности образовательной среды}

Современные образовательные процессы предъявляют высокие требования к функциональности учебных платформ. Для эффективного взаимодействия между преподавателями и студентами необходимо создавать приложения, которые обеспечивают организационную и техническую поддержку всех ключевых элементов образовательного процесса.

\subsubsection{Доступность материалов и заданий}
Материалы и задания должны быть доступны студентам в любое время. Приложение должно обеспечивать размещение учебных ресурсов в различных форматах (текст, видео, презентации) и упрощать их поиск и использование. Это позволяет студентам готовиться к занятиям и выполнять задания без привязки ко времени, а преподавателям — быстро обновлять и дополнять учебные модули.

\subsubsection{Автоматизация проверок и оценки}
Автоматизированная проверка заданий существенно ускоряет процесс получения обратной связи. Использование искусственного интеллекта для анализа кода позволяет выявлять ошибки, давать подсказки и оценивать работы без участия преподавателя. Это освобождает ресурсы преподавателя для индивидуальной поддержки студентов и более сложной экспертной оценки.

\subsubsection{Удобная система заданий и общения}
Приложение должно включать удобную систему создания и отслеживания заданий. Важно, чтобы преподаватели могли формулировать задания, прикреплять к ним материалы и получать результаты выполнения. Неотъемлемой частью также является возможность общения между участниками процесса — как в групповых, так и личных чатах, для обмена мнениями и получения поддержки.

\subsubsection{Интеграция всех процессов в одну систему}
Отдельные решения для чатов, размещения заданий и анализа кода создают фрагментированную среду. Необходима единая платформа, объединяющая все эти компоненты. Это упрощает взаимодействие, повышает удобство и эффективность обучения, а также снижает затраты на сопровождение и обучение работе с системой.

\subsubsection{Вывод}

Таким образом, при проектировании образовательной платформы следует учитывать потребности в постоянном доступе к материалам, автоматической проверке решений, поддержке взаимодействия и целостности функционала в рамках одного интерфейса.

\subsection{Коммуникация и взаимодействие}

Ключевым элементом платформы является система обмена сообщениями, включающая:
\begin{itemize}
  \item \textbf{Групповые чаты}:
    \begin{itemize}
      \item Общение участников учебной группы в режиме реального времени;
      \item Передача файлов (форматы: .jpg, .png, .mp4, .pdf, .zip и др.);
      \item Привязка к группам.
    \end{itemize}
  \item \textbf{Личная переписка}:
    \begin{itemize}
      \item Обмен сообщениями один на один (студент–преподаватель или студент–студент);
      \item Передача файлов тех же форматов, что и в групповых чатах.
    \end{itemize}
\end{itemize}

Обе системы поддерживают базовые функции:
\begin{itemize}
  \item Отображение истории сообщений;
  \item Индикаторы прочтения сообщений;
  \item Поиск по тексту переписки;
  \item Уведомления о новых сообщениях.
\end{itemize}

Таким образом, платформа предоставляет функциональность, охватывающую весь цикл учебной коммуникации — от административного управления структурами до взаимодействия по заданиям и материалам между преподавателями и студентами.

\subsection{Коммуникация и взаимодействие}

Ключевым элементом платформы является система обмена сообщениями, включающая следующие компоненты:

\begin{enumerate}
  \item Групповые чаты --- позволяют участникам учебной группы общаться в режиме реального времени, передавать файлы (поддерживаются форматы .jpg, .png, .mp4, .pdf, .zip и другие) и привязываются к конкретным группам;
  \item Личная переписка --- поддерживает обмен сообщениями один на один (между студентом и преподавателем либо между студентами), с возможностью передачи тех же форматов файлов, что и в групповых чатах;
  \item Отображение истории сообщений --- пользователи могут просматривать ранее отправленные и полученные сообщения;
  \item Индикаторы прочтения --- система отображает статус прочтения сообщений собеседником;
  \item Уведомления о новых сообщениях --- пользователи получают оповещения о входящих сообщениях в реальном времени.
\end{enumerate}

Таким образом, платформа предоставляет функциональность, охватывающую весь цикл учебной коммуникации.

\subsection{Введение в предметную область}

Современные образовательные процессы переживают значительные изменения под воздействием цифровых технологий. В условиях быстрого роста объёмов информации и перехода на дистанционное и смешанное обучение возникает потребность в создании платформ, которые объединяют различные образовательные инструменты в единую систему. Проблемы, с которыми сталкиваются преподаватели и студенты, включают фрагментацию существующих решений: чаты для общения, отдельные системы для размещения и проверки заданий, а также инструменты для анализа решений студентов.

Существующие платформы не всегда обеспечивают интеграцию всех этих функций в одном приложении, что приводит к необходимости использования множества разных сервисов для выполнения учебных задач. В рамках образовательных процессов это усложняет взаимодействие между преподавателями и студентами, увеличивает время на организацию обучения и снижает его эффективность.

Одной из важнейших задач является создание платформы, которая объединяет все эти компоненты в одном месте, обеспечивая удобный интерфейс для студентов и преподавателей. Такая система должна включать:
\begin{itemize}
  \item возможность создания и размещения учебных заданий,
  \item автоматическое тестирование решений студентов с использованием ИИ для проверки правильности кода,
  \item чат-функциональность для общения студентов с преподавателями и внутри групп,
  \item централизованный доступ к учебным материалам.
\end{itemize}

Интеграция всех этих функций в одну платформу позволит значительно упростить организацию учебного процесса, улучшить взаимодействие между преподавателями и студентами, а также повысить качество обучения за счёт автоматизации рутинных задач.

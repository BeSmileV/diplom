\subsection{ИИ-анализ кода на основе DeepSeek}

\subsubsection{Общее описание модуля DeepSeek}
DeepSeek — это современный облачный сервис для глубокого анализа исходного кода, построенный на основе передовых трансформерных моделей и нейронных сетей. Модуль предназначен для того, чтобы автоматизировать трудоёмкий процесс проверки решений студентов, снизить субъективность оценок и предоставить преподавателям максимально детализированную и качественную обратную связь. 

Платформа DeepSeek способна:
\begin{itemize}
	\item Выявлять самые разнообразные синтаксические и логические ошибки, обнаруживать неточности в реализации алгоритмов;
	\item Оценивать соответствие структурных блоков кода требованиям конкретного задания, обращая внимание на правильность использования функций и корректность их связи;
	\item Фиксировать стилевые отклонения и «code smells», которые могут усложнить поддержку и дальнейшее развитие проекта;
	\item Анализировать степень оригинальности решения с помощью семантического сравнения embedding‐представлений, что позволяет не только обнаружить плагиат, но и оценить творческий подход студента.
\end{itemize}

Доступ к вызовам DeepSeek строго ограничен: только преподаватель, работая в защищённом интерфейсе фронтенда, может инициировать анализ кода. Студенты видят лишь результат — отчёт с пояснениями и рекомендациями — без какой-либо информации о внутреннем устройстве сервиса.

\subsubsection{Функциональные возможности DeepSeek}

\paragraph{Для преподавателей:}
\begin{itemize}
	\item Возможность запускать анализ кода из единого интерфейса, не покидая окна браузера;
	\item Автоматически сгенерированные отчёты, где каждая найденная проблема снабжена подробным описанием и примером исправления;
	\item Графическое отображение метрик: сложность, уровень стиля, количество «code smells», показатели производительности;
	\item Механизм гибкой настройки критериев оценки — преподаватель может добавлять свои правила проверки в зависимости от характера задания;
\end{itemize}

\paragraph{Для студентов:}
\begin{itemize}
	\item Заявка на анализ через загрузку решения;
	\item Получение готового отчёта от преподавателя без прямого доступа к сервису.
\end{itemize}

\subsubsection{Принцип работы DeepSeek}
\begin{enumerate}
	\item \textbf{Лексический и синтаксический разбор:} исходный код разбивается на токены, строится абстрактное синтаксическое дерево, на основе которого проводится первичный анализ.
	\item \textbf{Семантический анализ:} алгоритмы трансформеров сопоставляют логику решения с огромной базой примеров, выявляя нетривиальные отклонения от оптимальной структуры.
	\item \textbf{Расчет метрик:} рассчитываются показатели цикломатической сложности, глубины вложенности, соответствия coding standard и других параметров качества.
	\item \textbf{Антиплагиатный компонент:} сравнение embedding‐векторов загруженного решения с репозиторием эталонных и ранее проверенных работ для определения степени оригинальности.
	\item \textbf{Формирование отчёта:} на выходе генерируется детальный JSON-документ, содержащий список найденных замечаний, метрик и чётких рекомендаций по улучшению.
\end{enumerate}


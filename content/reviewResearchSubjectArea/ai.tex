\subsection{Анализ текста программ на основе DeepSeek}

\subsubsection{Общее описание модуля DeepSeek}

DeepSeek —  это облачная платформа для глубинного анализа текста программ, основанная на передовых архитектурах трансформеров и нейронных сетей~\cite{deepseek_docs}. Ее учебный модуль автоматизирует ручную проверку студенческих работ, минимизирует субъективность оценивания и предоставляет преподавателям детальную, содержательную обратную связь.

\begin{enumerate}
  \item Выявлять самые разнообразные синтаксические и логические ошибки, обнаруживать неточности в реализации алгоритмов;
  \item Оценивать соответствие структурных блоков текста программы требованиям конкретного задания, обращая внимание на правильность использования функций и корректность их связи;
  \item Анализировать степень оригинальности решения с помощью семантического сравнения embedding‐представлений, что позволяет не только обнаружить плагиат, но и оценить творческий подход студента.
\end{enumerate}

Доступ к вызовам DeepSeek строго ограничен: напрямую вызвать проверку текста программы искусственным интеллектом нельзя --- проверка доступна только при отправке решения студентом.

\subsubsection{Функциональные возможности DeepSeek}

Общий функционал разделён на две части: для преподавателей и для студентов. Для преподаватель:

\begin{enumerate}
    \item Автоматически сгенерированные отчёты, где каждая найденная проблема снабжена подробным описанием и примером исправления;
    \item Механизм гибкой настройки критериев оценки — преподаватель может добавлять свои правила проверки в зависимости от характера задания.
\end{enumerate}

Студент, в свою очередь, имеет следующие возможности:

\begin{enumerate}
    \item Заявка на анализ через загрузку решения;
    \item Получение готового отчёта от преподавателя без прямого доступа к сервису.
\end{enumerate}

\subsubsection{Принцип работы DeepSeek}

\begin{enumerate}
  \item Во время лексического и синтаксического разбора текст программы разбивается на токены, строится абстрактное синтаксическое дерево, на основе которого проводится первичный анализ;
  \item Алгоритмы трансформеров сопоставляют логику решения с огромной базой примеров, выявляя нетривиальные отклонения от оптимальной структуры;
  \item Рассчитываются показатели цикломатической сложности, глубины вложенности, соответствия стандартам написания текста программы и других параметров качества;
  \item Сравнение векторов признаков загруженного решения с репозиторием эталонных и ранее проверенных работ для определения степени оригинальности;
  \item На выходе генерируется детальный JSON-документ, содержащий список найденных замечаний, метрик и чётких рекомендаций по улучшению.
\end{enumerate}

\subsection*{Выводы}
\addcontentsline{toc}{subsection}{Выводы по результатам анализа предметной области}

В результате проведённого анализа можно сделать вывод о наличии устойчивого запроса на интегрированное образовательное приложение, способное решать сразу несколько ключевых задач. Современные платформы зачастую фокусируются либо на предоставлении учебных материалов, либо на коммуникации, либо на автоматизации проверки знаний, при этом разрозненность этих функций создаёт неудобства для всех участников образовательного процесса.

Потребности преподавателей включают в себя удобное управление группами и заданиями, возможность оперативной обратной связи, загрузку и распространение материалов. Студентам, в свою очередь, важно иметь стабильный и понятный доступ к заданиям, личным сообщениям и учебным ресурсам, а также возможность взаимодействовать с преподавателями и одногруппниками в привычном цифровом формате.

Предлагаемое приложение должно закрыть этот разрыв, обеспечив единую среду, в которой объединены функции управления учебным процессом, общения, публикации и проверки заданий. Такой подход позволит повысить качество образовательного взаимодействия, сократить технические барьеры и обеспечить более высокую степень вовлечённости пользователей.

Таким образом, на основании проведённого анализа подтверждается необходимость разработки новой системы, в которой ключевые элементы образовательной среды будут интегрированы в одно приложение, удовлетворяющее современным требованиям пользователей.

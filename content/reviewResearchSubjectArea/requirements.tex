\subsection*{Требования к функциональности приложения}
\addcontentsline{toc}{subsection}{Требования к функциональности приложения}

Разрабатываемое приложение представляет собой образовательную платформу, ориентированную на университетскую среду. Основная цель — предоставить единое пространство для организации учебного процесса, взаимодействия между преподавателями и студентами, а также управления учебными структурами.

\subsubsection*{Регистрация и структура университетов}
\addcontentsline{toc}{subsubsection}{Регистрация и структура университетов}
Каждый университет имеет возможность зарегистрироваться на платформе и получить доступ к собственной административной панели. Через неё администраторы могут создавать внутреннюю структуру: институты, кафедры и учебные группы. Эти сущности используются как основа для управления доступом, назначения преподавателей и приглашения студентов.

\subsubsection*{Панель преподавателя}
\addcontentsline{toc}{subsubsection}{Панель преподавателя}
Преподаватели, закреплённые за кафедрами, получают доступ ко всем учебным группам, относящимся к соответствующей кафедре. Через панель преподавателя доступен следующий функционал:
\begin{itemize}
  \item создание групповых чатов для любой группы своей кафедры;
  \item размещение учебных материалов — как в групповых чатах, так и в личных сообщениях;
  \item формирование и отправка заданий для студентов;
  \item просмотр и анализ результатов выполнения заданий;
  \item предоставление обратной связи студентам.
\end{itemize}

\subsubsection*{Панель студента}
\addcontentsline{toc}{subsubsection}{Панель студента}
Студенты, присоединенные к определенным группам, имеют доступ к:
\begin{itemize}
  \item групповым чатам своей учебной группы;
  \item личной переписке с преподавателями;
  \item материалам, отправленным преподавателями;
  \item заданиям, опубликованным в рамках их группы;
  \item форме отправки решений и получению обратной связи.
\end{itemize}

\subsubsection*{Коммуникация и взаимодействие}
\addcontentsline{toc}{subsubsection}{Коммуникация и взаимодействие}
Ключевым элементом платформы является внутренняя система коммуникации, которая позволяет преподавателям и студентам эффективно взаимодействовать друг с другом. Предусмотрены как групповые чаты, привязанные к учебным группам, так и возможность отправки личных сообщений.

Таким образом, платформа предоставляет функциональность, охватывающую весь цикл учебной коммуникации — от административного управления структурами до взаимодействия по заданиям и материалам между преподавателями и студентами.
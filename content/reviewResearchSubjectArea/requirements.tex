\subsection{Требования к функциональности приложения}

Разрабатываемое приложение представляет собой образовательную платформу, ориентированную на университетскую среду. Основная цель~--- предоставить единое пространство для организации учебного процесса, взаимодействия между преподавателями и студентами, а также управления учебными структурами.

Каждый университет может зарегистрироваться на платформе и получить доступ к собственной административной панели. Через неё администраторы создают внутреннюю структуру, включающую институты, кафедры и учебные группы. Эти сущности служат основой для распределения доступа, назначения преподавателей и приглашения студентов.

Преподаватели, закреплённые за определёнными кафедрами, получают доступ ко всем учебным группам соответствующего подразделения. Через личную панель преподавателя реализован следующий функционал:
\begin{enumerate}
  \item Создание групповых чатов для любой группы своей кафедры;
  \item Размещение учебных материалов — как в групповых чатах, так и в личных сообщениях;
  \item Формирование и отправка заданий для студентов;
  \item Просмотр и анализ результатов выполнения заданий;
  \item Предоставление обратной связи студентам.
\end{enumerate}

Студенты, присоединённые к учебным группам, имеют доступ к следующему функционалу:
\begin{enumerate}
  \item Групповым чатам своей учебной группы;
  \item Личной переписке с преподавателями;
  \item Материалам, отправленным преподавателями;
  \item Заданиям, опубликованным в рамках их группы;
  \item Форме отправки решений и получению обратной связи.
\end{enumerate}

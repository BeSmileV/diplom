\subsection{Анализ существующих образовательных платформ}

Современные образовательные платформы, такие как Moodle~\cite{moodle_docs}, Google Classroom~\cite{google_classroom}, Microsoft Teams для образования~\cite{microsoft_teams_education}, а также специализированные решения, предназначенные для работы с программированием, предлагают различные функциональные возможности для взаимодействия преподавателей и студентов. Однако каждая из этих платформ имеет свои ограничения и не всегда покрывает все потребности в рамках единой системы.

\subsubsection{Система управления обучением Moodle}
Moodle является одной из самых популярных образовательных платформ, используемых во многих учебных заведениях~\cite{moodle_docs}. Она предоставляет инструменты для размещения учебных материалов, организации тестов и заданий, а также ведения онлайн-курсов. Однако, несмотря на свои возможности, Moodle не предоставляет встроенных решений для автоматической проверки текста программы студентов, а также не включает в себя продвинутые механизмы общения в реальном времени, что делает её менее эффективной для динамичного взаимодействия в процессе обучения.

\subsubsection{Платформа дистанционного обучения Google Classroom}
Google Classroom предлагает простоту в использовании и позволяет интегрировать различные Google сервисы~\cite{google_classroom}. Платформа позволяет преподавателям создавать задания, прикреплять материалы и отслеживать выполнение студентами. Однако Google Classroom не предоставляет функциональности для автоматического анализа решений, особенно в контексте программирования. Это требует интеграции с внешними инструментами, что усложняет использование системы в образовательных учреждениях.

\subsubsection{Платформа для онлайн-обучения Microsoft Teams}
Microsoft Teams, в отличие от Moodle и Google Classroom, активно используется для организации видеоконференций и групповых чатов~\cite{microsoft_teams_education}. Он позволяет преподавателям и студентам взаимодействовать в реальном времени, а также интегрирует различные сервисы Microsoft 365. Однако, как и в случае с другими платформами, Microsoft Teams не предоставляет функционала для интегрированного анализа текста программы студентов с использованием искусственного интеллекта, что ограничивает его возможности в обучении программированию.

\subsubsection{Платформы для анализа текста программы}
Существуют специализированные платформы, такие как CodeSignal~\cite{codesignal_wiki}, Codility~\cite{codility_wiki} и LeetCode~\cite{leetcode_wiki}, которые позволяют преподавателям и работодателям тестировать навыки программирования студентов. Эти системы используют алгоритмы для автоматической проверки решений, однако они ограничены в функционале взаимодействия с преподавателями и студентами, а также не обеспечивают централизованный доступ к учебным материалам и заданиям.

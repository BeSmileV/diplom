\subsection{Архитектура по Feature-Sliced Design}

Feature-Sliced Design (FSD) — это гибкий набор рекомендаций и подходов по логической организации клиентского кода, основанных на выделении независимых функциональных слоёв и зон ответственности. В отличие от строгих стандартов, FSD предоставляет разработчикам свободу выбора конкретных решений, сохраняя при этом единый общий каркас структуры. Такая архитектура повышает читаемость, масштабируемость и тестируемость приложения, а также упрощает командную разработку и поддержку кода.

\subsubsection{Слой \texttt{shared}}

Слой \texttt{shared} служит хранилищем нижнего уровня для общих и переиспользуемых компонентов, утилит и ресурсов, не зависящих от конкретной бизнес-логики:

\begin{itemize}
	\item \textbf{UI-компоненты общего назначения}: простые React-компоненты (кнопки, лоадеры, таблицы, модальные окна), которые не содержат бизнес-логики и могут использоваться в различных контекстах;
	\item \textbf{Провайдеры контекста}: \texttt{DragAndDropFilesProvider}, \texttt{EnterKeyHandlerProvider}, а также другие контексты, обеспечивающие единообразную работу с событиями и состояниями на уровне всего приложения;
	\item \textbf{Утилиты и хелперы}: функции для форматирования дат и чисел, генерации уникальных идентификаторов, работы с localStorage и пр.;
	\item \textbf{Кастомные хуки общего назначения}: \texttt{useSearchParamsListener}, \texttt{useWindowSize}, \texttt{usePreviousValue} и другие, позволяющие сократить дублирование кода;
	\item \textbf{SVG-иконки и графика}: импорт через SVGR для единообразного подключения и управления атрибутами SVG;
	\item \textbf{Компоненты навигации и управления состоянием}: \texttt{PaginationComponent} для управления пагинацией через URL-параметры, \texttt{Breadcrumbs}, \texttt{Tabs} и т. д.
\end{itemize}

Каждый элемент слоя \texttt{shared} сопровождается подробной документацией и примерами использования, что ускоряет обучение новых участников команды и снижает риски ошибок.

\subsubsection{Слой \texttt{entities}}

Слой \texttt{entities} отвечает за интеграцию с внешними сервисами и описывает доменные модели:

\begin{itemize}
	\item \texttt{api/}: тонкий слой-абстракция над HTTP-клиентами (\texttt{fetch}/\texttt{axios}), где функции названы в соответствии с операциями Swagger/OpenAPI (например, \texttt{getUserProfile}, \texttt{createOrder});
	\item \texttt{types/}: TypeScript-интерфейсы и типы для запросов и ответов (E.g. \texttt{UserProfileResponseType}, \texttt{OrderCreateRequestBodyType});
	\item \texttt{models/}: классы и мапперы (mapper) для преобразования сырых данных из API в объекты, удобные для работы в приложении;
	\item \texttt{services/}: дополнительные обёртки для работы с локальным кэшем (IndexedDB, localStorage) или реализации retry-логики и таймаутов;
\end{itemize}

Кроме того, здесь же располагаются функции для обработки ошибок и логирования, а также константы (URL-ы, ключи хранящихся данных) для единообразия использования.

\subsubsection{Слои \texttt{features}, \texttt{widgets}, \texttt{pages}}

Главные рабочие слои приложения, отвечающие за реализацию конкретной функциональности:

\begin{enumerate}
	\item \textbf{\texttt{pages}}: маршрутизация и верхний уровень страницы. Здесь описываются пути, разрешения доступа (guards), асинхронная загрузка данных на уровне страницы и выбор необходимого набора виджетов;
	\item \textbf{\texttt{widgets}}: презентационные и «умные» компоненты, разделённые по Smart/Presentational-паттерну:
	\begin{itemize}
		\item \emph{Presentational}: исключительно UI и пропсы, без прямой работы с API и глобальным состоянием;
		\item \emph{Smart}: обёртки, получающие данные через хуки из \texttt{features}, передающие результат в презентационные компоненты;
	\end{itemize}
	\item \textbf{\texttt{features}}: бизнес-логика и состояние, реализованное в виде кастомных хуков, функций-редьюсеров, слайсов Redux или Zustand:
	\begin{itemize}
		\item \texttt{hooks.ts}: главный хук-фабрика (например, \texttt{useRegistrationUniversity});
		\item \texttt{schema.ts}: схемы валидации через \texttt{Yup} или \texttt{Zod};
		\item \texttt{utils.ts}: вспомогательные функции и вычисляемые селекторы;
		\item \texttt{store.ts}: подключение к Redux/Redux Toolkit или настройка Zustand;
	\end{itemize}
\end{enumerate}

Пример процесса разработки новой функции: сначала добавляем маршрут и контекст в \texttt{pages}, затем создаём презентационный виджет, выносим всю логику в хук в \texttt{features}, настраиваем валидацию и состояние, и наконец интегрируем с API через слой \texttt{entities}.

\subsubsection{Пример: RegistrationUniversity}

\begin{lstlisting}[breaklines=true,caption=RegistrationUniversityWidget]
	// Presentation-компонент
	export function RegistrationUniversityWidget() {
		const { formDataRef, isError, setIsError, onSubmit, isLoading } =
		useRegistrationUniversity();
		
		return (
			<div>...</div>
		);
	}
\end{lstlisting}

\begin{lstlisting}[breaklines=true,caption=useRegistrationUniversity]
	// Smart-компонент: хук-фабрика
	export function useRegistrationUniversity() {
		const formDataRef = useRef<RegistrationData>();
		const [isError, setIsError] = useState<string[]>([]);
		const [isLoading, setIsLoading] = useState(false);
		const router = useRouter();
		
		const onSubmit = async () => {
			...
		};
		return { formDataRef, isError, setIsError, onSubmit, isLoading };
	}
\end{lstlisting}

\subsubsection{Соглашения по неймингу и структуре}

Для поддержания высокого уровня единого стиля и предсказуемости структуры проекта:

\begin{itemize}
	\item Имена функций API и типов дублируют бекенд-операции;
	\item Компоненты и хуки получают префиксы по зоне ответственности: \texttt{RegistrationForm}, \texttt{useFilesUpload} и т. д.;
	\item В каждом каталоге \texttt{features/FeatureName} обязателен минимум файлов: \texttt{index.ts}, \texttt{hooks.ts}, \texttt{schema.ts}, \texttt{utils.ts};
	\item Все слои должны быть описаны в README с диаграммой и ссылками на примеры использования;
	\item Код ревью включает проверку соответствия FSD-подходу и правил TypeScript;
\end{itemize}

\subsection{Архитектура по Feature-Sliced Design}

Feature-Sliced Design (FSD) — это гибкий набор рекомендаций и подходов по логической организации клиентского кода, основанных на выделении независимых функциональных слоёв и зон ответственности. В отличие от строгих стандартов, FSD предоставляет разработчикам свободу выбора конкретных решений, сохраняя при этом единый общий каркас структуры. Такая архитектура повышает читаемость, масштабируемость и тестируемость приложения, а также упрощает командную разработку и поддержку кода.

\subsubsection{Слой \texttt{shared}}

Слой \texttt{shared} служит хранилищем нижнего уровня для общих и переиспользуемых компонентов, утилит и ресурсов, не зависящих от конкретной бизнес-логики:
\begin{itemize}
  \item \textbf{UI-компоненты общего назначения}: простые React-компоненты (кнопки, лоадеры, таблицы, модальные окна), не содержащие бизнес-логику и используемые в различных контекстах;
  \item \textbf{Провайдеры контекста}: \texttt{DragAndDropFilesProvider}, \texttt{EnterKeyHandlerProvider} и другие, обеспечивающие единообразную работу с событиями и состояниями по всему приложению;
  \item \textbf{Утилиты и хелперы}: функции для форматирования дат и чисел, генерации уникальных идентификаторов, работы с \texttt{localStorage} и пр.;
  \item \textbf{Кастомные хуки общего назначения}: \texttt{useSearchParamsListener}, \texttt{useWindowSize}, \texttt{usePreviousValue} и другие, сокращающие дублирование кода;
  \item \textbf{SVG-иконки и графика}: импорт через SVGR для единообразного подключения и управления атрибутами SVG;
  \item \textbf{Компоненты навигации и управления состоянием}: \texttt{PaginationComponent} для пагинации через URL-параметры, \texttt{Breadcrumbs}, \texttt{Tabs} и т. д.
\end{itemize}

\subsubsection{Слой \texttt{entities}}

Слой \texttt{entities} отвечает за интеграцию с внешними сервисами и описывает доменные модели:
\begin{itemize}
  \item \texttt{api/}: тонкий слой-абстракция над HTTP-клиентами (\texttt{fetch}/\texttt{axios}), где функции названы в соответствии с операциями Swagger/OpenAPI (например, \texttt{getUserProfile}, \texttt{createOrder});
  \item \texttt{types/}: TypeScript-интерфейсы и типы для запросов и ответов (например, \texttt{UserProfileResponseType}, \texttt{OrderCreateRequestBodyType});
  \item \texttt{models/}: классы и mapper-функции для преобразования сырых данных из API в удобные объекты;
  \item \texttt{services/}: обёртки для работы с локальным кэшем (IndexedDB, \texttt{localStorage}) и реализации retry-логики и таймаутов.
\end{itemize}

\subsubsection{Слои \texttt{features}, \texttt{widgets}, \texttt{pages}}

Главные рабочие слои приложения, отвечающие за реализацию конкретной функциональности:
\begin{enumerate}
  \item \textbf{\texttt{pages}}: маршрутизация и верхний уровень страниц, описывающий пути, guards для доступа, асинхронную загрузку данных и выбор виджетов;
  \item \textbf{\texttt{widgets}}: презентационные и «умные» компоненты по Smart/Presentational-паттерну:
    \begin{itemize}
      \item \emph{Presentational}: исключительно UI и пропсы, без работы с API и глобальным состоянием;
      \item \emph{Smart}: обёртки, получающие данные через хуки из \texttt{features} и передающие их в презентационные компоненты.
    \end{itemize}
  \item \textbf{\texttt{features}}: бизнес-логика и состояние в виде кастомных хуков, редьюсеров, слайсов Redux или Zustand:
    \begin{itemize}
      \item \texttt{hooks.ts}: главный хук-фабрика (например, \texttt{useRegistrationUniversity});
      \item \texttt{schema.ts}: схемы валидации через \texttt{Yup} или \texttt{Zod};
      \item \texttt{utils.ts}: вспомогательные функции и селекторы;
      \item \texttt{store.ts}: подключение к Redux/Redux Toolkit или настройка Zustand.
    \end{itemize}
\end{enumerate}

\subsubsection{Пример: RegistrationUniversity}

Ниже приведён пример реализации виджета и соответствующего хука (см. листинги~\ref{lst:registration-university-widget} и~\ref{lst:use-registration-university}):

\begin{lstlisting}[breaklines=true,caption=RegistrationUniversityWidget,label=lst:registration-university-widget]
  // Presentation-компонент
  export function RegistrationUniversityWidget() {
    const { formDataRef, isError, setIsError, onSubmit, isLoading } =
      useRegistrationUniversity();

    return (
      <div>...</div>
    );
  }
\end{lstlisting}

\begin{lstlisting}[breaklines=true,caption=useRegistrationUniversity,label=lst:use-registration-university]
  // Smart-компонент: хук-фабрика
  export function useRegistrationUniversity() {
    const formDataRef = useRef<RegistrationData>();
    const [isError, setIsError] = useState<string[]>([]);
    const [isLoading, setIsLoading] = useState(false);
    const router = useRouter();

    const onSubmit = async () => {
      ...
    };
    return { formDataRef, isError, setIsError, onSubmit, isLoading };
  }
\end{lstlisting}

\subsubsection{Соглашения по неймингу и структуре}

Для поддержания единого стиля и предсказуемости структуры проекта:
\begin{itemize}
  \item Имена функций API и типов дублируют backend-операции;
  \item Компоненты и хуки получают префиксы по зоне ответственности (\texttt{RegistrationForm}, \texttt{useFilesUpload} и т. д.);
  \item В каждом каталоге \texttt{features/FeatureName} обязателен минимум файлов: \texttt{index.ts}, \texttt{hooks.ts}, \texttt{schema.ts}, \texttt{utils.ts};
  \item Все слои описаны в README с диаграммой и примерами использования;
  \item Код-ревью включает проверку соответствия FSD-подходу и правилам TypeScript.
\end{itemize}

\subsection{Архитектура по Feature-Sliced Design}

Feature-Sliced Design (FSD) — это гибкий набор рекомендаций и подходов по логической организации клиентского части программы, основанных на выделении независимых функциональных слоёв и зон ответственности. В отличие от строгих стандартов, FSD предоставляет разработчикам свободу выбора конкретных решений, сохраняя при этом единый общий каркас структуры. Такая архитектура повышает читаемость, масштабируемость и тестируемость приложения, а также упрощает командную разработку и поддержку приложения.

\subsubsection{Слой \textit{shared}}

Слой \textit{shared} служит хранилищем нижнего уровня для общих и переиспользуемых компонентов, утилит и ресурсов, не зависящих от конкретной бизнес-логики и состоит из следующих частей:

\begin{enumerate}
  \item UI-компоненты общего назначения включают простые React-компоненты (кнопки, лоадеры, таблицы, модальные окна), не содержащие бизнес-логику и используемые в различных контекстах.
  \item Провайдеры контекста, такие как \textit{DragAndDropFilesProvider} и \textit{EnterKeyHandlerProvider}, обеспечивают единообразную работу с событиями и состояниями по всему приложению.
  \item Утилиты и хелперы включают функции для форматирования дат и чисел, генерации уникальных идентификаторов, работы с \textit{localStorage} и другие вспомогательные средства.
  \item Кастомные хуки общего назначения (\textit{useSearchParamsListener}, \textit{useWindowSize}, \textit{usePreviousValue} и др.) позволяют сократить дублирование текста программы и упрощают работу с состоянием.
  \item SVG-иконки и графика подключаются через SVGR, что обеспечивает единообразие отображения и управление атрибутами SVG.
  \item Компоненты навигации и управления состоянием, включая \textit{PaginationComponent}, \textit{Breadcrumbs}, \textit{Tabs} и другие, обеспечивают взаимодействие с URL-параметрами и организацию структуры интерфейса.
\end{enumerate}
\subsubsection{Слой \textit{entities}}

Слой \textit{entities} отвечает за интеграцию с внешними сервисами.
\begin{enumerate}
  \item \textit{api/}: тонкий слой-абстракция над HTTP-клиентами (\textit{fetch}/\textit{axios}), где функции названы в соответствии с операциями Swagger/OpenAPI (например, \textit{getUserProfile}, \textit{createOrder});
  \item \textit{types/}: TypeScript-интерфейсы и типы для запросов и ответов (например, \textit{UserProfileResponseType}, \textit{OrderCreateRequestBodyType});
  \item \textit{models/}: классы и mapper-функции для преобразования сырых данных из API в удобные объекты;
  \item \textit{services/}: обёртки для работы с локальным кэшем (IndexedDB, \textit{localStorage}) и реализации retry-логики и таймаутов.
\end{enumerate}

\subsubsection{Слои \textit{features}, \textit{widgets}, \textit{pages}}

Главные рабочие слои приложения, отвечающие за реализацию конкретной функциональности:
\begin{enumerate}
  \item Слой \textit{pages} отвечает за маршрутизация и верхний уровень страниц, описывающий пути, guards для доступа, асинхронную загрузку данных и выбор виджетов.
  \item Слой \textit{widgets} содержит презентационные и «умные» компоненты по паттерну Smart/Presentational. Презентационный компонент (\textit{Presentational Component}) отвечает исключительно за UI и принимает данные через пропсы, не взаимодействуя напрямую с API или глобальным состоянием. Умный хук (\textit{Smart Hook}) инкапсулирует бизнес-логику и передаёт необходимые данные в презентационные компоненты.
  \item Слой \textit{features} реализует бизнес-логику и управление состоянием с помощью собственных хуков, редьюсеров и слайсов Redux. В рамках этого слоя используются следующие структуры: файл \textit{hooks.ts} содержит главный хук-фабрику (например, \textit{useRegistrationUniversity}); файл \textit{schema.ts} описывает схемы форм; файл \textit{utils.ts} содержит вспомогательные функции и селекторы; файл \textit{store.ts} подключает функциональность к библиотек Redux или Redux Toolkit.
\end{enumerate}

\subsubsection{Пример виджета RegistrationUniversity}

В листинге~\ref{lst:registration-university-widget} представлен презентационный компонент, отвечающий за отображение формы регистрации университета.

\begin{lstlisting}[breaklines=true,caption=RegistrationUniversityWidget,label=lst:registration-university-widget]
  // Presentation-компонент
  export function RegistrationUniversityWidget() {
    const { formDataRef, isError, setIsError, onSubmit, isLoading } =
      useRegistrationUniversity();

    return (
      <div>...</div>
    );
  }
\end{lstlisting}

В листинге~\ref{lst:use-registration-university} представлена функция, содержащий логику взаимодействия пользователя с формой регистрации университета и обработки отправки запроса на регистрацию.

\begin{lstlisting}[breaklines=true,caption=useRegistrationUniversity,label=lst:use-registration-university]
  // Smart-компонент: хук-фабрика
  export function useRegistrationUniversity() {
    const formDataRef = useRef<RegistrationData>();
    const [isError, setIsError] = useState<string[]>([]);
    const [isLoading, setIsLoading] = useState(false);
    const router = useRouter();

    const onSubmit = async () => {
      ...
    };
    return { formDataRef, isError, setIsError, onSubmit, isLoading };
  }
\end{lstlisting}

\subsubsection{Соглашения по именованию и структуре}

Для поддержания единого стиля и предсказуемости структуры проекта:
\begin{enumerate}
  \item Имена функций и типов для взаимодействия с серверной частью совпадают;
  \item Компоненты и хуки получают префиксы по зоне ответственности (\textit{RegistrationForm}, \textit{useFilesUpload} и т. д.);
  \item В каждом каталоге \textit{features/FeatureName} обязателен минимум файлов: \textit{index.ts} и сегмент;
  \item Названия типов и функций для взаимодействия с серверной частью должны содержать информацию о типе запроса и типе данных (тело запроса или возвращаемые данные).
\end{enumerate}

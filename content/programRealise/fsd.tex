\subsection{Архитектура по Feature-Sliced Design}

Feature-Sliced Design (FSD) — это гибкий набор рекомендаций и подходов по логической организации клиентского кода, основанных на выделении независимых функциональных слоёв и зон ответственности. В отличие от строгих стандартов, FSD предоставляет разработчикам свободу выбора конкретных решений, сохраняя при этом единый общий каркас структуры. Такая архитектура повышает читаемость, масштабируемость и тестируемость приложения, а также упрощает командную разработку и поддержку кода.

\subsubsection{Слой \textit{shared}}

Слой \textit{shared} служит хранилищем нижнего уровня для общих и переиспользуемых компонентов, утилит и ресурсов, не зависящих от конкретной бизнес-логики:
\begin{itemize}
  \item UI-компоненты общего назначения: простые React-компоненты (кнопки, лоадеры, таблицы, модальные окна), не содержащие бизнес-логику и используемые в различных контекстах;
  \item Провайдеры контекста: \textit{DragAndDropFilesProvider}, \textit{EnterKeyHandlerProvider} и другие, обеспечивающие единообразную работу с событиями и состояниями по всему приложению;
  \item Утилиты и хелперы: функции для форматирования дат и чисел, генерации уникальных идентификаторов, работы с \textit{localStorage} и пр.;
  \item Кастомные хуки общего назначения: \textit{useSearchParamsListener}, \textit{useWindowSize}, \textit{usePreviousValue} и другие, сокращающие дублирование кода;
  \item SVG-иконки и графика: импорт через SVGR для единообразного подключения и управления атрибутами SVG;
  \item Компоненты навигации и управления состоянием: \textit{PaginationComponent} для пагинации через URL-параметры, \textit{Breadcrumbs}, \textit{Tabs} и т. д.
\end{itemize}

\subsubsection{Слой \textit{entities}}

Слой \textit{entities} отвечает за интеграцию с внешними сервисами и описывает доменные модели:
\begin{itemize}
  \item \textit{api/}: тонкий слой-абстракция над HTTP-клиентами (\textit{fetch}/\textit{axios}), где функции названы в соответствии с операциями Swagger/OpenAPI (например, \textit{getUserProfile}, \textit{createOrder});
  \item \textit{types/}: TypeScript-интерфейсы и типы для запросов и ответов (например, \textit{UserProfileResponseType}, \textit{OrderCreateRequestBodyType});
  \item \textit{models/}: классы и mapper-функции для преобразования сырых данных из API в удобные объекты;
  \item \textit{services/}: обёртки для работы с локальным кэшем (IndexedDB, \textit{localStorage}) и реализации retry-логики и таймаутов.
\end{itemize}

\subsubsection{Слои \textit{features}, \textit{widgets}, \textit{pages}}

Главные рабочие слои приложения, отвечающие за реализацию конкретной функциональности:
\begin{enumerate}
  \item \textit{pages}: маршрутизация и верхний уровень страниц, описывающий пути, guards для доступа, асинхронную загрузку данных и выбор виджетов;
  \item \textit{widgets}: презентационные и «умные» компоненты по Smart/Presentational-паттерну:
    \begin{itemize}
      \item \emph{Presentational}: исключительно UI и пропсы, без работы с API и глобальным состоянием;
      \item \emph{Smart}: обёртки, получающие данные через хуки из \textit{features} и передающие их в презентационные компоненты.
    \end{itemize}
  \item \textit{features}: бизнес-логика и состояние в виде кастомных хуков, редьюсеров, слайсов Redux или Zustand:
    \begin{itemize}
      \item \textit{hooks.ts}: главный хук-фабрика (например, \textit{useRegistrationUniversity});
      \item \textit{schema.ts}: схемы валидации через \textit{Yup} или \textit{Zod};
      \item \textit{utils.ts}: вспомогательные функции и селекторы;
      \item \textit{store.ts}: подключение к Redux/Redux Toolkit или настройка Zustand.
    \end{itemize}
\end{enumerate}

\subsubsection{Пример: RegistrationUniversity}

Ниже приведён пример реализации виджета и соответствующего хука (см. листинги~\ref{lst:registration-university-widget} и~\ref{lst:use-registration-university}):

\begin{lstlisting}[breaklines=true,caption=RegistrationUniversityWidget,label=lst:registration-university-widget]
  // Presentation-компонент
  export function RegistrationUniversityWidget() {
    const { formDataRef, isError, setIsError, onSubmit, isLoading } =
      useRegistrationUniversity();

    return (
      <div>...</div>
    );
  }
\end{lstlisting}

\begin{lstlisting}[breaklines=true,caption=useRegistrationUniversity,label=lst:use-registration-university]
  // Smart-компонент: хук-фабрика
  export function useRegistrationUniversity() {
    const formDataRef = useRef<RegistrationData>();
    const [isError, setIsError] = useState<string[]>([]);
    const [isLoading, setIsLoading] = useState(false);
    const router = useRouter();

    const onSubmit = async () => {
      ...
    };
    return { formDataRef, isError, setIsError, onSubmit, isLoading };
  }
\end{lstlisting}

\subsubsection{Соглашения по именованию и структуре}

Для поддержания единого стиля и предсказуемости структуры проекта:
\begin{itemize}
  \item Имена функций API интерфейса и типов дублируют backend-операции;
  \item Компоненты и хуки получают префиксы по зоне ответственности (\textit{RegistrationForm}, \textit{useFilesUpload} и т. д.);
  \item В каждом каталоге \textit{features/FeatureName} обязателен минимум файлов: \textit{index.ts}, \textit{hooks.ts}, \textit{schema.ts}, \textit{utils.ts};
  \item Все слои описаны в README с диаграммой и примерами использования;
  \item Код-ревью включает проверку соответствия FSD-подходу и правилам TypeScript.
\end{itemize}

% Раздел 3.5: Административная панель университета
\subsection{Административная панель университета}
Административная панель университета разделена на несколько секций, каждая из которых отвечает за управление соответствующими данными: институтами, кафедрами, группами, преподавателями и студентами. Также предусмотрён модуль для формирования и отправки приглашений.

\subsubsection{Страница списка с таблицей данных}
На страницах списка отображается компонент \texttt{Table} с переданной конфигурацией \texttt{header} и \texttt{body}, а также обрабатываются события клика по строкам для навигации к деталям записи:
\begin{lstlisting}[caption={Компонент страницы списка институтов}]
export function AdminInstitutePage() {
    const { data, body, onClickRow, header } = useListAdminInstitute();

    return (
        <div className={AdminListPageStyle.AdminListPage}>
            <div className={AdminListPageStyle.table}>
                <Table
                    header={header}
                    body={body}
                    onClickRow={onClickRow}
                />
            </div>
            <PaginationComponent totalCount={data.total_count} />
        </div>
    );
}
\end{lstlisting}

\subsubsection{Страница просмотра/редактирования записи}
На странице детализации отображаются текущие данные и предоставляется возможность их изменения через генератор форм и кнопку сохранения. Также реализована операция удаления записи:
\begin{lstlisting}[caption={Компонент страницы детализации института}]
export function AdminInstituteDetailPage({ id }: { id: number }) {
    const { onUpdate, onDelete, onChangeFormData, data } = useDetailAdminInstitute(id);

    return (
        <div className={AdminDetailPageStyle.AdminDetailPage}>
            <div className={AdminDetailPageStyle.content}>
                <h3 className={AdminDetailPageStyle.header}>
                   	Данные института
                </h3>
                <FormBuilder<InstitutePatchType>
                    formDataDefault={data}
                    clearForm={true}
                    onChange={onChangeFormData}
                    schema={instituteSchema()}
                />
                <Button
                    size="large"
                    hierarchy="primary"
                    text="Сохранить"
                    width="fill"
                    onClick={onUpdate}
                />
            </div>
            <div className={AdminDetailPageStyle.action}>
                <h3 className={AdminDetailPageStyle.header}>
                   	Операции
                </h3>
                <Button
                    size="small"
                    hierarchy="primary"
                    warning={true}
                    iconLeft={<Trash01SVG />}
                    text="Удалить"
                    onClick={onDelete}
                    width="fill"
                />
            </div>
        </div>
    );
}
\end{lstlisting}

\subsubsection{Страница создания новой записи}
Компонент создания использует \texttt{FormBuilder} с соответствующей схемой и отправляет форму по событию нажатия кнопки:
\begin{lstlisting}[caption={Компонент страницы создания института}]
export function AdminInstituteCreatePage() {
    const { onChangeFormData, onSend } = useCreateAdminInstitute();

    return (
        <div className={AdminDetailPageStyle.AdminDetailPage}>
            <div className={AdminDetailPageStyle.content}>
                <h1 className={AdminDetailPageStyle.header}>
                   	Создание института
                </h1>
                <FormBuilder<InstitutePostType>
                    schema={instituteSchema()}
                    onChange={onChangeFormData}
                />
                <Button
                    onClick={onSend}
                    text="Создать"
                    size="large"
                    width="fill"
                />
            </div>
        </div>
    );
}
\end{lstlisting}

\subsubsection{Модальные виджеты для создания приглашений}
Для формирования приглашений используются два модальных компонента, управляемые состоянием \texttt{isActiveModalWindow}:
\begin{lstlisting}[caption={Выбор модального окна приглашения}]
const getModalWindow = () => {
    switch (isActiveModalWindow) {
        case 'teacher':
            return <CreateTeacherInvites onClose={() => setIsActiveModalWindow(undefined)} />;
        case 'students':
            return <CreateStudentsInvites onClose={() => setIsActiveModalWindow(undefined)} />;
    }
};
\end{lstlisting}

Кнопки для вызова модальных окон:
\begin{lstlisting}[caption={Кнопки вызова модальных приглашений}]
<ActionField
    title="Пригласить преподавателя"
    subtitle="Сформировать ссылку для регистрации преподавателя"
    onClick={() => setIsActiveModalWindow('teacher')}
/>
<ActionField
    title="Пригласить студентов"
    subtitle="Сформировать ссылку для регистрации студентов"
    onClick={() => setIsActiveModalWindow('students')}
/>
\end{lstlisting}

% Конец раздела 3.5
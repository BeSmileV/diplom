\subsection{Тестирование}

В ходе разработки были протестированы модули как основного приложения, так и используемой при разработке библиотеки.

\subsubsection{Покрытие текста программы в приложении}

\begin{table}[h]
  \centering
    \small
  \caption{Покрытие текста программы клиентской части приложения.}
  \label{tab:app-coverage}
  \begin{tabular}{lrrrr}
  	\toprule
  	\textbf{File}                & \textbf{\%Stmts} & \textbf{\%Branch} & \textbf{\%Funcs} & \textbf{\%Lines} \\ \midrule
  	All files                    &            95.65 &             77.77 &           100.00 &           100.00 \\
  	features/Chats/lib           &           100.00 &            100.00 &           100.00 &           100.00 \\
  	\quad mergeMessages.ts       &           100.00 &            100.00 &           100.00 &           100.00 \\
  	shared/ui/TextEditor/lib     &            90.47 &             68.42 &           100.00 &           100.00 \\
  	\quad processTextToTiptap.ts &            94.11 &             72.22 &           100.00 &           100.00 \\
  	\quad processTiptapToText.ts &            75.00 &              0.00 &           100.00 &           100.00 \\ \bottomrule
  \end{tabular}
\end{table}

\noindent
В таблице~\ref{tab:app-coverage} приведены четыре основные метрики покрытия:
\begin{enumerate}
  \item \%Stmts (Statements) — процент операторов текста программы, выполненных в ходе тестов;
  \item \%Branch (Branches) — процент ветвей условных операторов, затронутых тестами;
  \item \%Funcs (Functions) — процент функций, вызванных хотя бы одним тестом;
  \item \%Lines (Lines) — процент строк текста программы, исполненных тестами.
\end{enumerate}
Для ключевых модулей (например, \textit{mergeMessages.ts}) все метрики равны 100\,\%.

\subsubsection{Покрытие текста программы в отдельной библиотеке}

\begin{table}[h]
  \small
  \centering
  \caption{Покрытие текста программы библиотеки компонентов.}
  \label{tab:lib-coverage}
  \begin{tabular}{lrrrr}
  	\toprule
  	\textbf{File}                           & \textbf{\%Stmts} & \textbf{\%Branch} & \textbf{\%Funcs} & \textbf{\%Lines} \\ \midrule
  	All files                               &            52.89 &             48.76 &            20.68 &            52.65 \\
  	lib/dict/getDeepValue.ts                &            86.36 &             73.33 &           100.00 &            85.71 \\
  	lib/dict/setDeepValue.ts                &           100.00 &            100.00 &           100.00 &           100.00 \\
  	ui/DateTimePicker/lib/changeInterval.ts &           100.00 &             96.29 &           100.00 &           100.00 \\ \bottomrule
  \end{tabular}
\end{table}

\noindent
В таблице~\ref{tab:lib-coverage} показано, что основные утилитные функции библиотеки (\textit{getDeepValue}, \textit{setDeepValue}, \textit{changeInterval}) полностью или почти полностью покрыты.

\subsubsection{Методы тестирования}

В проекте использовались различные подходы к тестированию, направленные на проверку корректности функциональности, устойчивости к ошибкам и полноты покрытия. Ниже перечислены основные применённые методики:

\begin{enumerate}
  \item Модульное тестирование (unit testing) применяется для каждой функции и компонента, при этом тесты охватывают граничные и некорректные входные данные (например, \textit{undefined}, пустые массивы), типичные сценарии и пограничные случаи;
  \item TDD-подход (Test–Driven Development) основывается на цикле разработки, где сначала пишется тест, затем фиксируется его неуспешное выполнение, после чего создаётся минимальная реализация и проводится повторное тестирование, завершающееся рефакторингом;
  \item Покрытие ветвлений (branch coverage) достигается за счёт тестов, охватывающих все возможные пути выполнения условных операторов, включая конструкции \textit{if/else}, тернарные выражения и операторы \textit{switch};
  \item Round-trip-тесты проверяют корректность преобразований между текстом и HTML в функциях \textit{processTextToTiptap} и \textit{processTiptapToText}, особенно в сложных случаях с вложенными тегами и переносами строк;
  \item Инструментация и сбор покрытия реализуются через запуск команды \textit{npx jest --coverage}, которая автоматически добавляет счётчики в текст программы и формирует метрики по выражениям, ветвлениям, функциям и строкам.
\end{enumerate}


\subsubsection{Последовательная диаграмма работы модуля Classrooms}
Ниже приведена последовательная диаграмма, иллюстрирующая жизненный цикл взаимодействия преподавателя, студента и сервиса искусственного интеллекта при работе с виртуальными классами и заданиями. На рисунке~\ref{fig:classroom-flow} показаны этапы: создание класса, создание задания с указанием промта (текстового запроса) для автопроверки, получение списка заданий студентом, отправка решения, автоматическая проверка решения и выставление оценки преподавателем.

Выделены следующие этапы:

\begin{enumerate}
    \item При создании класса преподаватель открывает интерфейс создания класса, интерфейс отправляет данные класса на сервер и получает подтверждение.
    \item При создании задания преподаватель указывает условие и текстовый запрос для проверки искусственным интеллектом, интерфейс отправляет данные на сервер, сервер подтверждает сохранение.
    \item При получении списка заданий студент открывает интерфейс студента, интерфейс запрашивает список заданий по идентификатору класса, сервер возвращает список доступных заданий, интерфейс отображает их студенту.
    \item При отправке решения студент выбирает задание и отправляет решение через интерфейс, отправляемые данные (файлы решения и метаданные, например, идентификаторы студента и задания) принимаются сервером, сервер подтверждает получение.
    \item При автоматической проверке сервер получает решение и текстовый запрос, отправляет их во внешний сервис искусственного интеллекта, который анализирует текст программы (проверка корректности, стилистических нарушений и т.п.) и возвращает комментарии с оценкой. Сервер сохраняет результаты и передаёт их в интерфейсы преподавателя и студента, отображая им результат автопроверки (комментарии и предварительную оценку).
    \item При выставлении финальной оценки преподаватель открывает интерфейс выставления оценки, отправляет оценку через интерфейс, сервер сохраняет оценку и подтверждает её сохранение. Интерфейс отображает подтверждение преподавателю и обновлённую оценку студенту.
\end{enumerate}

Таким образом, последовательная диаграмма на рисунке~\ref{fig:classroom-flow} демонстрирует цикл взаимодействий: от создания класса и задания преподавателем до получения студентом финальной оценки после автоматической проверки и ручной оценки преподавателем.

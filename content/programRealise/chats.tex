% Раздел 3.7: Модуль «Chats» (Чаты)
\subsection{Модуль «Chats» (Чаты)}
Модуль «Chats» предназначен для организации обмена сообщениями в режиме реального времени между участниками системы (преподавателями и студентами). Ключевой технологией является библиотека Socket.IO, обеспечивающая надёжное WebSocket-соединение с возможностью автоматического восстановления. Вся бизнес-логика работы с соединением, событиями и состоянием сообщений инкапсулирована в отдельных функциях и хуках — это позволяет UI-компонентам оставаться «тонкими» и фокусироваться только на отображении.

В этом подразделе приведены основные элементы модуля «Chats»:
\begin{itemize}
  \item хук \texttt{useSocket} для управления жизненным циклом WebSocket-соединения;
  \item компаратор \texttt{messagesComp} и функция \texttt{mergeMessages} для аккуратного упорядочивания и объединения потоков сообщений;
  \item функция \texttt{addMessages} для ввода свежих и локально отправленных сообщений в общее состояние;
  \item функция \texttt{sendMessage}, реализующая стратегию optimistic UI и синхронизацию с сервером;
  \item компонент \texttt{ScrollProvider} для двусторонней (вперёд и назад) пагинации сообщений.
\end{itemize}

\subsubsection{Хук \texttt{useSocket}}
Хук \texttt{useSocket} реализует следующие задачи:
\begin{enumerate}
  \item Аутентификация через JWT: периодическое обновление токена доступа и передача его в заголовках при установке соединения;
  \item Установка соединения с сервером по Socket.IO, включая указание URL, пользовательских заголовков и пути для WebSocket-подключения;
  \item Подписка на базовые события \texttt{connect}, \texttt{disconnect}, \texttt{connect\_error} для управления флагом \texttt{isConnected};
  \item Динамическая регистрация пользовательских обработчиков событий через параметр \texttt{onEvents};
  \item Предоставление методов \texttt{emit}, \texttt{disconnect}, \texttt{reconnect}, \texttt{connect} для внешнего управления соединением.
\end{enumerate}

Процесс работы хука выглядит следующим образом:
\begin{itemize}
  \item При монтировании компонента запускается эффект для получения сессии пользователя посредством функции \texttt{getClientSession}. Результат (JWT-токен) сохраняется в локальном состоянии и автоматически обновляется через заданный таймаут \texttt{TOKEN\_UPDATE\_TIMEOUT}. Это позволяет избежать простоя соединения из-за истёкшего токена и гарантирует, что при переподключении будет использован свежий токен авторизации.
  \item Второй эффект отвечает за само подключение к Socket.IO. Он создаёт новый экземпляр \texttt{io(url, options)}, где в \texttt{options.extraHeaders} передаётся поле \texttt{Authorization}. Затем регистрируются «родные» события соединения и пользовательские события, переданные через проп \texttt{onEvents}. После этого ссылка на сокет сохраняется в рефе \texttt{socketRef}. Возврат функции очистки гарантирует, что при размонтировании или изменении зависимостей соединение будет корректно разорвано.
  \item Метод \texttt{emit} инкапсулирует вызов \texttt{socket.emit}, проверяя предварительно существование объекта сокета. Методы \texttt{disconnect}, \texttt{reconnect} и \texttt{connect} позволяют при необходимости вручную управлять соединением, например, при смене URL или после обработки ошибки.
\end{itemize}

\begin{lstlisting}[caption={Кастомный хук useSocket}, label={lst:useSocket}]
export function useSocket({ url, headers, onEvents }) {
    const socketRef = useRef(null);
    const [isConnected, setIsConnected] = useState(false);
    const [token, setToken] = useState(null);
    
    // 1. Обновление токена доступа с фиксированным интервалом
    useEffect(() => {
        const updateAuth = () => getClientSession().then(setToken);
        updateAuth();
        const timer = setInterval(updateAuth, TOKEN_UPDATE_TIMEOUT);
        return () => clearInterval(timer);
    }, []);
    
    // 2. Установка и подписка на WebSocket
    useEffect(() => {
        const socket = io(url, { extraHeaders: { Authorization: token?.access_token }, path: '/ws/socket.io' });
        socket.on('connect',    () => setIsConnected(true));
        socket.on('disconnect', () => setIsConnected(false));
        socket.on('connect_error', () => setIsConnected(false));
        // Регистрируем пользовательские события
        Object.entries(onEvents || {}).forEach(([event, handler]) => {
            socket.on(event, handler);
        });
        socketRef.current = socket;
        return () => socket.disconnect();
    }, [url, token, JSON.stringify(headers), JSON.stringify(onEvents)]);
    
    // 3. Эмиттер и ручное управление
    const emit      = (...args) => socketRef.current?.emit(...args);
    const disconnect = ()      => socketRef.current?.disconnect();
    const reconnect  = ()      => { socketRef.current?.disconnect(); socket.connect(); };
    const connect    = ()      => socketRef.current?.connect();
    
    return { emit, disconnect, reconnect, connect, isConnected };
}
\end{lstlisting}

В результате хук \texttt{useSocket} становится универсальным инструментом: он подходит для любых namespaces и событий, достаточно определить схему типов для \texttt{OnEvents} и \texttt{EmitEvents}.

\subsubsection{Корректное объединение потоков сообщений}
Передача и получение сообщений в чатах осуществляется в виде потоков, которые могут содержать старые и новые элементы. Для упорядочивания и исключения дубликатов разработаны:
\begin{itemize}
  \item \texttt{messagesComp}: компаратор, сравнивающий два сообщения по серверному \texttt{id} или по локальному \texttt{local\_id}, а затем по полю \texttt{created\_at};
  \item \texttt{mergeMessages}: редуктор, объединяющий два списка \texttt{base} и \texttt{insert} в хронологическом порядке, сохраняя уникальность.
\end{itemize}
Компаратор сначала проверяет полное совпадение идентификаторов — это позволяет найти случаи, когда локально отправленное сообщение (ещё без \texttt{id} от сервера) соответствует уже подтверждённому сервером сообщению по \texttt{local\_id}. Если идентификаторы различны, они сравниваются по временной метке \texttt{created\_at}. Такая стратегия обеспечивает корректную сортировку и «слияние» потоков.

\begin{lstlisting}[caption={Компаратор сообщений messagesComp}, label={lst:messagesComp}]
const messagesComp = (a, b) => {
    // 1) Идентификаторы совпадают → одинаковое сообщение
    if ((a.id && b.id && a.id === b.id) ||
        (a.local_id && b.local_id && a.local_id === b.local_id)) {
        return 0; // EQUAL
    }
    // 2) Сравнение по дате создания
    return new Date(a.created_at) < new Date(b.created_at) ? -1 : 1;
};
\end{lstlisting}

\begin{lstlisting}[caption={Функция слияния mergeMessages}, label={lst:mergeMessages}]
export function mergeMessages({ base, insert, comp }) {
    const result = [];
    let i = 0, j = 0;
    // Идём по двум спискам параллельно, добавляя меньший элемент
    while (i < base.length && j < insert.length) {
        if (comp(base[i], insert[j]) <= 0) result.push(base[i++]);
        else result.push(insert[j++]);
    }
    // Добавляем остатки
    return result.concat(base.slice(i)).concat(insert.slice(j));
}
\end{lstlisting}

\subsubsection{Добавление сообщений и оптимистичный UI}
Для плавного UX при отправке сообщений используется механизм optimistic UI:
\begin{enumerate}
  \item При отправке формируется временное сообщение \texttt{tempMsg} с полем \texttt{local\_id} и статусом \texttt{'sending'};
  \item Это сообщение сразу добавляется в общий поток через \texttt{addMessages};
  \item После подтверждения от сервера (приём события \texttt{'message\_sent'}) локальное сообщение заменяется на настоящий объект с серверным \texttt{id} благодаря функции слияния.
\end{enumerate}

\begin{lstlisting}[caption={Функция addMessages}, label={lst:addMessages}]
function addMessages(newMsgs) {
    setMessages(prev => mergeMessages({
        base: prev || [],
        insert: Array.isArray(newMsgs) ? newMsgs : [newMsgs],
        comp: messagesComp
    }));
}
\end{lstlisting}

\begin{lstlisting}[caption={Отправка сообщения sendMessage}, label={lst:sendMessage}]
function sendMessage(message) {
    const localId = generateLocalId();
    const tempMsg = { ...message, local_id: localId, status: 'sending', created_at: new Date().toISOString() };
    // 1) Мгновенно отображаем в UI
    addMessages(tempMsg);
    // 2) Отправляем на сервер, сервер вернёт событие о доставке
    emit('send_message', message);
}
\end{lstlisting}

Такой подход позволяет пользователю видеть своё сообщение немедленно, даже если сеть медленная, и обеспечивает плавную замену локального объекта на серверный при получении подтверждения.

\subsubsection{Двусторонняя пагинация через ScrollProvider}
Для загрузки как старых, так и новых сообщений используется компонент \texttt{ScrollProvider}. Он вызывает соответствующие колбэки при достижении верха или низа скролла, что позволяет:
\begin{itemize}
  \item при прокрутке вверх дозагружать более ранние сообщения;
  \item при прокрутке вниз обновлять список свежими сообщениями;
  \item сохранять позицию скролла при подгрузке исторических или новых данных.
\end{itemize}

\begin{lstlisting}[caption={Использование ScrollProvider}, label={lst:ScrollProvider}]
<ScrollProvider onScrollTop={loadOlder} onScrollBottom={loadNewer} accuracy={200}>
    {messages.map(m => <ChatMessageItem key={m.id || m.local_id} {...m} />)}
</ScrollProvider>
\end{lstlisting}

Комбинация хука \texttt{useSocket}, функций \texttt{mergeMessages}/\texttt{addMessages} и компонента \texttt{ScrollProvider} создаёт единый модуль чатов, отвечающий требованиям надёжности, масштабируемости и приятного UX. 

% Конец раздела 3.7

% Раздел 3.4: Модуль «Регистрация и вход"
\subsection{Модуль «Регистрация и вход»}
В системе предусмотрены три сценария регистрации: для университета, студентов и преподавателей. В URL-параметрах \texttt{invite\_id} передаются данные приглашения для последующей валидации и передачи на бэкенд.

% Выбор виджета регистрации по типу
\begin{lstlisting}[caption={Выбор виджета регистрации}]
const getForm = () => {
    const type = getSearchParams(REGISTRATION_TYPE_PARAM_NAME) as RegistrationTypesType;
    const inviteId = getInviteId();
    switch (type) {
        case 'teacher':
            return <RegistrationTeacherWidget inviteId={inviteId} />;
        case 'student':
            return <RegistrationStudentWidget inviteId={inviteId} />;
        case 'university':
        default:
            return <RegistrationUniversityWidget />;
    }
};
\end{lstlisting}

Далее \texttt{invite\_id} передаётся в соответствующий виджет, который запрашивает на сервере данные приглашения. В случае неверного или просроченного ID отображается сообщение об ошибке.

% Виджет регистрации преподавателя
\begin{lstlisting}[caption={RegistrationTeacherWidget}]
export function RegistrationTeacherWidget({ inviteId }: RegistrationPropsType) {
    const { initData, onSubmit, isError, setIsError, formDataRef } =
        useRegistrationStudentAndTeacher<typeof registerTeacher>({
            inviteId,
            registrationRequest: registerTeacher
        });

    if (initData === undefined) {
        return 'Loading';
    }

    if (initData === null) {
        // Неверное приглашение или оно истекло
        return 'Error';
    }

	...
}
\end{lstlisting}

% Конец раздела 3.4

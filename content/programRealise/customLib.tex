% Раздел 3.2: Собственная UI-библиотека и генерация форм
\subsection{Собственная UI-библиотека и генерация форм}

\subsubsection{Общая идея и мотивация}
Для обеспечения единого стилистического и функционального каркаса клиентского приложения была разработана собственная библиотека компонентов и утилит, распространяемая через npm-пакет. В её составе присутствуют:
\begin{itemize}
  \item SCSS-миксины для реализации адаптивных сеток и гибкой типографики,
  \item набор готовых UI-компонентов (например, \texttt{Button}, \texttt{Tag}, \texttt{ScrollProvider}), не содержащих бизнес-логику,
  \item провайдеры глобальных состояний и контекстов (например, для управления скроллом или обработкой событий клавиатуры),
  \item унифицированный генератор форм \texttt{FormBuilder}, позволяющий описывать структуру и поведение сложных форм через декларативную схему.
\end{itemize}
Применение данной библиотеки ускоряет процессы разработки и упрощает поддержку интерфейса, так как все ключевые решения собраны в централизованном модуле с единым API и консистентной документацией.

\subsubsection{Компонент \texttt{FormBuilder}: концепция и API}
Ключевым элементом библиотеки является компонент \texttt{FormBuilder}. Он реализует маршрутизацию данных и событий между декларативной схемой формы и её полями. Основные принципы работы:
\begin{enumerate}
  \item Пользователь задаёт схему параметров формы, представляющую собой массив объектов с полем \texttt{type} и соответствующим набором \texttt{props}.
  \item \texttt{FormBuilder} инициализирует внутреннее состояние формы и передаёт каждому полю текущие значения и функции обработки изменений.
  \item При срабатывании события изменения значения поле уведомляет \texttt{FormBuilder}, который обновляет общую модель данных и вызывает коллбэк \texttt{onChange}.
\end{enumerate}

\paragraph{Определение схемы формы}

Схема формы описывается функцией, возвращающей массив объектов. Каждый объект содержит:

\begin{itemize}
  \item \texttt{type}: идентификатор типа элемента (например, \texttt{input\_field}, \texttt{array\_fields}),
  \item \texttt{props}: набор свойств, необходимых для рендеринга и обработки (имя поля, текст метки, дополнительные параметры).
\end{itemize}

Пример описания схемы формы приведён в листинге~\ref{lst:form-scheme}.

\begin{lstlisting}[caption=Пример описания схемы формы,label=lst:form-scheme]
export function inviteTeacherScheme(): FORM_BUILDER_SCHEMA {
  return [
    {
      type: 'input_field',
      props: {
        name: 'email',
        labelText: 'Email'
      }
    },
    {
      type: 'input_field',
      props: {
        type: 'select',
        name: 'department_id',
        ownerInputComponent: <DepartmentSelectField />
      }
    }
  ];
}
\end{lstlisting}

Пример использования компонента \texttt{FormBuilder} приведён в листинге~\ref{lst:formbuilder-usage}.

\begin{lstlisting}[caption=Использование \texttt{FormBuilder},label=lst:formbuilder-usage]
<FormBuilder schema={inviteTeacherScheme()} 
			 onChange={onChangeFormData}/>
\end{lstlisting}

Компонент \texttt{FormBuilder} автоматически распределяет данные между полями и собирает итоговый объект формы, передавая его через \texttt{onChange}.

\subsubsection{Типы элементов схемы и их поведение}
Ниже приведены ключевые типы схем, поддерживаемые \texttt{FormBuilder}, и описание их функциональности.

\paragraph{INPUT\_FIELD\_SCHEMA}
Тип \texttt{input\_field} отвечает за отображение и управление единичным полем ввода.
\begin{itemize}
  \item \texttt{name}: ключ в итоговом объекте данных,
  \item \texttt{labelText}: отображаемая метка поля,
  \item \texttt{hintText} (опционально): текст подсказки под полем,
  \item \texttt{type} (опционально): уточняет тип поля (например, \texttt{select}, \texttt{datetime}),
  \item \texttt{ownerInputComponent} (опционально): пользовательский компонент, принимающий \texttt{value}, \texttt{onChange}, \texttt{isError}, \texttt{onBlur}.
\end{itemize}

\begin{lstlisting}[caption={Пример INPUT\_FIELD\_SCHEMA}]
const schema: INPUT_FIELD_SCHEMA = {
  type: 'input_field',
  props: {
    name: 'username',
    labelText: 'Имя пользователя',
    hintText: 'Введите ваш логин'
  }
};
\end{lstlisting}

\paragraph{ARRAY\_FIELDS\_SCHEMA}
Тип \texttt{array\_fields} предназначен для формирования массива однотипных входных полей. Все вложенные \texttt{input\_field} будут собраны в массив.

\begin{itemize}
  \item \texttt{name}: имя массива в итоговой модели,
  \item \texttt{children}: массив схем элементов, входящих в каждую запись.
\end{itemize}

Пример схемы показан в листинге~\ref{lst:array-fields-schema}.

\begin{lstlisting}[caption={Пример \texttt{ARRAY\_FIELDS\_SCHEMA}},label={lst:array-fields-schema}]
const schema: ARRAY_FIELDS_SCHEMA = [
  {
    type: 'array_fields',
    props: {
      name: 'subjects',
      children: [
        {
          type: 'input_field',
          props: {
            name: 'subjectName',
            labelText: 'Название предмета'
          }
        }
      ]
    }
  }
];
\end{lstlisting}

\paragraph{FORM\_WRAPPER\_SCHEMA}
Тип \texttt{form\_wrapper} служит для группировки полей без сброса счётчика массивов. Внутренние поля записываются как вложенный объект.

\begin{itemize}
  \item \texttt{name}: имя ключа в итоговом объекте,
  \item \texttt{children}: схема вложенных элементов.
\end{itemize}

Пример схемы показан в листинге~\ref{lst:form-wrapper-schema}.

\begin{lstlisting}[caption={Пример \texttt{FORM\_WRAPPER\_SCHEMA}},label={lst:form-wrapper-schema}]
const schema: FORM_WRAPPER_SCHEMA = [{
  type: 'form_wrapper',
  props: {
    name: 'teacherInfo',
    children: [
      {
        type: 'input_field',
        props: { name: 'firstName', labelText: 'Имя' }
      },
      {
        type: 'input_field',
        props: { name: 'lastName', labelText: 'Фамилия' }
      }
    ]
  }
}];
\end{lstlisting}

\paragraph{BLOCK\_WRAPPER\_SCHEMA}
Тип \texttt{block\_wrapper} позволяет визуально группировать элементы без изменения структуры данных: поля в нём обрабатываются как часть текущего массива или объекта. Пример схемы показан в листинге~\ref{lst:block-wrapper-schema}.

\begin{lstlisting}[caption={Пример \texttt{BLOCK\_WRAPPER\_SCHEMA}},label={lst:block-wrapper-schema}]
const schema: BLOCK_WRAPPER_SCHEMA = [
  {
    type: 'block_wrapper',
    props: {
      children: [
        {
          type: 'input_field',
          props: {
            name: 'code',
            labelText: 'Код'
          }
        }
      ]
    }
  }
];
\end{lstlisting}

\paragraph{REACT\_NODE\_SCHEMA}
Тип \texttt{react\_node} предназначен для вставки произвольного React-элемента в форму. Пример схемы показан в листинге~\ref{lst:react-node-schema}.

\begin{lstlisting}[caption={Пример \texttt{REACT\_NODE\_SCHEMA}},label={lst:react-node-schema}]
const schema: REACT_NODE_SCHEMA = [
  {
    type: 'react_node',
    props: {
      node: <CustomSeparator />
    }
  }
];
\end{lstlisting}

\subsubsection{Преимущества и выводы}
Использование декларативного \texttt{FormBuilder} существенно снижает сложность создания многоуровневых форм:
\begin{itemize}
  \item консистентность API при описании разных типов полей,
  \item возможность единообразной валидации и управления ошибками,
  \item лёгкость расширения — добавление новых типов полей или обёрток сводится к регистрации нового \texttt{type} и соответствующего рендерера,
  \item повышение читаемости кода: структура формы полностью отражена в схеме без дублирования логики в компонентах.
\end{itemize}

% Конец раздела 3.2

% Раздел 3.2: Собственная UI-библиотека и генерация форм
\subsection{Собственная UI-библиотека и генерация форм}

\subsubsection{Общая идея и мотивация}
Для обеспечения единого стилистического и функционального каркаса клиентского приложения была разработана собственная библиотека компонентов и утилит, распространяемая через npm-пакет. В её составе присутствуют:
\begin{enumerate}
  \item Набор готовых UI-компонентов (например, \textit{Button}, \textit{Tag}, \textit{ScrollProvider}), не содержащих бизнес-логику;
  \item Провайдеры глобальных состояний и контекстов (например, для управления скроллом или обработкой событий клавиатуры);
  \item Унифицированный генератор форм \textit{FormBuilder}, позволяющий описывать структуру и поведение сложных форм через декларативную схему.
\end{enumerate}
Применение данной библиотеки ускоряет процессы разработки и упрощает поддержку интерфейса, так как все ключевые решения собраны в централизованном модуле с единым API интерфейс и консистентной документацией.

\subsubsection{Компонент \textit{FormBuilder}}
Ключевым элементом библиотеки является компонент \textit{FormBuilder}. Он реализует маршрутизацию данных и событий между декларативной схемой формы и её полями. Основные принципы работы:
\begin{enumerate}
  \item Пользователь задаёт схему параметров формы, представляющую собой массив объектов с полем \textit{type} и соответствующим набором параметров;
  \item Компонент \textit{FormBuilder} инициализирует внутреннее состояние формы и передаёт каждому полю текущие значения и функции обработки изменений;
  \item При срабатывании события изменения значения поле уведомляет компонент \textit{FormBuilder}, который обновляет общую модель данных и вызывает функцию обратного вызова \textit{onChange}.
\end{enumerate}

Схема формы описывается функцией, возвращающей массив объектов. Каждый объект содержит поле \textit{type} — идентификатор типа элемента (например, \textit{input\_field} или \textit{array\_fields}), а также \textit{props} — набор свойств, необходимых для рендеринга и обработки (таких как имя поля, текст метки и дополнительные параметры). Пример описания схемы формы приведён в листинге~\ref{lst:form-scheme}.

\begin{lstlisting}[caption=Пример описания схемы формы,label=lst:form-scheme]
export function inviteTeacherScheme(): FORM_BUILDER_SCHEMA {
  return [
    {
      type: 'input_field',
      props: {
        name: 'email',
        labelText: 'Email'
      }
    },
    {
      type: 'input_field',
      props: {
        type: 'select',
        name: 'department_id',
        ownerInputComponent: <DepartmentSelectField />
      }
    }
  ];
}
\end{lstlisting}

Пример использования компонента \textit{FormBuilder} приведён в листинге~\ref{lst:formbuilder-usage}.

\begin{lstlisting}[caption=Использование компонента \textit{FormBuilder},label=lst:formbuilder-usage]
<FormBuilder schema={inviteTeacherScheme()} 
			 onChange={onChangeFormData}/>
\end{lstlisting}

Компонент \textit{FormBuilder} автоматически распределяет данные между полями и собирает итоговый объект формы, передавая его через функцию \textit{onChange}.

\subsubsection{Типы элементов схемы и их поведение}
Ниже приведены ключевые типы схем, поддерживаемые компонентом \textit{FormBuilder}, и описание их функциональности.

Схема \textit{INPUT\_FIELD\_SCHEMA} отвечает за отображение и управление единичным полем ввода. Она включает обязательное поле \textit{name} (ключ в итоговом объекте данных), а также опциональные параметры: \textit{labelText} (отображаемая метка поля), \textit{hintText} (текст подсказки под полем), \textit{type} (уточнение типа поля, например \textit{select} или \textit{datetime}) и \textit{ownerInputComponent} (пользовательский компонент, принимающий параметры: \textit{value}, \textit{onChange}, \textit{isError} и \textit{onBlur}). Пример использования схемы представлен в листинге~\ref{lst:input-fields-schema}.

\begin{lstlisting}[caption={Пример схемы INPUT\_FIELD\_SCHEMA},label={lst:input-fields-schema}]
const schema: INPUT_FIELD_SCHEMA = {
  type: 'input_field',
  props: {
    name: 'username',
    labelText: 'Имя пользователя',
    hintText: 'Введите ваш логин'
  }
};
\end{lstlisting}

Схема \textit{ARRAY\_FIELDS\_SCHEMA} предназначена для формирования массива однотипных входных полей, где все вложенные элементы \textit{input\_field} объединяются в массив. Обязательными параметрами схемы являются: \textit{name} (имя массива в итоговой модели данных) и \textit{children} (массив схем элементов, составляющих каждую отдельную запись). Пример использования данной схемы представлен в листинге~\ref{lst:array-fields-schema}.

\begin{lstlisting}[caption={Пример схемы \textit{ARRAY\_FIELDS\_SCHEMA}},label={lst:array-fields-schema}]
const schema: ARRAY_FIELDS_SCHEMA = [
  {
    type: 'array_fields',
    props: {
      name: 'subjects',
      children: [
        {
          type: 'input_field',
          props: {
            name: 'subjectName',
            labelText: 'Название предмета'
          }
        }
      ]
    }
  }
];
\end{lstlisting}

Схема \textit{FORM\_WRAPPER\_SCHEMA} обеспечивает группировку полей без сброса индексов массивов, сохраняя вложенные элементы как объект. Обязательными параметрами выступают \textit{name} (ключ в результирующем объекте) и \textit{children} (схема вложенных элементов). Пример реализации приведён в листинге~\ref{lst:form-wrapper-schema}.

\begin{lstlisting}[caption={Пример схемы \textit{FORM\_WRAPPER\_SCHEMA}},label={lst:form-wrapper-schema}]
const schema: FORM_WRAPPER_SCHEMA = [{
  type: 'form_wrapper',
  props: {
    name: 'teacherInfo',
    children: [
      {
        type: 'input_field',
        props: { name: 'firstName', labelText: 'Имя' }
      },
      {
        type: 'input_field',
        props: { name: 'lastName', labelText: 'Фамилия' }
      }
    ]
  }
}];
\end{lstlisting}

Схема \textit{BLOCK\_WRAPPER\_SCHEMA} позволяет визуально группировать элементы без изменения структуры данных: поля в нём обрабатываются как часть текущего массива или объекта. Пример схемы показан в листинге~\ref{lst:block-wrapper-schema}.

\begin{lstlisting}[caption={Пример схемы \textit{BLOCK\_WRAPPER\_SCHEMA}},label={lst:block-wrapper-schema}]
const schema: BLOCK_WRAPPER_SCHEMA = [
  {
    type: 'block_wrapper',
    props: {
      children: [
        {
          type: 'input_field',
          props: {
            name: 'code',
            labelText: 'Код'
          }
        }
      ]
    }
  }
];
\end{lstlisting}

Схема \textit{REACT\_NODE\_SCHEMA} предназначена для вставки произвольного React-элемента в форму. Пример схемы показан в листинге~\ref{lst:react-node-schema}.

\begin{lstlisting}[caption={Пример схемы \textit{REACT\_NODE\_SCHEMA}},label={lst:react-node-schema}]
const schema: REACT_NODE_SCHEMA = [
  {
    type: 'react_node',
    props: {
      node: <CustomSeparator />
    }
  }
];
\end{lstlisting}

\subsubsection{Преимущества и выводы}
Использование компонента \textit{FormBuilder} существенно снижает сложность создания многоуровневых форм:
\begin{enumerate}
  \item Единообразие описания различных полей;
  \item Возможность единообразной валидации и управления ошибками;
  \item Добавление новых полей сводится к регистрации нового блока схемы;
  \item Повышение читаемости текста программы: структура формы полностью отражена в схеме без дублирования логики в компонентах.
\end{enumerate}

% Конец раздела 3.2

% Раздел 3.2: Собственная UI-библиотека и генерация форм
\subsection{Собственная UI-библиотека и генерация форм}

\subsubsection{Общая идея и мотивация}
Для обеспечения единого стилистического и функционального каркаса клиентского приложения была разработана собственная библиотека компонентов и утилит, распространяемая через npm-пакет. В её составе присутствуют:
\begin{itemize}
	\item SCSS-миксины для реализации адаптивных сеток и гибкой типографики;
	\item Набор готовых UI-компонентов (например, \texttt{Button}, \texttt{Tag}, \texttt{ScrollProvider}), не содержащих бизнес-логику;
	\item Провайдеры глобальных состояний и контекстов (например, для управления скроллом или обработкой событий клавиатуры);
	\item Унифицированный генератор форм \texttt{FormBuilder}, позволяющий описывать структуру и поведение сложных форм через декларативную схему.
\end{itemize}
Применение данной библиотеки ускоряет процессы разработки и упрощает поддержку интерфейса, так как все ключевые решения собраны в централизованном модуле с единым API и консистентной документацией.

\subsubsection{Компонент \texttt{FormBuilder}: концепция и API}
Ключевым элементом библиотеки является компонент \texttt{FormBuilder}. Он реализует маршрутизацию данных и событий между декларативной схемой формы и её полями. Основные принципы работы:
\begin{enumerate}
	\item Пользователь задаёт схему параметров формы, представляющую собой массив объектов с полем \texttt{type} и соответствующим набором \texttt{props}.
	\item \texttt{FormBuilder} инициализирует внутреннее состояние формы и передаёт каждому полю текущие значения и функции обработки изменений.
	\item При срабатывании события изменения значения поле уведомляет \texttt{FormBuilder}, который обновляет общую модель данных и вызывает коллбэк \texttt{onChange}.
\end{enumerate}

\paragraph{Определение схемы формы}
Схема формы описывается функцией, возвращающей массив объектов. Каждый объект содержит:
\begin{itemize}
	\item \texttt{type}: идентификатор типа элемента (например, \texttt{input\_field}, \texttt{array\_fields});
	\item \texttt{props}: набор свойств, необходимых для рендеринга и обработки (имя поля, текст метки, дополнительные параметры).
\end{itemize}

% Пример схемы
\begin{lstlisting}[caption=Пример описания схемы формы]
export function inviteTeacherScheme(): FORM_BUILDER_SCHEMA {
	return [
	{
		type: 'input_field',
		props: {
			name: 'email',
			labelText: 'Email'
		}
	},
	{
		type: 'input_field',
		props: {
			type: 'select',
			name: 'department_id',
			ownerInputComponent: <DepartmentSelectField />
		}
	}
	];
}
\end{lstlisting}

\begin{lstlisting}[caption=Использование \texttt{FormBuilder}]
<FormBuilder schema={inviteTeacherScheme()} onChange={onChangeFormData}/>
\end{lstlisting}

Компонент \texttt{FormBuilder} автоматически распределяет данные между полями и собирает итоговый объект формы, передавая его через \texttt{onChange}.

\subsubsection{Типы элементов схемы и их поведение}
Ниже приведены ключевые типы схем, поддерживаемые \texttt{FormBuilder}, и описание их функциональности.

\paragraph{INPUT\_FIELD\_SCHEMA}
Тип \texttt{input\_field} отвечает за отображение и управление единичным полем ввода.
\begin{itemize}
	\item \texttt{name}: ключ в итоговом объекте данных.
	\item \texttt{labelText}: отображаемая метка поля.
	\item \texttt{hintText} (опционально): текст подсказки под полем.
	\item \texttt{type} (опционально): уточняет тип поля (например, \texttt{select}, \texttt{datetime}).
	\item \texttt{ownerInputComponent} (опционально): пользовательский компонент, принимающий \texttt{value}, \texttt{onChange}, \texttt{isError}, \texttt{onBlur}.
\end{itemize}

% Пример INPUT_FIELD_SCHEMA
\begin{lstlisting}[caption={Пример INPUT\_FIELD\_SCHEMA}]
const schema: INPUT_FIELD_SCHEMA = {
	type: 'input_field',
	props: {
		name: 'username',
		labelText: 'Имя пользователя',
		hintText: 'Введите ваш логин'
	}
};
\end{lstlisting}

\paragraph{ARRAY\_FIELDS\_SCHEMA}
Тип \texttt{array\_fields} предназначен для формирования массива однотипных входных полей. Все вложенные \texttt{input\_field} будут собраны в массив.
\begin{itemize}
	\item \texttt{name}: имя массива в итоговой модели.
	\item \texttt{children}: массив схем элементов, входящих в каждую запись.
\end{itemize}

% Пример ARRAY_FIELDS_SCHEMA
\begin{lstlisting}[caption={Пример ARRAY\_FIELDS\_SCHEMA}]
const schema: ARRAY_FIELDS_SCHEMA = [
	{
		type: 'array_fields',
		props: {
			name: 'subjects',
			children: [
				{
					type: 'input_field',
					props: {
						name: 'subjectName',
						labelText: 'Название предмета'
					}
				}
			]
		}
	}
];
\end{lstlisting}

\paragraph{FORM\_WRAPPER\_SCHEMA}
Тип \texttt{form\_wrapper} служит для группировки полей без сброса счётчика массивов. Внутренние поля записываются как вложенный объект.
\begin{itemize}
	\item \texttt{name}: имя ключа в итоговом объекте.
	\item \texttt{children}: схема вложенных элементов.
\end{itemize}

% Пример FORM_WRAPPER_SCHEMA
\begin{lstlisting}[caption={Пример FORM\_WRAPPER\_SCHEMA}]
const schema: FORM_WRAPPER_SCHEMA = [{
	type: 'form_wrapper',
	props: {
		name: 'teacherInfo',
		children: [
			{
				type: 'input_field',
				props: { name: 'firstName', labelText: 'Имя' }
			},
			{
				type: 'input_field',
				props: { name: 'lastName', labelText: 'Фамилия' }
			}
		]
	}
}];
\end{lstlisting}

\paragraph{BLOCK\_WRAPPER\_SCHEMA}
Тип \texttt{block\_wrapper} позволяет визуально группировать элементы без изменения структуры данных: поля в нём обрабатываются как часть текущего массива или объекта.

% Пример BLOCK_WRAPPER_SCHEMA
\begin{lstlisting}[caption={Пример BLOCK\_WRAPPER\_SCHEMA}]
const schema: BLOCK_WRAPPER_SCHEMA = [
	{
		type: 'block_wrapper',
		props: {
			children: [
				{
					type: 'input_field',
					props: {
						name: 'code',
						labelText: 'Код'
					}
				}
			]
		}
	}
];
\end{lstlisting}

\paragraph{REACT\_NODE\_SCHEMA}
Тип \texttt{react\_node} предназначен для вставки произвольного React-элемента в произвольном месте формы.
\begin{lstlisting}[caption={Пример REACT\_NODE\_SCHEMA}]
const schema: REACT_NODE_SCHEMA = [
	{
		type: 'react_node',
		props: {
			node: <CustomSeparator />
		}
	}
];
\end{lstlisting}

\subsubsection{Преимущества и выводы}
Использование декларативного \texttt{FormBuilder} существенно снижает сложность создания многоуровневых форм:
\begin{itemize}
	\item Консистентность API при описании разных типов полей;
	\item Возможность единообразной валидации и управления ошибками;
	\item Лёгкость расширения — добавление новых типов полей или обёрток сводится к регистрации нового \texttt{type} и соответствующего рендерера;
	\item Повышение читаемости кода: структура формы полностью отражена в схеме без дублирования логики в компонентах.
\end{itemize}

% Конец раздела 3.2

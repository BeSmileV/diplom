% Раздел 3.3: Интеграция Auth.js
\subsection{Интеграция Auth.js}
Для реализации авторизации в приложении использовалась библиотека Auth.js (ранее известная как NextAuth.js). Данный пакет рекомендован в официальной документации Next.js и обеспечивает гибкую работу с JWT-сессиями, однако требует дополнительной настройки для поддержки типизации, обновления токенов и обработки ошибок.

\subsubsection{Типизация данных сессии}
Для корректной работы TypeScript необходимо расширить интерфейсы, предоставляемые Auth.js. Пример декларации модулей:
\begin{lstlisting}[caption={Типизация JWT и Session в Auth.js}]
	import { AuthJWTErrorType } from "@/features/next-auth/types";
	
	export declare module "next-auth/jwt" {
		interface JWT {
			access_token?: string;
			refresh_token?: string;
			token_type?: string;
			error?: AuthJWTErrorType;
		}
	}
	
	export declare module "next-auth" {
		interface Session {
			user: {
				access_token?: string;
				refresh_token?: string;
				token_type?: string;
				error?: AuthJWTErrorType;
			};
		}
		interface User {
			access_token?: string;
			refresh_token?: string;
			token_type?: string;
		}
	}
\end{lstlisting}

\subsubsection{Конфигурация провайдера и функция \texttt{authorize}}
Основная логика входа пользователя реализуется в методе \texttt{authorize}, где учётные данные передаются на сервер, и в случае успешного ответа возвращается объект пользователя:
\begin{lstlisting}[caption={Реализация функции authorize}]
	async authorize(credentials, _req) {
		if (credentials.username && credentials.password) {
			const userData = {
				username: credentials.username as string,
				password: credentials.password as string,
			};
			
			const authResponse = await login(userData);
			if (authResponse?.success) {
				return authResponse.data;
			} else {
				throw new InvalidLoginError(
				`Backend server error: ${JSON.stringify(authResponse?.data)}`
				);
			}
		}
		throw new InvalidLoginError('Unknown authentication error');
	}
\end{lstlisting}

\subsubsection{JWT-callback и обновление токенов}
Для передачи и обновления JWT в сессии используется callback \texttt{jwt}, принимающий параметр \texttt{trigger}, позволяющий определить сценарий обработки:
\begin{lstlisting}[caption={JWT-callback с учётом trigger}]
	async jwt({ token, trigger, user, session }): Promise<JWT> {
		if (trigger === 'signIn') {
			return { ...user, error: null };
		}
		if (token == null) {
			return { ...session?.user, error: 'another' } as JWT;
		}
		if (trigger === 'update') {
			return await refreshingProcess(token);
		}
		return { ...token, error: null };
	},
\end{lstlisting}

Важным аспектом является недопущение состояния гонки при одновременном запросе обновления токена: для этого на клиенте сохраняется единственный промис обновления, который переиспользуется при множественных запросах:
\begin{lstlisting}[caption={Механизм предотвращения race condition при рефреше токена}]
	import { JWT } from "next-auth/jwt";
	
	export type RefreshPromiseStateType = Promise<JWT | null> | null;
	let tokenPromiseState: RefreshPromiseStateType = null;
	
	export const setTokenPromiseState = (
	promise: RefreshPromiseStateType
	): void => {
		tokenPromiseState = promise;
	};
	
	export const getTokenPromiseState = (): RefreshPromiseStateType => {
		return tokenPromiseState;
	};
	
	let tokenState: JWT | null = null;
	export const setTokenState = (newTokenState: JWT | null): void => {
		tokenState = newTokenState;
	};
	export const getTokenState = (): JWT | null => {
		return tokenState;
	};
	
	let timeoutState: NodeJS.Timeout | null = null;
\end{lstlisting}

\subsubsection{Проверка сессии в Middleware}
Для защиты маршрутов на уровне middleware необходимо предварительно проверять наличие валидной сессии и доступность токена:
\begin{lstlisting}[caption={Пример проверки сессии в Middleware}]
	const session = (await auth()) as Session | undefined;
	const tokenPayload = getAccessTokenPayload(
	session?.user?.access_token
	);
	if (
	session?.user?.access_token &&
	(isTokenAvailable(session.user.access_token) ||
	isTokenAvailable(session.user.refresh_token)) &&
	!session.user.error &&
	tokenPayload
	) {
		// доступ разрешён
	} else {
		// перенаправление на страницу входа
	}
\end{lstlisting}

Данный подход обеспечивает централизованную обработку авторизации и автоматическое обновление токенов без дублирования логики в разных частях приложения.

% Конец раздела 3.3

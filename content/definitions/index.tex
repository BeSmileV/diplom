\newpage
\ESKDthisStyle{formII}
\ESKDcolumnII{OПРЕДЕЛЕНИЯ, СOКРAЩЕНИЯ И OБOЗНAЧЕНИ}
\section*{OПРЕДЕЛЕНИЯ, СOКРAЩЕНИЯ И OБOЗНAЧЕНИЯ}
\addcontentsline{toc}{section}{OПРЕДЕЛЕНИЯ, СOКРAЩЕНИЯ И OБOЗНAЧЕНИЯ}
\setcounter{page}{4}

\begin{itemize}
  \item ИИ (AI) — искусственный интеллект.
  \item CSR (Client-Side Rendering) — отрисовка страницы на стороне клиента после загрузки минимального каркаса с сервера.
  \item SSR (Server-Side Rendering) — рендеринг веб-страниц на стороне сервера перед отправкой клиенту.
  \item SSG (Static Site Generation) — генерация статических HTML-страниц на этапе сборки приложения.
  \item JWT (JSON Web Token) — формат токена для безопасной передачи информации между сторонами в виде JSON-объекта.
  \item API (Application Programming Interface) — программный интерфейс, обеспечивающий взаимодействие между компонентами программного обеспечения.
  \item UI (User Interface) — пользовательский интерфейс.
  \item JSX — синтаксис, объединяющий JavaScript и XML-подобную разметку, применяемый в React.
  \item Socket.IO — библиотека для организации двусторонних WebSocket-соединений между клиентом и сервером с автоматическим фоллбеком.
  \item CSRF (Cross-Site Request Forgery) — тип атаки, при которой злоумышленник вынуждает браузер пользователя выполнить нежелательный запрос от его имени; в веб-приложениях применяется защита с помощью специальных токенов.
  \item XSS (Cross-Site Scripting) — уязвимость, позволяющая внедрять скрипты стороннего происхождения на страницы; одной из мер защиты является хранение JWT в HTTP-only cookie.
  \item SSO (Single Sign-On) — единый вход, позволяющий пользователю аутентифицироваться один раз и использовать доступ ко множеству сервисов без повторной авторизации.
  \item HTTP (Hypertext Transfer Protocol) — протокол передачи гипертекстовых данных между клиентом и сервером; используется для REST-запросов.
  \item REST (Representational State Transfer) — архитектурный стиль взаимодействия с API через HTTP-методы (GET, POST и др.).
  \item JSON (JavaScript Object Notation) — текстовый формат обмена данными, широко используемый в REST-API и для передачи полезной нагрузки токенов.
  \item HTML (HyperText Markup Language) — язык разметки веб-страниц, создающий структуру документа.
  \item CSS (Cascading Style Sheets) — язык каскадных таблиц стилей для описания внешнего вида HTML-элементов.
  \item SCSS (Sassy CSS) — расширение CSS с поддержкой переменных, вложенности и других возможностей препроцессора.
  \item CRUD (Create, Read, Update, Delete) — базовые операции над данными, которые выполняются при создании, чтении, обновлении и удалении записей.
  \item CLI (Command Line Interface) — интерфейс командной строки, используемый для генерации проекта и выполнения скриптов (например, Angular CLI или Next.js CLI).
  \item MVP (Minimum Viable Product) — минимально жизнеспособный продукт, концепция, описывающая начальную версию продукта с базовым функционалом.
\end{itemize}
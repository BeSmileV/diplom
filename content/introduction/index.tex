\newpage
\ESKDthisStyle{formII}
\section*{Введение}
\addcontentsline{toc}{section}{Введение}

Развитие цифровых технологий в сфере образования значительно меняет способы взаимодействия между преподавателями и студентами, предоставляя новые возможности для обучения и обмена информацией. В условиях дистанционного и смешанного обучения особенно важной становится необходимость создания платформ, которые бы объединяли образовательные инструменты в едином пространстве. Веб-приложения, которые решают задачи взаимодействия, позволяют сократить барьеры между преподавателями и студентами, улучшить коммуникацию и повысить качество образования. Цифровая среда должна обеспечивать не только размещение учебных материалов и заданий, но и средства для общения, автоматической оценки и анализа решений с использованием современных технологий, включая искусственный интеллект.

\textbf{Актуальность} темы заключается в потребности создания интегрированной образовательной платформы, которая объединяет функции чатов, проведения занятий и автоматического анализа решений, используя возможности ИИ. На данный момент отсутствует единая система, которая бы эффективно сочетала в себе эти ключевые аспекты: возможность общения через чаты, создание заданий и автоматизированную проверку решений с помощью ИИ. Современные платформы, как правило, фрагментированы — отдельные системы для чатов, другие для размещения заданий, третьи для автоматической проверки кода, что значительно усложняет организацию учебного процесса и снижает его эффективность. Разработка интегрированного решения, которое объединило бы эти элементы, позволяет улучшить качество образовательного процесса, повысив продуктивность студентов и преподавателей, а также упростив взаимодействие и автоматизировав многие рутинные задачи.

\textbf{Целью} данной работы является разработка клиентской части образовательной платформы, которая будет включать функции взаимодействия между преподавателями и студентами, автоматизированную проверку кода, а также возможности общения в рамках чатов. Особое внимание уделяется созданию такого интерфейса, который позволит преподавателям и студентам взаимодействовать в едином пространстве, где будут доступны все образовательные инструменты и ресурсы.

\textbf{Для достижения поставленной цели необходимо решить следующие задачи:}
\begin{enumerate}
\item Проанализировать предметную область и существующие системы, выявив их сильные и слабые стороны.
\item Определить архитектурные и технологические решения, подходящие для реализации клиентской части платформы.
\item Спроектировать пользовательский интерфейс, обеспечивающий интуитивное и удобное взаимодействие для преподавателей и студентов.
\item Разработать компоненты для управления учебными структурами (институт, кафедра, группа), заданиями и чатами.
\item Интегрировать средства для автоматизированной проверки решений студентов с применением ИИ.
\item Реализовать тестирование бизнес-логики приложения для обеспечения её корректности и эффективности.
\end{enumerate}

\textbf{Структура пояснительной записки} включает следующие разделы:
\begin{itemize}
\item В первом разделе рассматриваются особенности предметной области, проводится анализ существующих решений и обоснование выбора технологий и методов проектирования. Приводится обзор существующих образовательных платформ и их недостатков, а также объясняется необходимость разработки интегрированного решения.
\item Во втором разделе описывается архитектура клиентской части приложения, структура пользовательского интерфейса, проектирование компонентов и их взаимодействие. Рассматриваются решения для реализации системы чатов, создания и проверки заданий, а также интеграции ИИ-анализа.
\item В третьем разделе приводится описание реализации: структура кода, используемые технологии (Next.js, React, TypeScript, Redux, Auth.js), описание экрана и взаимодействий, примеры реализации различных компонентов системы.
\item В заключении приводятся выводы по выполненной работе, оценивается эффективность разработанного интерфейса и функционала, а также определяются направления для дальнейшего развития и улучшения системы. Указываются перспективы внедрения ИИ в образовательные платформы для улучшения процессов оценки и взаимодействия.
\end{itemize}

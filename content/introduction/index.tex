\newpage
\ESKDthisStyle{formII}
\section*{ВВЕДЕНИЕ}
\addcontentsline{toc}{section}{ВВЕДЕНИЕ}
\ESKDcolumnII{ВВЕДЕНИЕ}

Развитие цифровых технологий в сфере образования значительно меняет способы взаимодействия между преподавателями и студентами, предоставляя новые возможности для обучения и обмена информацией. В условиях дистанционного и смешанного обучения особенно важной становится необходимость создания приложения, которое бы объединяло образовательные инструменты в едином пространстве. Приложения, решающие задачи взаимодействия, позволяют сократить барьеры между преподавателями и студентами, улучшить коммуникацию и повысить качество образования. Цифровая среда должна обеспечивать не только размещение учебных материалов и заданий, но и средства для общения, автоматической оценки и анализа решений с использованием современных технологий, включая искусственный интеллект.

\textbf{Актуальность} темы заключается в потребности создания интегрированного образовательного Web–приложения, которое объединяет функции чатов, проведения занятий и автоматического анализа решений, используя возможности ИИ. Сейчас отсутствует единое решение, которое бы эффективно сочетало в себе эти ключевые аспекты: общения через чаты, создание заданий и автоматическую проверку решений с помощью ИИ. Современные системы, как правило, фрагментированы — отдельные модули для чатов, другие для размещения заданий, третьи для автоматической проверки кода, что значительно усложняет организацию учебного процесса и снижает его эффективность. Разработка интегрированного решения, которое объединит эти элементы, позволяет улучшить качество образовательного процесса, повысив продуктивность студентов и преподавателей, а также упростив взаимодействие и автоматизировав многие рутинные задачи.

\textbf{Целью} данной работы является улучшение и упрощение опыта работы преподавателей и процесса обучения студентов путем разработки front-end части интегрированного образовательного Web-приложения, обеспечивающего эффективное взаимодействие, автоматизированную проверку решений и удобные средства коммуникации.

\textbf{Для достижения поставленной цели необходимо решить следующие задачи:}

\begin{enumerate}
  \item Проанализировать предметную область и существующие системы, выявив их сильные и слабые стороны;
  \item Определить архитектурные и технологические решения, подходящие для реализации front-end части Web-приложения;
  \item Спроектировать пользовательский интерфейс, обеспечивающий интуитивное и удобное взаимодействие для преподавателей и студентов;
  \item Разработать компоненты для управления учебными структурами (институт, кафедра, группа), заданиями и чатами;
  \item Интегрировать средства для автоматизированной проверки решений студентов с применением ИИ;
  \item Реализовать тестирование бизнес-логики Web–приложения для обеспечения её корректности и эффективности.
\end{enumerate}

\textbf{Структура пояснительной записки} включает следующие разделы:
\begin{itemize}
  \item В первом разделе рассматриваются особенности предметной области, проводится анализ существующих решений и обоснование выбора технологий и методов проектирования;
  \item Во втором разделе описывается архитектура front-end части Web-приложения, структура пользовательского интерфейса, проектирование компонентов и их взаимодействие;
  \item В третьем разделе приводится описание реализации: структура кода, используемые технологии (Next.js, React, TypeScript, Redux, Auth.js), описание экранов и взаимодействий, примеры реализации различных компонентов Web-приложения;
  \item В заключении приводятся выводы по выполненной работе, оценивается эффективность разработанного интерфейса и функционала, а также определяются направления для дальнейшего развития и улучшения системы.
\end{itemize}

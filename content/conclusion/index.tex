\newpage
\ESKDthisStyle{formII}
\section*{ЗАКЛЮЧЕНИЕ}
\addcontentsline{toc}{section}{ЗАКЛЮЧЕНИЕ}
\ESKDcolumnII{ЗАКЛЮЧЕНИЕ}

В рамках проведённой дипломной работы было улучшено и упрощено взаимодействие преподавателей и студентов в образовательном процессе университета посредством разработки клиентской части интегрированного образовательного приложения. Основная цель заключалась в повышении качества опыта работы преподавателей и обучения студентов, обеспечивая единое пространство для коммуникаций, автоматизированной проверки заданий и управления учебными ресурсами.

Для достижения поставленной цели были выполнены следующие задачи:
\begin{enumerate}
  \item Проанализирована предметная область и существующие образовательные системы, выявлены их сильные и слабые стороны;
  \item Определены архитектурные и технологические решения (Next.js, React, TypeScript, Redux, Auth.js, Socket.IO) для реализации гибкой, надёжной и удобной клиентской части приложения;
  \item Спроектирован пользовательский интерфейс, обеспечивающий интуитивное и удобное взаимодействие для преподавателей и студентов (навигация, адаптивная вёрстка, единые UI-компоненты);
  \item Разработаны компоненты для управления учебными структурами (институты, кафедры, группы), заданиями и чатами, включая административную панель, что упростило работу преподавателей при ведении учебного процесса;
  \item Интегрированы средства автоматизированной проверки решений студентов с применением ИИ (DeepSeek) в модуле виртуальных классов, что повысило качество и скорость проверки лабораторных работ;
  \item Реализовано тестирование бизнес-логики клиентского приложения с использованием Jest, что подтвердило корректность и стабильность основных функций.
\end{enumerate}

Таким образом, все поставленные задачи выполнены, а основная цель достигнута: предложенные архитектурные решения обеспечили удобство работы преподавателей и улучшили опыт обучения студентов. Разработанный интерфейс продемонстрировал эффективность в упрощении коммуникаций, автоматизации рутинных задач и повышении качества образовательного процесса.

С практической точки зрения, данное клиентское приложение обладает следующими преимуществами:
\begin{enumerate}
  \item Быстрая разработка за счёт переиспользования компонентов и генерации форм, что позволяет преподавателям оперативно адаптировать интерфейс под учебный процесс;
  \item Надёжная безопасность благодаря механизму управления сессиями (JWT, Auth.js) и авторизации;
  \item Удобство сопровождения за счёт модульной структуры и чёткого разделения ответственности, что облегчает расширение и поддержку проекта;
  \item Расширяемость и готовность к новым сценариям благодаря гибкой конфигурации интерфейса, что позволяет быстро внедрять новые образовательные инструменты;
  \item Высокая устойчивость системы, подтверждённая результатами тестирования бизнес-логики, что гарантирует стабильность работы приложения для всех участников образовательного процесса.
\end{enumerate}

Перспективными направлениями дальнейшего развития являются:
\begin{enumerate}
  \item Расширение функционала анализа текста программ на основе искусственного интеллекта за счёт надстроек для автоматической проверки студенческих работ по заданным паттернам;
  \item Добавление офлайн-режима работы с последующей синхронизацией изменений при восстановлении связи, что улучшит опыт студентов в условиях нестабильного интернета;
  \item Разработка мобильных клиентских приложений для повышения доступности приложения и удобства работы преподавателей и студентов на разных устройствах;
  \item Внедрение системы аналитики пользовательской активности для оптимизации интерфейса, оценки эффективности учебного процесса и принятия обоснованных решений по улучшению образовательной среды.
\end{enumerate}

Итоги работы подтверждают, что предложенные решения действительно упростили работу преподавателей, улучшили опыт обучения студентов и повысили качество образовательного процесса. Открываются новые возможности для дальнейших исследований и совершенствования системы.

\newpage
\ESKDthisStyle{formII}
\section*{ЗАКЛЮЧЕНИЕ}
\ESKDcolumnII{ЗАКЛЮЧЕНИЕ}

В рамках проведённой дипломной работы была спроектирована и реализована клиентская часть веб-приложения для управления образовательным процессом в университете. Основная цель заключалась в создании масштабируемой и легко расширяемой архитектуры интерфейса, обеспечивающей высокую консистентность, читаемость и удобство поддержки. С этой целью были выполнены следующие ключевые мероприятия:

\begin{itemize}
  \item 
  Исследован и успешно применён модульный подход, позволивший чётко разделить проект на уровни ответственности и упростить навигацию по коду.

  \item 
  Разработана собственная библиотека пользовательского интерфейса, включающая набор визуальных компонентов и механизм автоматической генерации форм на основе декларативных описаний. Эта библиотека обеспечила единообразие стилистики и сократила время на создание новых интерфейсных элементов.

  \item 
  Внедрена система аутентификации и авторизации с поддержкой долгосрочных сессий, способная автоматически обновлять токены доступа без вмешательства пользователя и обеспечивать надёжную защиту маршрутов с учётом разных типов пользователей.

  \item 
  Реализован универсальный механизм регистрации для различных ролей участников (администрации университета, преподавателей и студентов), включающий проверку приглашений и гибкую обработку ошибок серверной части.

  \item 
  Создана административная панель с возможностями просмотра, создания, редактирования и удаления данных университета, включая управление структурой институтов, кафедр и учебных групп. Интерфейс обеспечил интуитивный доступ к основным операциям благодаря адаптированным формам и табличным представлениям.

  \item 
  Построен модуль виртуальных классов, обеспечивающий создание и организацию учебных групп, назначение заданий и сбор обратной связи. Интеграция автоматизированного анализа решений студентов с использованием AI-технологий позволила существенно повысить качество проверки кода лабораторных работ.

  \item 
  Разработана система мгновенного обмена сообщениями в реальном времени, включающая надёжное WebSocket-соединение, устойчивое к временным сбоям сети, а также удобные механизмы загрузки истории переписки в обе стороны и оптимистичного обновления интерфейса при отправке новых сообщений.
\end{itemize}

Проведённая работа подтверждает достижение поставленных задач: предложенные архитектурные решения обеспечили гибкость и расширяемость, а использование современных технологий и инструментов ускорило реализацию функционала и повысило устойчивость системы. 

С практической точки зрения, клиентское приложение демонстрирует следующие преимущества:
\begin{itemize}
  \item Повышенная скорость разработки за счёт переиспользования компонентов и генерации форм;
  \item Улучшенная безопасность благодаря продуманной схеме управления сессиями и авторизацией;
  \item Удобство сопровождения за счёт модульности и чёткого разделения ответственности;
  \item Расширяемость и поддержка новых сценариев благодаря гибкому подходу к конфигурации интерфейса.
\end{itemize}

В дальнейшем целесообразно рассмотреть следующие направления развития:
\begin{itemize}
  \item Расширение возможностей автоматизированного анализа решений с учётом различных языков программирования и более глубокой семантической проверки;
  \item Добавление функционала офлайн-режима с последующей синхронизацией изменений;
  \item Разработку мобильных клиентских приложений для повышения доступности и удобства работы на мобильных устройствах;
  \item Внедрение системы аналитики и мониторинга пользовательской активности для оптимизации интерфейса и оценки эффективности процессов обучения.
\end{itemize}

Подведённые итоги свидетельствуют о высокой эффективности предложенного подхода и открывают перспективы дальнейших исследований и развития системы.

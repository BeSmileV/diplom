\subsection{Проектирование взаимодействия с сервером и WebSocket}

Клиентская часть приложения активно взаимодействует с сервером для получения и отправки данных, а также поддерживает постоянное соединение с помощью WebSocket в рамках подсистемы обмена сообщениями. При проектировании механизма взаимодействия были учтены требования безопасности, стабильности соединения, обработки ошибок, а также необходимость автоматического обновления сессионных данных пользователя.

\subsubsection{Аутентификация и управление токенами}
Для обеспечения защищённого доступа к функциональности платформы используется система авторизации с применением JSON Web Token (JWT). Управление сессией пользователя реализовано через библиотеку \textit{Auth.js}, которая выполняет роль промежуточного слоя между клиентом и системой хранения токенов.

\begin{enumerate}
  \item Аутентификация пользователя:
  \begin{itemize}
    \item пользователь выполняет вход с помощью логина/пароля или через OAuth-провайдеров (например, Google);
    \item Auth.js инициирует процесс аутентификации и получает JWT при успешной проверке.
  \end{itemize}
  
  \item Работа с токеном:
  \begin{itemize}
    \item полученный JWT токен содержит минимальный необходимые полезные данные (например, идентификатор пользователя, роль и срок действия);
    \item токен сохраняется в \textit{HTTP-only} cookie с флагами \textit{Secure} и \textit{SameSite=Strict}, что предотвращает XSS- и CSRF-атаки.
  \end{itemize}
  
  \item Доступ к защищённым ресурсам:
  \begin{itemize}
    \item при обращении к серверу автоматически прикрепляется токен к запросу;
    \item при недействительном или истёкшем токене Auth.js обновляет его через refresh-токен, если он присутствует.
  \end{itemize}
\end{enumerate}

\subsubsection{Унифицированная функция отправки запросов}
Для унификации сетевых запросов была реализована функция \textit{sendRequest}, которая обеспечивает:

\begin{itemize}
  \item преобразование данных в JSON через \textit{JSON.stringify};
  \item добавление заголовков, включая \textit{Authorization: Bearer};
  \item обработка ошибок и повторная попытка после обновления токена;
  \item поддержка различных HTTP-методов.
\end{itemize}

Эта функция служит единой точкой для выполнения HTTP-запросов, упрощая их отправку и обработку.

\subsubsection{Взаимодействие через WebSocket}
Для реализации обмена сообщениями в реальном времени используется библиотека \textit{Socket.IO}, обеспечивающая:
\begin{itemize}
  \item автоматическое переподключение при обрыве соединения;
  \item передачу структурированных событий с именами и аргументами;
  \item интеграцию с промежуточным программным обеспечением для авторизации;
  \item работу с пространствами имён и комнатами;
  \item резревные транспорты при недоступности WebSocket.
\end{itemize}

На стороне клиента реализован хук \textit{useSocket}, который:
\begin{itemize}
  \item инициализирует соединение с сервером;
  \item отправляет и получает события с типизированными данными;
  \item подписывается и отписывается от каналов;
  \item управляет жизненным циклом подключения и логирует события;
  \item обрабатывает ошибки соединения.
\end{itemize}

При установлении WebSocket-соединения клиент передаёт токен доступа в параметрах, сервер проверяет его и активирует соединение. В случае истечения срока действия токена:
\begin{enumerate}
  \item инициируется его обновление;
  \item текущее соединение закрывается;
  \item создаётся новое соединение с обновлённым токеном.
\end{enumerate}

\subsubsection*{Вывод}

Реализованные механизмы автоматического обновления токенов, единая функция отправки запросов и продуманная интеграция WebSocket через \textit{Socket.IO} обеспечивают безопасность, надёжность и масштабируемость взаимодействия клиентской части с сервером в режиме реального времени.

\subsection{Проектирование взаимодействия с сервером и WebSocket}

Клиентская часть приложения активно взаимодействует с сервером для получения и отправки данных, а также поддерживает постоянное соединение с помощью WebSocket в рамках подсистемы обмена сообщениями. При проектировании механизма взаимодействия были учтены требования безопасности, стабильности соединения, обработки ошибок, а также необходимость автоматического обновления сессионных данных пользователя.

\subsubsection{Аутентификация и управление токенами}

Для обеспечения защищённого доступа к функциональности платформы используется система авторизации с применением JSON Web Token (JWT). Управление сессией пользователя реализовано через библиотеку \texttt{Auth.js}, которая выполняет роль промежуточного слоя между клиентом и системой хранения токенов.

Поскольку access token не хранится в открытом виде на стороне клиента, его получение и обновление возможны только через специальные механизмы обращения к серверу. При первичном входе пользователя токен сохраняется в защищённой cookie. Для последующего взаимодействия клиенту необходимо перед каждым сетевым запросом удостовериться в актуальности токена.

В рамках проектирования был реализован механизм автоматической валидации access token. Каждый раз перед отправкой запроса выполняется проверка срока его действия. В случае, если срок истёк, инициируется обращение к серверу с целью его обновления через \texttt{Auth.js}. После получения нового токена он автоматически обновляется в cookie, и запрос выполняется повторно.

Дополнительно, учитывая асинхронную природу работы клиента, был предусмотрен механизм защиты от многократного обновления токена в случае параллельных запросов. Реализована единая точка обращения к логике проверки и обновления токена. Если несколько запросов запускаются одновременно, и access token требует обновления, то все они получают результат единого процесса рефреша, исключая избыточные сетевые обращения. Это позволяет сократить нагрузку на сервер и избежать конфликтов при замене токенов.

\subsubsection{Унифицированная функция отправки запросов}

Для стандартизации сетевого взаимодействия была разработана функция \texttt{sendRequest}, инкапсулирующая всю логику подготовки и отправки запросов к серверу. Она реализует следующие функции:
\begin{itemize}
  \item Преобразование данных из формата JavaScript в JSON (\texttt{JSON.stringify});
  \item Добавление заголовков, включая авторизационный \texttt{Authorization: Bearer};
  \item Обработка возможных ошибок и повторная попытка в случае обновления токена;
  \item Поддержка различных HTTP-методов.
\end{itemize}

Данный подход позволяет централизованно управлять всей логикой сетевого взаимодействия и снижает вероятность ошибок при интеграции новых клиентских модулей.

\subsubsection{Взаимодействие через WebSocket}

Для реализации функциональности обмена сообщениями и других сценариев, требующих обновления данных в режиме реального времени, в клиентской части приложения применяется технология WebSocket. В отличие от традиционного HTTP-взаимодействия, WebSocket обеспечивает постоянное двустороннее соединение между клиентом и сервером, позволяя оперативно обмениваться событиями без необходимости постоянного опроса сервера.

В рамках проекта используется библиотека \texttt{Socket.IO}, которая предоставляет высокоуровневую обёртку над стандартным WebSocket-протоколом и значительно упрощает реализацию клиентского взаимодействия. Преимуществами \texttt{Socket.IO} являются:
\begin{itemize}
  \item Поддержка автоматического переподключения при обрыве соединения;
  \item Передача структурированных событий с именами и аргументами;
  \item Интеграция с механизмами авторизации и middleware;
  \item Гибкость при работе с пространствами имён и комнатами;
  \item Совместимость с fallback-транспортами (в случае недоступности WebSocket).
\end{itemize}

На стороне клиента был разработан и реализован специализированный хук \texttt{useSocket}, инкапсулирующий всю логику работы с соединением. Он обеспечивает:
\begin{itemize}
  \item Инициализацию соединения с сервером по заданному адресу;
  \item Отправку и приём событий с типизированной структурой данных;
  \item Автоматическую подписку и отписку от необходимых каналов;
  \item Управление жизненным циклом подключения;
  \item Обработку ошибок и логирование сетевых событий.
\end{itemize}

Для каждого подключённого пользователя создаётся уникальное пространство взаимодействия, определяемое его ролью и идентификатором. Это позволяет реализовать маршрутизацию сообщений между конкретными участниками чата (включая как личные, так и групповые диалоги), а также централизованно управлять доступом к отдельным коммуникационным потокам.

Особое внимание в проектировании было уделено вопросу авторизации в рамках WebSocket-сессии. В момент установления соединения клиент передаёт access token в параметрах подключения. На стороне сервера выполняется проверка подлинности токена, после чего соединение активируется. Однако, с учётом ограниченного срока действия access token, реализована логика автоматического переподключения. При обнаружении истечения токена:
\begin{enumerate}
  \item Инициируется обновление токена через заранее определённый механизм;
  \item Закрывается текущее соединение;
  \item После получения нового токена создаётся новое подключение с обновлёнными параметрами.
\end{enumerate}

Таким образом, WebSocket-подсистема функционирует устойчиво и прозрачно для конечного пользователя, обеспечивая бесперебойную передачу сообщений даже в условиях потери соединения или истечения сессии. Благодаря использованию \texttt{Socket.IO} удалось добиться высокой надёжности, расширяемости и лёгкости сопровождения реализации в рамках модульной архитектуры клиентской части.

\subsubsection{Вывод}

Взаимодействие клиентской части с сервером реализовано с учётом требований к безопасности, надёжности и масштабируемости. Реализованные механизмы автоматического обновления токенов, централизованная отправка запросов и продуманная интеграция WebSocket-соединения позволяют обеспечить стабильную работу интерфейса и корректную обработку всех пользовательских сценариев в режиме реального времени.

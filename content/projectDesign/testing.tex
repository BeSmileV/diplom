\subsection{Покрытие бизнес-логики юнит-тестами}

Наш подход к обеспечению надёжности клиентского приложения фокусируется на обязательном юнит-тестировании бизнес-логики при помощи Jest. Тестирование UI-компонентов считается вторичным: написание и поддержка сравнений HTML-вывода часто оказывается более трудоёмким и хрупким, чем простая визуальная валидация. Визуальный осмотр интерфейса преподавателем или дизайнером даёт более быстрый и надёжный результат без лишних накладных расходов.

Основные принципы нашего подхода:
\begin{enumerate}
  \item Юнит-тесты покрывают функции, отвечающие за валидацию данных, расчёт оценок и другие критичные механизмы, гарантируя корректность работы независимо от изменений UI;
  \item Модульные тесты интерфейсов не используются: динамика верстки и частые мелкие правки приводят к избыточным провалам тестов и дополнительным усилиям на их поддержку;
  \item Благодаря отказу от snapshot-тестирования HTML структура текста программы остаётся гибкой, а команда освобождает время на развитие функциональности вместо постоянной правки тестов;
  \item Автоматический запуск тестов бизнес-логики при каждом пуше позволяет мгновенно обнаруживать регрессии и поддерживать стабильность продукта;
  \item Для окончательной валидации интерфейса используется ручной осмотр ключевых страниц после сборки, что даёт уверенность в корректности отображения без сложных технических средств.
\end{enumerate}

Такой подход обеспечивает надёжность самой логики приложения и упрощает работу с UI: вместо громоздких автоматизированных тестов на вёрстку мы применяем человеческую экспертизу для финальной проверки внешнего вида и пользовательского опыта.

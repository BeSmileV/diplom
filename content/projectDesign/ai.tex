\subsection{Интеграция ИИ-модуля DeepSeek в архитектуру проекта}

\subsubsection{Контейнеризация DeepSeek}
Для обеспечения независимого жизненного цикла и лёгкой масштабируемости ИИ-компонента DeepSeek развёртывается в виде изолированного Docker-контейнера. Такой подход позволяет:
\begin{itemize}
	\item Быстро запускать и останавливать сервис без влияния на основное приложение.
	\item Поддерживать разные версии DeepSeek параллельно, экспериментируя с обновлениями моделей.
	\item Мигрировать между хостами и облачными средами с минимальными изменениями конфигурации.
\end{itemize}
Контейнер содержит все необходимые зависимости: предобученные трансформерные модели, библиотеки для анализа AST и семантической обработки, а также механизмы формирования отчётов. При этом фронтенд остаётся полностью изолированным от этих деталей — ему достаточно знать только адрес и формат запросов.

\subsubsection{Взаимодействие клиентской части с DeepSeek}
Клиентская часть приложения, реализованная на React и Next.js, отправляет HTTP-запросы через единый API-шлюз. Основная логика взаимодействия упрощена до следующих действий:
\begin{itemize}
	\item Преподаватель выбирает работу студента и нажатием кнопки инициирует анализ.
	\item Фронтенд направляет корректно сформированный запрос к API, где указаны идентификатор работы и необходимые параметры оценки.
	\item По готовности отчёта фронтенд периодически проверяет статус анализа и загружает результаты в привычном формате.
	\item Все шаги интегрированы в единый пользовательский поток: преподаватель не видит инфраструктуры контейнеров, а получает только готовый отчёт.
\end{itemize}
Такая схема взаимодействия гарантирует простоту клиентской логики и минимальную связность с внутренним устройством ИИ-сервиса.

\subsubsection{Преимущества контейнеризированного подхода}
Контейнеризация DeepSeek даёт следующие ключевые плюсы:
\begin{itemize}
	\item \textbf{Изоляция нагрузки:} анализ кода выполняется в отдельном окружении, не влияя на отзывчивость пользовательского интерфейса.
	\item \textbf{Горизонтальное масштабирование:} при необходимости обрабатывать большое число запросов можно запускать несколько инстансов контейнера.
	\item \textbf{Упрощённое сопровождение:} обновление ИИ-компонента сводится к выпуску нового Docker-образа без правок в фронтенде.
	\item \textbf{Гибкость развертывания:} контейнеры можно запускать как локально, так и в облаке, сохраняя одинаковую конфигурацию.
\end{itemize}


\subsubsection{Вывод}
Таким образом, архитектура остаётся прозрачной на уровне клиентского приложения, а DeepSeek надёжно инкапсулирован и готов к дальнейшему эволюционному развитию.

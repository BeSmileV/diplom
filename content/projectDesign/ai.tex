\subsection{Интеграция ИИ-модуля DeepSeek в архитектуру проекта}

\subsubsection{Контейнеризация DeepSeek}
Для обеспечения независимого жизненного цикла и лёгкой масштабируемости ИИ-компонента DeepSeek развёртывается в виде изолированного Docker-контейнера. Такой подход позволяет:
\begin{itemize}
  \item быстро запускать и останавливать сервис без влияния на основное приложение;
  \item поддерживать разные версии DeepSeek параллельно, экспериментируя с обновлениями моделей;
  \item мигрировать между хостами и облачными средами с минимальными изменениями конфигурации.
\end{itemize}

\subsubsection{Взаимодействие клиентской части с DeepSeek}
Клиентская часть приложения, реализованная на React и Next.js, отправляет HTTP-запросы через единый API-шлюз. Основная логика взаимодействия упрощена до следующих действий:
\begin{itemize}
  \item преподаватель выбирает работу студента и нажатием кнопки инициирует анализ;
  \item фронтенд направляет корректно сформированный запрос к API с указанием идентификатора работы и параметров оценки;
  \item по готовности отчёта фронтенд периодически проверяет статус анализа и загружает результаты;
  \item все шаги интегрированы в единый пользовательский поток: преподаватель получает только готовый отчёт без знания инфраструктуры контейнеров.
\end{itemize}

\subsubsection{Преимущества контейнеризированного подхода}
Контейнеризация DeepSeek даёт следующие ключевые плюсы:
\begin{itemize}
  \item \textbf{Изоляция нагрузки}: анализ кода выполняется в отдельном окружении, не влияя на отзывчивость интерфейса;
  \item \textbf{Горизонтальное масштабирование}: при большом числе запросов можно запускать несколько инстансов контейнера;
  \item \textbf{Упрощённое сопровождение}: обновление ИИ-компонента сводится к выпуску нового Docker-образа без правок во фронтенде;
  \item \textbf{Гибкость развертывания}: контейнеры можно запускать локально и в облаке с одинаковой конфигурацией.
\end{itemize}

\subsubsection{Вывод}
Таким образом, архитектура остаётся прозрачной на уровне клиентского приложения, а DeepSeek надёжно инкапсулирован и готов к дальнейшему эволюционному развитию.

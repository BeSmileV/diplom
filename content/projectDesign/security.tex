\subsection{Обеспечение безопасности клиентской части}

Безопасность пользовательского взаимодействия является важнейшей составляющей архитектуры клиентской части платформы. В условиях, когда доступ к различным модулям приложения осуществляется на основе ролей, а взаимодействие с данными сопровождается отображением пользовательского контента, особое внимание уделяется как управлению доступом, так и защите от потенциальных атак, включая межсайтовое выполнение скриптов (XSS). В данной подсистеме реализован комплекс механизмов, направленных на защиту данных и поведения интерфейса со стороны клиента.

\subsubsection{Механизм разграничения доступа с использованием Middleware}

Одним из ключевых компонентов обеспечения безопасности клиентской части является система контроля доступа на основе промежуточного слоя — \texttt{middleware}. В рамках архитектуры \texttt{Next.js}, middleware представляет собой функцию, исполняемую при каждом запросе к защищённым маршрутам. Она позволяет перехватывать обращения к страницам до их рендеринга и на этой стадии выполнять необходимые проверки: наличие токена, его валидность, а также права пользователя.

В контексте реализуемой платформы, при обращении пользователя к любой защищённой странице клиентская логика через middleware извлекает JWT-токен из cookies и дешифрует его содержимое, получая так называемый payload — полезную нагрузку токена. Внутри неё содержится вся необходимая информация о пользователе: уникальный идентификатор, срок действия сессии и, что особенно важно, роль в системе (\texttt{admin}, \texttt{teacher}, \texttt{student}).

На основе этой информации middleware выполняет следующие действия:
\begin{itemize}
  \item Если пользователь не авторизован (отсутствует валидный токен) — происходит автоматический редирект на страницу входа.
  \item Если пользователь авторизован, но не обладает достаточными правами — осуществляется перенаправление на главную страницу или отображается сообщение об отказе в доступе.
  \item Если пользователь обладает необходимой ролью — доступ к ресурсу предоставляется, и страница загружается с соответствующим контентом.
\end{itemize}

Таким образом, механизм \texttt{middleware} обеспечивает надёжную фильтрацию обращений к различным частям интерфейса, предотвращая несанкционированный доступ и обеспечивая соответствие поведения приложения политике разграничения прав.

\subsubsection{Роль и защита при работе с форматируемым текстом}

Дополнительным вектором потенциальной угрозы в клиентских приложениях является отображение форматируемого текста, особенно если пользователь имеет возможность редактировать его содержимое. В таких случаях возрастает риск внедрения вредоносных скриптов, замаскированных под обычный HTML.

Для решения данной задачи в проекте используется специализированная библиотека \texttt{tiptap} — расширяемый редактор форматированного текста, основанный на \texttt{ProseMirror}. Одним из ключевых преимуществ \texttt{tiptap} является контроль над тем, какие HTML-теги и атрибуты допускаются к интерпретации и отображению. Таким образом, даже если пользователь попытается вставить опасный HTML-код (например, \texttt{<script>} или инъекцию с обработчиком событий), редактор проигнорирует или удалит такие элементы на этапе парсинга.

Технически это реализуется следующим образом:
\begin{itemize}
  \item При вводе текстов редактор не сохраняет «сырые» HTML-строки, а формирует безопасное представление контента на основе строго описанных схем;
  \item При рендеринге текста из базы или состояния редактор отображает только те элементы, которые были описаны как допустимые;
  \item Расширения (extensions), добавляемые к \texttt{tiptap}, позволяют точно контролировать список разрешённых действий и структур (например, разрешить только \texttt{<strong>}, \texttt{<em>}, \texttt{<ul>} и \texttt{<code>}).
\end{itemize}

Таким образом, даже при наличии активной формы редактирования форматируемого текста, пользовательская среда остаётся защищённой от внедрения опасного контента. Редактор фактически выступает в роли фильтра, строго ограничивающего возможный набор HTML-инструкций.

\subsubsection{Вывод}

Комплекс реализованных решений позволяет эффективно защитить клиентскую часть приложения как от внешнего вмешательства, так и от ошибочного доступа со стороны пользователей. Механизм middleware обеспечивает надёжную проверку подлинности сессии и прав доступа до загрузки страниц, а редактор \texttt{tiptap} гарантирует безопасность при работе с форматируемым текстом. Такое сочетание архитектурных и прикладных решений способствует созданию устойчивой и безопасной пользовательской среды.

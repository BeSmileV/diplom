\subsection{Проектирование интерфейсных подсистем и экранов}

Одной из ключевых задач при проектировании клиентской части является логическое и функциональное разделение интерфейса на подсистемы, каждая из которых реализует отдельный аспект пользовательского взаимодействия. Такое разделение позволяет обеспечить модульность, переиспользуемость компонентов и устойчивость к изменениям.

Проект разрабатывается в архитектуре Feature-Sliced Design, что накладывает дополнительную дисциплину на организацию экранов и компонентов: все подсистемы формируются из \texttt{entities}, \texttt{features}, \texttt{widgets} и собираются в \texttt{pages}, а общая инфраструктура — в слое \texttt{shared}.

\subsubsection{Выделение ключевых интерфейсных подсистем}

Клиентская часть разработанной платформы организована в виде набора функционально обособленных интерфейсных подсистем, каждая из которых отвечает за определённый аспект пользовательского взаимодействия и бизнес-логики. Такое разграничение позволяет повысить масштабируемость и сопровождаемость системы, а также упростить процесс тестирования и внедрения новых функций.

На основании анализа требований к функциональности приложения и сценариев использования пользователями различных ролей (администратор, преподаватель, студент), были выделены следующие ключевые подсистемы.

\begin{enumerate}
  \item \textbf{Подсистема авторизации и регистрации}\\
  Отвечает за обеспечение безопасного входа в систему, регистрацию новых пользователей и управление сессиями. Аутентификация реализована с применением библиотеки \texttt{Auth.js} и технологии JSON Web Token (JWT), что позволяет надёжно разграничивать доступ к различным разделам интерфейса в зависимости от роли пользователя.  

  Регистрация в системе представлена в виде трёх пользовательских сценариев, адаптированных под особенности образовательного процесса:
  \begin{itemize}
    \item Первый сценарий реализован для новых организаций (институтов) и сопровождается созданием административной учётной записи. На этом этапе формируется корневая структура управления учреждением.
    \item Второй и третий сценарии предназначены для регистрации преподавателей и студентов соответственно. Оба сценария доступны исключительно по индивидуальным приглашениям, что обеспечивает контроль над составом участников образовательного процесса и предотвращает несанкционированный доступ.
  \end{itemize}
  
  Подсистема тесно связана с механизмами контроля прав доступа и маршрутизации, определяя поведение интерфейса в зависимости от текущего статуса пользователя.

  \item \textbf{Подсистема управления университетом}\\
  Реализует административную логику, связанную с конфигурацией организационной структуры образовательного учреждения.
  
  Основными функциями данной подсистемы являются:
  \begin{itemize}
    \item Создание и удаление структурных единиц — институтов, кафедр, учебных групп;
    \item Управление персоналом: добавление и блокировка преподавателей и студентов;
    \item Генерация приглашений для входа новых участников на платформу с конкретной ролью;
    \item Отображение данных по структуре учреждения.
  \end{itemize}
  
  Визуально подсистема представлена в виде панели управления с множеством таблиц, форм и интерактивных элементов, обеспечивающих быстрый доступ к ключевым административным операциям. Все действия защищены авторизацией и доступны только пользователям с соответствующими правами доступа.

  \item \textbf{Подсистема работы с заданиями и отправкой решений}\\
  Данная подсистема предназначена для организации учебной деятельности: выдачи заданий преподавателями, загрузки решений студентами и автоматизированного анализа этих решений. 
  
  Основной интерфейс включает:
  \begin{itemize}
    \item Панель создания и редактирования заданий с параметрами проверки;
    \item Представление активных и завершённых заданий для студентов;
    \item Историю отправок с отображением результатов и статуса проверки.
  \end{itemize}
  
  Задания связаны с группами. Система также предоставляет базовую аналитику по результатам выполнения.

  \item \textbf{Подсистема обмена сообщениями (чаты)}\\ 
  В рамках образовательного процесса большое значение имеет возможность коммуникации. Подсистема реализует обмен сообщениями как в рамках учебной группы, так и в формате личной переписки.  
  Технически реализация основана на технологии WebSocket с использованием библиотеки \texttt{Socket.IO}, что обеспечивает мгновенную доставку сообщений и минимальную задержку при передаче данных. 
  
  Основной функционал включает:
  \begin{itemize}
    \item Подключение к соответствующим «комнатам» (группам или диалогам);
    \item Отправку и приём текстовых сообщений;
    \item Отображение истории переписки;
    \item Поддержку вложений и индикаторов прочтения.
  \end{itemize}
  
  Доступ к системе чатов осуществляется только после успешной авторизации, что исключает участие анонимных пользователей и обеспечивает безопасность переписки.

  \item \textbf{Подсистема AI-анализа решений}\\
  Одной из уникальных особенностей платформы является использование искусственного интеллекта для автоматической оценки студенческих заданий.
  
  Подсистема предназначена для получения и визуализации результатов AI-анализа, включающих:
  \begin{itemize}
    \item Оценку корректности кода;
    \item Проверку на соответствие заданию;
    \item Выявление потенциальных ошибок и некорректных конструкций;
    \item Комментарии, рекомендации и текстовые пояснения.
  \end{itemize}
  
  Результаты анализа отображаются в виде отчёта с возможностью преподавателя оставить дополнительные замечания. Таким образом, снижается нагрузка на преподавателя и повышается объективность оценивания.

\end{enumerate}

Каждая из указанных подсистем обладает чётко определёнными входными и выходными данными, а также взаимодействует с другими модулями системы. Например, подсистема работы с заданиями напрямую связана как с AI-анализом, так и с интерфейсами преподавателя и студента, а система чатов — с механизмами авторизации и маршрутизации. Такое проектирование обеспечивает гибкость, надёжность и чёткую масштабируемость клиентской архитектуры.


\subsubsection{Страницы и их структура}

Разработка интерфейсной части веб-приложения требует не только реализации функциональных компонентов, но и проектирования логически связанных экранов, отражающих ключевые сценарии взаимодействия пользователя с системой. В рамках платформы каждая страница представляет собой самостоятельный интерфейсный модуль, обслуживающий одну или несколько бизнес-задач, соответствующих определённой роли: студент, преподаватель, администратор.

Процесс формирования страниц реализован с применением маршрутизации, встроенной в фреймворк \texttt{Next.js}, что обеспечивает высокую производительность и поддержку серверного рендеринга. Страницы не только представляют визуальный уровень приложения, но и координируют работу между компонентами пользовательского интерфейса, бизнес-логикой и хранилищем состояния. 

Архитектурно страницы собираются из обособленных функциональных элементов, разработанных согласно принципам FSD: пользовательские действия реализуются в слое \texttt{features}, отображаемые сущности формируются на базе \texttt{entities}, а объединение этих блоков происходит внутри \texttt{widgets}. Такой подход позволяет повысить согласованность, переиспользуемость и модульность кода, а также снижает зависимость между различными частями интерфейса.

Ниже приведён перечень ключевых страниц, отражающих основную логику пользовательского взаимодействия.

\begin{itemize}
  \item \textbf{Страница авторизации}\\  
  Отвечает за вход пользователя в систему. Содержит форму для ввода учётных данных, а также реализует логику валидации, передачи данных на сервер, обработки ошибок и сохранения сессионного токена. После успешной авторизации пользователь перенаправляется на главную страницу, соответствующую его роли.

  \item \textbf{Страница заданий}\\
  Представляет собой ключевой интерфейс для организации и выполнения учебной деятельности. Интерфейс страницы включает:
  \begin{itemize}
    \item Список классов и учебных групп, к которым привязан пользователь;
    \item Перечень активных заданий в рамках каждой группы;
    \item Доступ к подробному описанию заданий, срокам сдачи и параметрам оценивания;
    \item Отправку решений и просмотр результатов, включая отчёты AI-анализа.
  \end{itemize}
  Для преподавателя дополнительно предоставляется интерфейс управления заданиями, а также доступа к аналитике по группам и студентам.

  \item \textbf{Административная панель института}\\
  Данная страница является основным рабочим инструментом пользователя с ролью администратора. Интерфейс включает:
  \begin{itemize}
    \item Управление иерархией образовательного учреждения (институты, кафедры, группы);
    \item Назначение и блокировка пользователей (студентов и преподавателей);
    \item Просмотр структуры учреждения в табличной форме;
    \item Генерация и отправка приглашений на регистрацию;
    \item Журнал событий и контроль активности пользователей.
  \end{itemize}
  Все действия на данной странице требуют повышенного уровня доступа и сопровождаются системой уведомлений о результатах операций.

  \item \textbf{Страница чатов}\\
  Реализует коммуникационную составляющую платформы. Пользователь получает доступ к:
  \begin{itemize}
    \item Перечню активных диалогов (личных и групповых);
    \item Истории сообщений в рамках выбранного чата;
    \item Форме для отправки сообщений и файлов;
    \item Интерактивным элементам: индикаторы доставки, статус прочтения, поиск по переписке.
  \end{itemize}
  Для преподавателей также предусмотрена возможность создания новых групповых чатов для своих учебных групп.

\end{itemize}

Все функциональные страницы приложения, за исключением экранов регистрации и входа, используют единый шаблон компоновки \texttt{AppLayout}, обеспечивающий целостность визуального восприятия и унификацию пользовательского опыта. Данный шаблон включает в себя общие элементы интерфейса — верхнюю панель навигации, боковое меню и основной контейнер для отображения содержимого, который динамически наполняется в зависимости от текущего маршрута. 

Использование общего каркаса позволяет сохранить структурную согласованность между различными разделами системы, облегчает адаптацию пользователей к интерфейсу и упрощает внедрение изменений. Кроме того, архитектурное разделение логики и представления на уровне страниц способствует инкапсуляции ответственности, а также повышает читаемость и сопровождаемость кода. В рамках маршрутизации обеспечивается централизованное управление доступом, фильтрацией и визуализацией данных с учётом ролей пользователей.

Таким образом, структура страниц приложения отражает как технические требования архитектуры, так и практическую ориентацию на удобство и эффективность работы конечных пользователей.

\subsubsection{Компоненты и принципы их структурирования}

Компонентная модель проекта выстроена на основе принципов повторного использования, инкапсуляции и чёткого разделения ответственности между уровнями абстракции. Все компоненты, применяемые в рамках клиентского интерфейса, условно делятся на два основных класса: общие (универсальные) и специфические (бизнес-ориентированные).

\begin{itemize}
  \item \textbf{Общие компоненты} (\texttt{shared/ui}) представляют собой переиспользуемые элементы пользовательского интерфейса, не зависящие от предметной области. К ним относятся кнопки, поля ввода, модальные окна, индикаторы загрузки, элементы навигации, уведомления и другие базовые визуальные элементы. Такие компоненты широко применяются на всех уровнях интерфейса и не содержат бизнес-логики.
  
  \item \textbf{Специфические компоненты}, разрабатываемые в слоях \texttt{entities} и \texttt{widgets}, предназначены для реализации прикладной логики и отображения конкретных сущностей системы. Примерами являются компоненты отображения сообщений в чате, карточек заданий, панели управления преподавателя, таблиц пользователей и др. Они обладают внутренним состоянием и часто включают обращение к хранилищу или API.
\end{itemize}

Такое структурное разграничение существенно упрощает масштабирование проекта, облегчает поддержку и повторное использование элементов, а также способствует разделению труда между разработчиками.

\subsubsection{Распределение логики по слоям архитектуры}

Функциональная логика клиентской части системы строго распределяется по слоям архитектуры Feature-Sliced Design, что обеспечивает высокую модульность и инкапсуляцию поведения. Каждому слою соответствует свой уровень ответственности:

\begin{itemize}
  	\item В слое \texttt{entities} сосредоточена модель предметной области: типизация, структура сущностей, атомарные компоненты отображения, такие как \texttt{Registration}, \texttt{Department}, \texttt{Group}. Данный слой реализует описание и базовое представление данных без привязки к конкретным действиям пользователя.
  
	\item Слой \texttt{features} содержит реализацию отдельных действий, составляющих пользовательские сценарии: отправка сообщений, регистрация, загрузка задания, подтверждение действия и т.д. Эти модули инкапсулируют конкретные шаги взаимодействия пользователя с интерфейсом, часто включая локальное состояние и вызовы к API. \texttt{Features} могут быть использованы многократно и комбинироваться для построения более сложных сценариев.
	
	\item Слой \texttt{widgets} представляет собой реализацию полноценных пользовательских сценариев — законченных интерфейсных блоков, решающих определённую задачу. Примеры: интерфейс чата, панель с заданиями, административный модуль управления группами. Каждый виджет объединяет несколько фич и сущностей, обеспечивая завершённую и логически связанную единицу поведения.
	
	\item Слой \texttt{pages} выполняет роль точки входа и финальной сборки пользовательских сценариев. Здесь происходит выбор и компоновка виджетов в зависимости от маршрута, роли пользователя и контекста сессии. Кроме того, на уровне страниц задаются глобальные обёртки, обеспечиваются ограничения доступа, инициализируются загрузки данных и подключаются необходимые провайдеры. Таким образом, \texttt{pages} являются связующим слоем между навигацией и пользовательским опытом.
\end{itemize}

Такое строгое распределение обязанностей по слоям позволяет исключить дублирование логики, минимизировать связанность между модулями и обеспечить чёткую иерархию ответственности.

\subsubsection{UX-решения и пользовательские сценарии}

Для повышения удобства и доступности платформы, особенно в условиях использования её разными категориями пользователей, были реализованы следующие решения в области пользовательского опыта (UX):

\begin{itemize}
  \item \textbf{Централизованная навигация} — через универсальный макет, включающий боковую и верхнюю панели, интерфейс остаётся единообразным и интуитивно понятным вне зависимости от текущего маршрута.
  \item \textbf{Toast-уведомления} — реализация мгновенной обратной связи при выполнении действий: успешная отправка формы, ошибка сети, получение новых сообщений.
  \item \textbf{Обработка пустых состояний и ошибок} — предусмотрены интерфейсы для ситуаций отсутствия данных, ошибок загрузки или недоступности сервера.
 \end{itemize}

В результате, пользователь получает предсказуемый и непрерывный опыт взаимодействия с системой вне зависимости от своей роли и уровня подготовки.

\subsubsection*{Вывод}

Проектирование интерфейсной части приложения основывается на чётком структурном и функциональном разграничении компонентов, ориентированном на принципы модульности и масштабируемости. Использование архитектуры Feature-Sliced Design позволяет изолировать бизнес-логику, визуальные компоненты и маршрутизацию, что делает интерфейс легко расширяемым и сопровождаемым.

Реализованная организация интерфейса, объединяющая единый шаблон компоновки, повторно используемые компоненты и специфические бизнес-модули, способствует формированию целостного пользовательского опыта. Выбранные UX-решения обеспечивают удобство и логичность навигации, а также высокую отзывчивость системы при взаимодействии с пользователем.


Интерфейсная часть проекта построена на модульной архитектуре, основанной на бизнес-функциях. Подсистемы выделены логически, а их реализация изолирована в независимые модули, что повышает удобство поддержки, расширения и переиспользования компонентов.

\subsection{Архитектура клиентской части системы}

В качестве архитектурного подхода к проектированию клиентской части приложения выбран \textbf{Feature-Sliced Design (FSD)} — современный метод модульной архитектуры, ориентированный на frontend-приложения, использующие стек React, Redux, TypeScript, Next.js и др.

\subsubsection{Общее описание Feature-Sliced Design}

Feature-Sliced Design (FSD) — это архитектурный подход, в котором структура проекта строится вокруг \textbf{бизнес-логики и функциональных сценариев}, а не вокруг технических понятий вроде компонентов или страниц. Вместо вертикального деления на слои, FSD предлагает горизонтальное деление на сегменты (срезы), соответствующие смысловым блокам приложения.

Целью FSD является обеспечение \textbf{масштабируемости, читаемости и модульности} клиентской части. Он упрощает командную разработку, делает код более устойчивым к изменениям и облегчает тестирование.

\subsubsection{Слои архитектуры FSD}

В FSD определены следующие ключевые слои, каждый из которых имеет своё назначение и строгие границы ответственности:

\begin{table}[H]
\centering
\caption{Слои архитектуры FSD}
\begin{tabular}{|p{3cm}|p{11cm}|}
\hline
\textbf{Слой} & \textbf{Назначение} \\
\hline
\texttt{app} & Точка входа в приложение, глобальные конфигурации, роутинг, провайдеры \\
\hline
\texttt{processes} & Бизнес-процессы, объединяющие несколько фич в сценарии (опционально) \\
\hline
\texttt{pages} & Отдельные страницы приложения, собирают из виджетов и фич \\
\hline
\texttt{widgets} & Крупные интерфейсные блоки, объединяющие фичи и сущности \\
\hline
\texttt{features} & Отдельные бизнес-функции: авторизация, отправка сообщений, регистрация и т.д. \\
\hline
\texttt{entities} & Базовые предметные сущности: пользователь, задание, сообщение и т.д. \\
\hline
\texttt{shared} & Универсальные модули: UI-компоненты, утилиты, темы, типы и пр. \\
\hline
\end{tabular}
\end{table}

\subsubsection{Описание слоёв}

\paragraph{Слой \texttt{app}.}  
Содержит глобальную инициализацию: конфигурацию провайдеров (\texttt{AuthProvider}, \texttt{StoreProvider}), роутинг, глобальные стили и интеграцию с внешними сервисами.

\paragraph{Слой \texttt{shared}.}  
Хранилище нейтральных, переиспользуемых компонентов и утилит, не зависящих от предметной области. Примеры: \texttt{Button}, \texttt{Input}, \texttt{formatDate()}, \texttt{UserRole}.

\paragraph{Слой \texttt{entities}.}  
Описывает предметные сущности и их внутреннюю логику. Каждая сущность содержит типизацию, модели, UI-компоненты (например, \texttt{UserCard}), работу с хранилищем данных.

\paragraph{Слой \texttt{features}.}  
Бизнес-функциональность, реализующая действия пользователя. Примеры: \texttt{sendMessage}, \texttt{submitTask}, \texttt{loginUser}. Каждая фича использует одну или несколько сущностей.

\paragraph{Слой \texttt{widgets}.}  
Крупные интерфейсные блоки, собирающие интерфейс из сущностей и фич. Например: \texttt{ChatWindow}, \texttt{TaskList}, \texttt{Sidebar}.

\paragraph{Слой \texttt{pages}.}  
Реализует страницы приложения. Каждая страница включает в себя нужные виджеты и фичи, но не содержит логики.

\paragraph{Слой \texttt{processes}.}  
(Опционально) Используется для объединения фич и сущностей в единые бизнес-сценарии, например: процесс регистрации нового пользователя.

\subsection{Пример структуры проекта}


\subsubsection{Преимущества применения FSD}

\begin{itemize}
  \item Чёткое разделение ответственности и слабая связность модулей;
  \item Масштабируемость проекта без деградации архитектуры;
  \item Упрощённое тестирование и переиспользование;
  \item Быстрое включение новых разработчиков в проект;
  \item Возможность гибкой миграции между проектами с похожей архитектурой.
\end{itemize}

Применение FSD позволило реализовать гибкую и масштабируемую архитектуру клиентской части, обеспечив стабильность, понятность и надёжность структуры при росте количества модулей и функционала.

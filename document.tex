% !TeX program = xelatex

%%% Загружаем заголовочный файл, который хранит все настройки и все
%%% подгружаемые пакеты
\newcommand{\No}{\textnumero}

%%% Здесь выбираются необходимые графы
\documentclass[russian,utf8,pointsection,nocolumnsxix,nocolumnxxxi,nocolumnxxxii]{eskdtext}
\usepackage{fontspec}
\defaultfontfeatures{Mapping=tex-text} % Для того чтобы работали стандартные сочетания символов ---, --, << >> и т.п.

%%% Что бы работал eskdx и некоторые другие пакеты LaTeX
\usepackage{xecyr}

%%% Для работы шрифтов
\usepackage{xunicode,xltxtra}

%%% Для работы с русскими текстами (расстановки переносов, последовательность комманд строго обязательна)
% \usepackage{polyglossia}
% \setdefaultlanguage{russian}
%\newfontfamily{\cyrillicfontt}{GOST_B}
%\set{Times New Roman}
\setmainfont{Times New Roman}
\setromanfont{Times New Roman} 
\setsansfont{Times New Roman} 
\setmonofont{Times New Roman} 

% polyglossia only
% \newfontfamily\cyrillicfont{GOST_type_A} 
% \newfontfamily\cyrillicfontrm{GOST_type_A}
% \newfontfamily\cyrillicfonttt{GOST_type_A}
% \newfontfamily\cyrillicfontsf{GOST_type_A}
%\defaultfontfeatures{Mapping=tex-text}

%%% Для работы со сложными формулами
\usepackage{amsmath}
\usepackage{amssymb}

%%% Что бы использовать символ градуса (°) - \degree
\usepackage{gensymb}


%%% Для переноса составных слов
%\XeTeXinterchartokenstate=1
\XeTeXcharclass `\- 24
\XeTeXinterchartoks 24 0 ={\hskip\z@skip}
\XeTeXinterchartoks 0 24 ={\nobreak}

%%% Ставим подпись к рисункам. Вместо «рис. 1» будет «Рисунок 1»
\addto{\captionsrussian}{\renewcommand{\figurename}{Рисунок}}
%%% Убираем точки после цифр в заголовках
\def\russian@capsformat{%
  \def\postchapter{\@aftersepkern}%
  \def\postsection{\@aftersepkern}%
  \def\postsubsection{\@aftersepkern}%
  \def\postsubsubsection{\@aftersepkern}%
  \def\postparagraph{\@aftersepkern}%
  \def\postsubparagraph{\@aftersepkern}%
}



% Автоматически переносить на след. строку слова которые не убираются
% в строке
\sloppy

%%% Для вставки рисунков
\usepackage{graphicx}

%%% Для вставки интернет ссылок, полезно в библиографии
\usepackage{url}

%%% Подподразделы(\subsubsection) не выводим в содержании
\setcounter{tocdepth}{2}

%%% Каждый раздел (\section) с новой страницы
\let\stdsection\section
\renewcommand\section{\newpage\stdsection}

%%% В введении нумерация подразделов идёт с буквой «В» (например В.1)
\makeatletter
\renewcommand\thesubsection{\ifnum\c@section=0{В.\arabic{subsection}}\else{\arabic{section}.\arabic{subsection}}\fi}
\makeatother

\usepackage{hyperref}

\usepackage{titlesec}

\titleformat{\section}[block]{\normalfont\Large\bfseries}{\thesection}{1em}{}
\titlespacing*{\section}{20pt}{14pt}{8pt} % слева/сверху/снизу

\titleformat{\subsection}[block]{\normalfont\large\bfseries}{\thesubsection}{1em}{}
\titlespacing*{\subsection}{20pt}{12pt}{6pt} % Уменьшаем отступ сверху до 0pt и снизу до 8pt

\titleformat{\subsubsection}[block]{\normalfont\normalsize\bfseries}{\thesubsubsection}{1em}{}
\titlespacing*{\subsubsection}{20pt}{10pt}{4pt} % Уменьшаем отступ сверху до 0pt и снизу до 6pt


%%% Загружаем настройки пакета eskdx, там нужно заполнить информацию
%%% о документе - ФИО авторов, название документов и т.п.
%%% Название документа
\ESKDtitle{ Название документа }
\ESKDdocName{ Название дипломной работы  Название дипломной работы     }

\renewcommand{\ESKDcolumnXfIIname}{Руковод.}
\renewcommand{\ESKDcolumnXfIVname}{Консул.}
\renewcommand{\ESKDcolumnXfVIname}{Зав. Каф.}

\ESKDauthor{Иванов А. Б.}
\ESKDchecker{Мельников А.Б. }
\ESKDnormContr{ Н.Кнтр.~И.О. }
\ESKDapprovedBy{Поляков В.М.}

%%% Для титульника
\ESKDtitleApprovedBy{ Должность утверждающего }{ Фам. утвер. }
\ESKDtitleAgreedBy{ Должность первого согласовавшего }{ Фам. первого согл. }
\ESKDtitleAgreedBy{ Должность второго согласовавшего }{ Фам. второго согл. }
\ESKDtitleAgreedBy{ Должность третьего согласовавшего }{ Фам. третьего согл. }
\ESKDtitleDesignedBy{ Должность первого автора }{ Фам. первого автора }
\ESKDtitleDesignedBy{ Должность второго автора }{ Фам. второго автора }

\ESKDdepartment{ Ведомство }
\ESKDcompany{ Предприятие }
\ESKDclassCode{ Код по классификатору }


\ESKDdate{ 2025/04/21 }




\begin{document}


% 1) ТИТУЛЬНЫЙ ЛИСТ
\begin{titlepage}
    \centering
    {\small \textbf{МИНОБРНАУКИ РОССИИ}}\\
    {\small ФЕДЕРАЛЬНОЕ ГОСУДАРСТВЕННОЕ БЮДЖЕТНОЕ ОБРАЗОВАТЕЛЬНОЕ УЧРЕЖДЕНИЕ}\\
    {\small ВЫСШЕГО ОБРАЗОВАНИЯ}\\
    \textbf{
    «БЕЛГОРОДСКИЙ ГОСУДАРСТВЕННЫЙ ТЕХНОЛОГИЧЕСКИЙ \\
    УНИВЕРСИТЕТ им. В.Г. ШУХОВА» \\
    (БГТУ им. В.Г. Шухова) \\
    }
    
    \vfill % Первый заполнитель
    
    \raggedright
    Институт \textit{информационных технологий и управляющих систем}\\
    Кафедра \textit{программного обеспечения вычислительной техники и автоматизированных систем}\\
    Направление подготовки \textit{09.03.04 – Программная инженерия}\\
    Направленность (профиль) образовательной программы \textit{Разработка программно-информационных систем}
    
    \centering
    \vfill
    
    \textbf{ВЫПУСКНАЯ КВАЛИФИКАЦИОННАЯ РАБОТА}\\
    {\large на тему:\\[1ex]
    «\textbf{Разработка front-end Web – приложения – учебной среды с чатами и AI-анализом кода лабораторных работ}»}
    
    \vfill % Третий заполнитель
    
    \raggedright
    \begin{tabular}{@{} l l @{}}
        \textbf{Студент:}       & Бондаренко Сергей Владимирович \\
        \textbf{Зав. кафедрой}: & канд. техн. наук, доц. Поляков В.М. \\
        \textbf{Руководитель:}  & Мельников А.Б.
    \end{tabular}
    
    \vspace{2cm} 
    
    \centering
    \begin{minipage}{0.7\textwidth}
        \textbf{К защите допустить:\\
        Зав. кафедрой \underline{\hspace{4cm}} /Поляков В.М./\\
        «\underline{\hspace{1cm}}» \underline{\hspace{2cm}} 2024 г.}
    \end{minipage}
    
    \vfill
    
    \textbf{Белгород 2025 г.}
    
\end{titlepage}

\begin{titlepage}
    \centering
    {\small \textbf{МИНОБРНАУКИ РОССИИ}}\\
    {\small ФЕДЕРАЛЬНОЕ ГОСУДАРСТВЕННОЕ БЮДЖЕТНОЕ ОБРАЗОВАТЕЛЬНОЕ УЧРЕЖДЕНИЕ}\\
    {\small ВЫСШЕГО ОБРАЗОВАНИЯ}\\
    \textbf{
    «БЕЛГОРОДСКИЙ ГОСУДАРСТВЕННЫЙ ТЕХНОЛОГИЧЕСКИЙ \\
    УНИВЕРСИТЕТ им. В.Г. ШУХОВА» \\
    (БГТУ им. В.Г. Шухова) \\
    }
    
    \vfill % Первый заполнитель
    
    \raggedright
    Институт \textit{информационных технологий и управляющих систем}\\
    Кафедра \textit{программного обеспечения вычислительной техники и автоматизированных систем}\\
    Направление подготовки \textit{09.03.04 – Программная инженерия}\\
    Направленность (профиль) образовательной программы \textit{Разработка программно-информационных систем}
    
    \centering
    \vfill
    
    \textbf{ВЫПУСКНАЯ КВАЛИФИКАЦИОННАЯ РАБОТА}\\
    {\large на тему:\\[1ex]
    «\textbf{Разработка front-end Web – приложения – учебной среды с чатами и AI-анализом кода лабораторных работ}»}
    
    \vfill % Третий заполнитель
    
    \raggedright
    \begin{tabular}{@{} l l @{}}
        \textbf{Студент:}       & Бондаренко Сергей Владимирович \\
        \textbf{Зав. кафедрой}: & канд. техн. наук, доц. Поляков В.М. \\
        \textbf{Руководитель:}  & Мельников А.Б.
    \end{tabular}
    
    \vfill
    
    \centering
    \begin{minipage}{0.7\textwidth}
        \textbf{К защите допустить:\\
        Зав. кафедрой \underline{\hspace{4cm}} /Поляков В.М./\\
        «\underline{\hspace{1cm}}» \underline{\hspace{2cm}} 2024 г.}
    \end{minipage}
    
    \vfill
    
    \textbf{Белгород 2025 г.}
    
\end{titlepage}

% 2) ЗАДАНИЕ НА ВКР
%\begin{titlepage}
%    \centering
%    {\small МиНобрнауки россии}\\[1ex]
%    {\small федеральное государственное бюджетное образовательное учреждение}\\[1ex]
%    {\small высшего образования}\\[2ex]
%    {\Large «БЕЛГОРОДСКИЙ ГОСУДАРСТВЕННЫЙ ТЕХНОЛОГИЧЕСКИЙ}\\
%    {\Large УНИВЕРСИТЕТ им. В.Г. ШУХОВА»}\\[1ex]
%    {\small (БГТУ им. В.Г. Шухова)}\\[3ex]
%
%    {\normalsize Институт информационных технологий и управляющих систем}\\[1ex]
%    {\normalsize Кафедра программного обеспечения вычислительной техники}\\
%    {\normalsize и автоматизированных систем}\\[1ex]
%    {\normalsize Направление подготовки 09.03.04 – Программная инженерия}\\
%    {\normalsize Профиль: Разработка программно-информационных систем}\\[4ex]
%
%    {\bfseries УТВЕРЖДАЮ:}\\[1ex]
%    Зав. кафедрой \underline{\hspace{5cm}}\\
%    «\underline{\hspace{1cm}}» \underline{\hspace{2cm}} 2024 г.\\[4ex]
%
%    {\bfseries ЗАДАНИЕ}\\[1ex]
%    на выпускную квалификационную работу студента\\[1ex]
%    {\bfseries Азарова Александра Вячеславовича}\\[4ex]
%
%    1. Вид ВКР: бакалаврская работа\\[6ex]
%
%    \begin{flushleft}
%        Дата выдачи задания «\underline{\hspace{1cm}}» \underline{\hspace{3cm}} 2024 г.\\[4ex]
%        \underline{\hspace{6cm}}\\
%        (подпись руководителя)\\[6ex]
%        Задание принял к исполнению:\\[1ex]
%        \underline{\hspace{8cm}}\\
%        (подпись студента)
%    \end{flushleft}
%\end{titlepage}

% 3) КАЛЕНДАРНЫЙ ПЛАН
%\begin{titlepage}
%    \section*{КАЛЕНДАРНЫЙ ПЛАН}
%    \begin{center}
%        \begin{tabular}{|c|p{7cm}|c|p{3cm}|}
%            \hline
%            № п/п & Наименование этапов работы & Срок выполнения & Примечание\\
%            \hline
%            1 & Постановка задачи, изучение аналогов & 20.01.24--18.02.24 & Выполнено\\
%            \hline
%            2 & Исследование методов решения задачи & 19.02.24--25.03.24 & Выполнено\\
%            \hline
%            3 & Проектирование и разработка приложения & 26.03.24--19.04.24 & Выполнено\\
%            \hline
%            4 & Тестирование ЭВМ приложения & 20.04.24--02.05.24 & Выполнено\\
%            \hline
%            5 & Оформление пояснительной записки & 03.05.24--18.06.24 & Выполнено\\
%            \hline
%        \end{tabular}
%    \end{center}
%
%    \vspace{2em}
%    \noindent
%    Студент: \underline{\hspace{6cm}} Азаров Александр Вячеславович\\
%    (Ф.И.О., подпись)\\[2ex]
%    Руководитель: \underline{\hspace{6cm}} Полунин Александр Иванович\\
%    (Ф.И.О., подпись)
%\end{titlepage}

% 4) ФОРМА ПРОВЕРКИ НА ЗАИМСТВОВАНИЕ
%\begin{titlepage}
%    {\Large «Результаты проверки ЭВ ВКР на заимствование»}\\[2ex]
%
%    Кафедра программного обеспечения вычислительной техники и автоматизированных систем\\[3ex]
%
%    \begin{tabular}{lp{8cm}l}
%        Студент: & \underline{\hspace{6cm}} Азаров А.В. & ПВ-201\\
%                & (Фамилия Инициалы) & (Группа)\\
%        Дата: & \underline{\hspace{4cm}} & \\
%    \end{tabular}\\[2ex]
%
%    Тема ВКР: «Разработка алгоритмов распознавания объектов, оснащённых маскировочными средствами»\\[3ex]
%
%    ВКР прошла проверку на объем заимствований.\\[1ex]
%    Итоговая оценка заимствований: \underline{\hspace{4cm}}\\[3ex]
%
%    Работа проверила: \underline{\hspace{5cm}} Яковлева Л.И.\\
%    (Фамилия Инициалы, подпись)\\[3ex]
%
%    Руководитель ВКР: канд. техн. наук, доцент Полунин А.И.\\
%    (Должность, уч. степень, звание, подпись)
%\end{titlepage}



\newpage
\ESKDthisStyle{formII}
\section*{OПРЕДЕЛЕНИЯ, СOКРAЩЕНИЯ И OБOЗНAЧЕНИЯ}


\tableofcontents
\ESKDcolumnII{текст}
\newpage
\ESKDthisStyle{formII}
\section*{Введение}
\addcontentsline{toc}{section}{Введение}


\newpage
\ESKDthisStyle{formII}
\section*{Основная часть}
\addcontentsline{toc}{section}{Основная часть}


\newpage
\ESKDthisStyle{formII}
\section*{Список литературы}
\addcontentsline{toc}{section}{Список литературы}

\newpage

\ESKDthisStyle{formII}
\section*{Приложения}
\addcontentsline{toc}{section}{Приложения}

\end{document}

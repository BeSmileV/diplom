% !TeX program = xelatex

%%% Загружаем заголовочный файл, который хранит все настройки и все
%%% подгружаемые пакеты
\newcommand{\No}{\textnumero}

\PassOptionsToPackage{hidelinks}{hyperref}


%%% Здесь выбираются необходимые графы
\documentclass[russian,utf8,pointsection,nocolumnsxix,nocolumnxxxi,nocolumnxxxii]{eskdtext}
\usepackage{fontspec}
\defaultfontfeatures{Mapping=tex-text}

%%% Чтобы работал eskdx и другие пакеты
\usepackage{xecyr}

%%% Шрифты и кодировки
\usepackage{xunicode,xltxtra}
\usepackage{listings}
\usepackage{longtable}
\usepackage{caption}

%%% Русский текст
\setmainfont{Times New Roman}
\setromanfont{Times New Roman}
\setsansfont{Times New Roman}
\setmonofont{Times New Roman}

%%% Математика
\usepackage{amsmath,amssymb}

%%% Символ градуса
\usepackage{gensymb}

%%% Перенос составных слов
\XeTeXcharclass `\- 24
\XeTeXinterchartoks 24 0 ={\hskip\z@skip}
\XeTeXinterchartoks 0 24 ={\nobreak}

%%% Подпись «Рисунок» вместо «рис. 1»
\addto{\captionsrussian}{\renewcommand{\figurename}{Рисунок}}

%%% Убираем точки после цифр в заголовках
\def\russian@capsformat{%
	\def\postchapter{\@aftersepkern}%
	\def\postsection{\@aftersepkern}%
	\def\postsubsection{\@aftersepkern}%
	\def\postsubsubsection{\@aftersepkern}%
	\def\postparagraph{\@aftersepkern}%
	\def\postsubparagraph{\@aftersepkern}%
}

\sloppy

%%% Графика, ссылки, ToC
\usepackage{graphicx,float,url}
\setcounter{tocdepth}{3}

%%% Гиперссылки
\usepackage{hyperref}

%%% Заголовки разделов
\usepackage{titlesec}
\titleformat{\section}[block]{\normalfont\Large\bfseries}{\thesection}{1em}{}
\titlespacing*{\section}{1cm}{14pt}{8pt}

\titleformat{\subsection}[block]{\normalfont\large\bfseries}{\thesubsection}{1em}{}
\titlespacing*{\subsection}{1cm}{12pt}{6pt}

\titleformat{\subsubsection}[block]{\normalfont\normalsize\bfseries}{\thesubsubsection}{1em}{}
\titlespacing*{\subsubsection}{1cm}{10pt}{4pt}

%%% === Патч нумерации subsubsection ===
\usepackage{etoolbox}               % подтягиваем \preto

% 1) объявляем счётчик
\newcounter{subsubsectcount}

% 2) перед каждым \section: разрыв страницы + сброс нашего счётчика
\preto\section{%
	\clearpage
	\setcounter{subsubsectcount}{0}%
}

% 3) сохраняем оригинальную команду
%\let\origsubsubsection\subsubsection
%
%% 4) переопределяем логику
%\renewcommand{\subsubsection}[1]{%
%	\stepcounter{subsubsectcount}%
%	\ifnum\value{subsubsectcount}>6
%	% после 5-й — звёздочная версия (без номера и без ToC)
%	\origsubsubsection*{#1}%
%	\else
%	% первые 5 — как обычно
%	\origsubsubsection{#1}%
%	\fi
%}
%%% ====================================

%%% 1) Подключаем пакеты (у вас уже есть)
\usepackage{listings}
\usepackage{xcolor}
\usepackage{fontspec}
\setmonofont{JetBrains Mono}

%%% 1) Базовый стиль
\lstdefinestyle{custom}{
	basicstyle=\ttfamily\fontsize{10pt}{12pt}\selectfont,
	frame=single,
	backgroundcolor=\color{gray!10},
	keywordstyle=\bfseries\color{blue!70!black},
	commentstyle=\itshape\color{gray!60},
	stringstyle=\color{red!70!black},
	numbers=left,
	numberstyle=\tiny\color{gray!50},
	xleftmargin=2em,
	xrightmargin=2em,
	showstringspaces=false,
}

%%% 2) Язык TypeScript/TSX
\lstdefinelanguage{TypeScript}{
	sensitive=true,          % регистрозависимый
	breaklines=true,         % перенос длинных строк
	morekeywords={%
		abstract,any,as,asserts,async,await,boolean,break,case,catch,class,const,continue,%
		debugger,declare,default,delete,do,else,enum,export,extends,false,finally,for,%
		from,function,get,if,implements,import,in,instanceof,interface,let,module,namespace,%
		never,new,null,number,of,package,private,protected,public,readonly,require,return,%
		set,static,string,super,switch,symbol,this,throw,true,try,type,typeof,undefined,%
		unique,unknown,var,void,while,with,yield%
	},
	morecomment=[l]{//},
	morecomment=[s]{/*}{*/},
	morestring=[b]",
	morestring=[b]',
	alsoletter={<,>,/},
	moredelim=**[is][\color{blue!50!black}]{<}{>},
	literate={%
		{<}{{$<$}}1
		{>}{{$>$}}1
		{/}{{$/$}}1
	}%
}  % <-- убедитесь, что закрывающая } на месте

%%% 3) Стиль для TS/TSX
\lstdefinestyle{tscustom}{
style=custom,
language=TypeScript,
tabsize=2,
breakatwhitespace=false,
columns=fullflexible,
keepspaces=true,
breakautoindent=true,
breakindent=1em,
postbreak=\mbox{\textcolor{gray}{$\hookrightarrow$}\space},
}

%%% 4) Устанавливаем tscustom по умолчанию
\lstset{style=tscustom}

\usepackage[parentracker=true,
backend=biber,
hyperref=false,
bibencoding=utf8,
style=numeric-comp,
language=auto,
autolang=other,
citestyle=gost-numeric,
defernumbers=true,
bibstyle=gost-numeric,
sorting=none,
]{biblatex}
\addbibresource{bibliography.bib}


\usepackage{subcaption}
\usepackage{booktabs}

\setlength{\parindent}{1cm}

\usepackage{enumitem}
% Для первого уровня списка enumerate
\setlist[enumerate,1]{label=\arabic*), leftmargin=1cm}

% Для второго уровня (если нужно)
\setlist[enumerate,2]{label=\alph*), leftmargin=1cm}

% Для itemize (если нужно настроить)
\setlist[itemize]{leftmargin=1cm}

%%% Настройка основного шрифта и интервала
%\usepackage{setspace} % Добавляем пакет для интервалов
%\onehalfspacing       % Полуторный межстрочный интервал
%\renewcommand{\normalsize}{\fontsize{14pt}{16pt}\selectfont} % 14pt с интервалом 1.5

%%% Абзацный отступ (уже есть, но дублируем для надежности)
\setlength{\parindent}{1cm} 

%%% Настройка отступов для заголовков
\usepackage{titlesec}
% Section: 1cm слева, 14pt сверху, 8pt снизу
\titleformat{\section}[block]{\normalfont\bfseries}{\thesection}{0.5em}{}
\titlespacing*{\section}{1cm}{14pt}{8pt}

% Subsection
\titleformat{\subsection}[block]{\normalfont\bfseries}{\thesubsection}{0.5em}{}
\titlespacing*{\subsection}{1cm}{12pt}{6pt}

% Subsubsection
\titleformat{\subsubsection}[block]{\normalfont\bfseries}{\thesubsubsection}{0.5em}{}
\titlespacing*{\subsubsection}{1cm}{10pt}{4pt}

%%% Убираем заголовок "Содержание" у оглавления
\makeatletter
\renewcommand{\tableofcontents}{%
    \@starttoc{toc}%
}
\makeatother


\newfontfamily\jetbrainsmono{JetBrains Mono}[
  Scale=1.0,
  ItalicFont={JetBrains Mono Italic},
  BoldFont={JetBrains Mono Bold},
  BoldItalicFont={JetBrains Mono Bold Italic},
]

\usepackage{titletoc}
% ----------------------------
% 1) Настройка секций
% ----------------------------
\titlecontents{section}% уровень — section
  [0.5cm]                        % отступ слева
  {}                           % код перед строкой
  {\contentslabel{0.8em}\hspace{0cm}} % нумерация шириной 1.8em + 0.4em после неё
  {}                           % для ненумерованных
  {\titlerule*[0.5em]{.}\contentspage}% dot-линия (0.5em между точками) + номер страницы
  []

% ----------------------------
% 2) (по желанию) subsection/subsubsection
% ----------------------------
\titlecontents{subsection}
  [0.8cm]                      % дополнительный отступ для вложенности
  {}%
  {\contentslabel{1.0em}\hspace{0.5cm}}
  {}%
  {\titlerule*[0.3em]{.}\contentspage}
  []

\titlecontents{subsubsection}
  [1.8cm]
  {}%
  {\contentslabel{2em}\hspace{0.3cm}}
  {}%
  {\titlerule*[0.2em]{.}\contentspage}
  []
  
\usepackage{caption,newfloat}
\DeclareFloatingEnvironment[
  fileext=lol,
  name=Листинг,
  listname=Список\,листингов
]{listing}

% 1) формат «Листинг <номер> – »
\DeclareCaptionLabelFormat{dash}{#1~#2~–~}

% 2) настраиваем подписи именно для окружения listing
\captionsetup[lstlisting]{%
  labelformat=dash,    % наш «–» вместо «:»
  labelsep=none,       % разделитель мы уже вставили в labelformat
  font=normal,          % маленький шрифт, как для рисунков/таблиц
  justification=raggedright, % выравнивание по левому краю
  singlelinecheck=false % даже в одну строку выравнивать по raggedright
}

\usepackage[hidelinks]{hyperref}

%%% Загружаем настройки пакета eskdx, там нужно заполнить информацию
%%% о документе - ФИО авторов, название документов и т.п.
%%% Название документа
\ESKDtitle{ Название документа }
\ESKDdocName{ Название дипломной работы  Название дипломной работы     }

\renewcommand{\ESKDcolumnXfIIname}{Руковод.}
\renewcommand{\ESKDcolumnXfIVname}{Консул.}
\renewcommand{\ESKDcolumnXfVIname}{Зав. Каф.}

\ESKDauthor{Иванов А. Б.}
\ESKDchecker{Мельников А.Б. }
\ESKDnormContr{ Н.Кнтр.~И.О. }
\ESKDapprovedBy{Поляков В.М.}

%%% Для титульника
\ESKDtitleApprovedBy{ Должность утверждающего }{ Фам. утвер. }
\ESKDtitleAgreedBy{ Должность первого согласовавшего }{ Фам. первого согл. }
\ESKDtitleAgreedBy{ Должность второго согласовавшего }{ Фам. второго согл. }
\ESKDtitleAgreedBy{ Должность третьего согласовавшего }{ Фам. третьего согл. }
\ESKDtitleDesignedBy{ Должность первого автора }{ Фам. первого автора }
\ESKDtitleDesignedBy{ Должность второго автора }{ Фам. второго автора }

\ESKDdepartment{ Ведомство }
\ESKDcompany{ Предприятие }
\ESKDclassCode{ Код по классификатору }


\ESKDdate{ 2025/04/21 }




\begin{document}


% 1) ТИТУЛЬНЫЙ ЛИСТ
\begin{titlepage}
    \centering
    {\small \textbf{МИНОБРНАУКИ РОССИИ}}\\
    {\small ФЕДЕРАЛЬНОЕ ГОСУДАРСТВЕННОЕ БЮДЖЕТНОЕ ОБРАЗОВАТЕЛЬНОЕ УЧРЕЖДЕНИЕ}\\
    {\small ВЫСШЕГО ОБРАЗОВАНИЯ}\\
    \textbf{
    «БЕЛГОРОДСКИЙ ГОСУДАРСТВЕННЫЙ ТЕХНОЛОГИЧЕСКИЙ \\
    УНИВЕРСИТЕТ им. В.Г. ШУХОВА» \\
    (БГТУ им. В.Г. Шухова) \\
    }
    
    \vfill % Первый заполнитель
    
    \raggedright
    Институт \textit{информационных технологий и управляющих систем}\\
    Кафедра \textit{программного обеспечения вычислительной техники и автоматизированных систем}\\
    Направление подготовки \textit{09.03.04 – Программная инженерия}\\
    Направленность (профиль) образовательной программы \textit{Разработка программно-информационных систем}
    
    \centering
    \vfill
    
    \textbf{ВЫПУСКНАЯ КВАЛИФИКАЦИОННАЯ РАБОТА}\\
    {\large на тему:\\[1ex]
    «\textbf{Разработка front-end Web – приложения – учебной среды с чатами и AI-анализом кода лабораторных работ}»}
    
    \vfill % Третий заполнитель
    
    \raggedright
    \begin{tabular}{@{} l l @{}}
        \textbf{Студент:}       & Бондаренко Сергей Владимирович \\
        \textbf{Зав. кафедрой}: & канд. техн. наук, доц. Поляков В.М. \\
        \textbf{Руководитель:}  & Мельников А.Б.
    \end{tabular}
    
    \vspace{2cm} 
    
    \centering
    \begin{minipage}{0.7\textwidth}
        \textbf{К защите допустить:\\
        Зав. кафедрой \underline{\hspace{4cm}} /Поляков В.М./\\
        «\underline{\hspace{1cm}}» \underline{\hspace{2cm}} 2024 г.}
    \end{minipage}
    
    \vfill
    
    \textbf{Белгород 2025 г.}
    
\end{titlepage}

\begin{titlepage}
    \centering
    {\small \textbf{МИНОБРНАУКИ РОССИИ}}\\
    {\small ФЕДЕРАЛЬНОЕ ГОСУДАРСТВЕННОЕ БЮДЖЕТНОЕ ОБРАЗОВАТЕЛЬНОЕ УЧРЕЖДЕНИЕ}\\
    {\small ВЫСШЕГО ОБРАЗОВАНИЯ}\\
    \textbf{
    «БЕЛГОРОДСКИЙ ГОСУДАРСТВЕННЫЙ ТЕХНОЛОГИЧЕСКИЙ \\
    УНИВЕРСИТЕТ им. В.Г. ШУХОВА» \\
    (БГТУ им. В.Г. Шухова) \\
    }
    
    \vfill % Первый заполнитель
    
    \raggedright
    Институт \textit{информационных технологий и управляющих систем}\\
    Кафедра \textit{программного обеспечения вычислительной техники и автоматизированных систем}\\
    Направление подготовки \textit{09.03.04 – Программная инженерия}\\
    Направленность (профиль) образовательной программы \textit{Разработка программно-информационных систем}
    
    \centering
    \vfill
    
    \textbf{ВЫПУСКНАЯ КВАЛИФИКАЦИОННАЯ РАБОТА}\\
    {\large на тему:\\[1ex]
    «\textbf{Разработка front-end Web – приложения – учебной среды с чатами и AI-анализом кода лабораторных работ}»}
    
    \vfill % Третий заполнитель
    
    \raggedright
    \begin{tabular}{@{} l l @{}}
        \textbf{Студент:}       & Бондаренко Сергей Владимирович \\
        \textbf{Зав. кафедрой}: & канд. техн. наук, доц. Поляков В.М. \\
        \textbf{Руководитель:}  & Мельников А.Б.
    \end{tabular}
    
    \vfill
    
    \centering
    \begin{minipage}{0.7\textwidth}
        \textbf{К защите допустить:\\
        Зав. кафедрой \underline{\hspace{4cm}} /Поляков В.М./\\
        «\underline{\hspace{1cm}}» \underline{\hspace{2cm}} 2024 г.}
    \end{minipage}
    
    \vfill
    
    \textbf{Белгород 2025 г.}
    
\end{titlepage}

% 2) ЗАДАНИЕ НА ВКР
%\begin{titlepage}
%    \centering
%    {\small МиНобрнауки россии}\\[1ex]
%    {\small федеральное государственное бюджетное образовательное учреждение}\\[1ex]
%    {\small высшего образования}\\[2ex]
%    {\Large «БЕЛГОРОДСКИЙ ГОСУДАРСТВЕННЫЙ ТЕХНОЛОГИЧЕСКИЙ}\\
%    {\Large УНИВЕРСИТЕТ им. В.Г. ШУХОВА»}\\[1ex]
%    {\small (БГТУ им. В.Г. Шухова)}\\[3ex]
%
%    {\normalsize Институт информационных технологий и управляющих систем}\\[1ex]
%    {\normalsize Кафедра программного обеспечения вычислительной техники}\\
%    {\normalsize и автоматизированных систем}\\[1ex]
%    {\normalsize Направление подготовки 09.03.04 – Программная инженерия}\\
%    {\normalsize Профиль: Разработка программно-информационных систем}\\[4ex]
%
%    {\bfseries УТВЕРЖДАЮ:}\\[1ex]
%    Зав. кафедрой \underline{\hspace{5cm}}\\
%    «\underline{\hspace{1cm}}» \underline{\hspace{2cm}} 2024 г.\\[4ex]
%
%    {\bfseries ЗАДАНИЕ}\\[1ex]
%    на выпускную квалификационную работу студента\\[1ex]
%    {\bfseries Азарова Александра Вячеславовича}\\[4ex]
%
%    1. Вид ВКР: бакалаврская работа\\[6ex]
%
%    \begin{flushleft}
%        Дата выдачи задания «\underline{\hspace{1cm}}» \underline{\hspace{3cm}} 2024 г.\\[4ex]
%        \underline{\hspace{6cm}}\\
%        (подпись руководителя)\\[6ex]
%        Задание принял к исполнению:\\[1ex]
%        \underline{\hspace{8cm}}\\
%        (подпись студента)
%    \end{flushleft}
%\end{titlepage}

% 3) КАЛЕНДАРНЫЙ ПЛАН
%\begin{titlepage}
%    \section*{КАЛЕНДАРНЫЙ ПЛАН}
%    \begin{center}
%        \begin{tabular}{|c|p{7cm}|c|p{3cm}|}
%            \hline
%            № п/п & Наименование этапов работы & Срок выполнения & Примечание\\
%            \hline
%            1 & Постановка задачи, изучение аналогов & 20.01.24--18.02.24 & Выполнено\\
%            \hline
%            2 & Исследование методов решения задачи & 19.02.24--25.03.24 & Выполнено\\
%            \hline
%            3 & Проектирование и разработка приложения & 26.03.24--19.04.24 & Выполнено\\
%            \hline
%            4 & Тестирование ЭВМ приложения & 20.04.24--02.05.24 & Выполнено\\
%            \hline
%            5 & Оформление пояснительной записки & 03.05.24--18.06.24 & Выполнено\\
%            \hline
%        \end{tabular}
%    \end{center}
%
%    \vspace{2em}
%    \noindent
%    Студент: \underline{\hspace{6cm}} Азаров Александр Вячеславович\\
%    (Ф.И.О., подпись)\\[2ex]
%    Руководитель: \underline{\hspace{6cm}} Полунин Александр Иванович\\
%    (Ф.И.О., подпись)
%\end{titlepage}

% 4) ФОРМА ПРОВЕРКИ НА ЗАИМСТВОВАНИЕ
%\begin{titlepage}
%    {\Large «Результаты проверки ЭВ ВКР на заимствование»}\\[2ex]
%
%    Кафедра программного обеспечения вычислительной техники и автоматизированных систем\\[3ex]
%
%    \begin{tabular}{lp{8cm}l}
%        Студент: & \underline{\hspace{6cm}} Азаров А.В. & ПВ-201\\
%                & (Фамилия Инициалы) & (Группа)\\
%        Дата: & \underline{\hspace{4cm}} & \\
%    \end{tabular}\\[2ex]
%
%    Тема ВКР: «Разработка алгоритмов распознавания объектов, оснащённых маскировочными средствами»\\[3ex]
%
%    ВКР прошла проверку на объем заимствований.\\[1ex]
%    Итоговая оценка заимствований: \underline{\hspace{4cm}}\\[3ex]
%
%    Работа проверила: \underline{\hspace{5cm}} Яковлева Л.И.\\
%    (Фамилия Инициалы, подпись)\\[3ex]
%
%    Руководитель ВКР: канд. техн. наук, доцент Полунин А.И.\\
%    (Должность, уч. степень, звание, подпись)
%\end{titlepage}



\newpage
\ESKDthisStyle{formII}
\section*{OПРЕДЕЛЕНИЯ, СOКРAЩЕНИЯ И OБOЗНAЧЕНИЯ}


\tableofcontents
\ESKDcolumnII{текст}
\newpage
\ESKDthisStyle{formII}
\section*{Введение}
\addcontentsline{toc}{section}{Введение}


\newpage
\ESKDthisStyle{formII}
\section*{Основная часть}
\addcontentsline{toc}{section}{Основная часть}


\newpage
\ESKDthisStyle{formII}
\section*{Список литературы}
\addcontentsline{toc}{section}{Список литературы}

\newpage

\ESKDthisStyle{formII}
\section*{Приложения}
\addcontentsline{toc}{section}{Приложения}

\end{document}

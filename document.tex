% !TeX program = xelatex

%%% Загружаем заголовочный файл, который хранит все настройки и все
%%% подгружаемые пакеты
\newcommand{\No}{\textnumero}

%%% Здесь выбираются необходимые графы
\documentclass[russian,utf8,pointsection,nocolumnsxix,nocolumnxxxi,nocolumnxxxii]{eskdtext}
\usepackage{fontspec}
\defaultfontfeatures{Mapping=tex-text} % Для того чтобы работали стандартные сочетания символов ---, --, << >> и т.п.

%%% Что бы работал eskdx и некоторые другие пакеты LaTeX
\usepackage{xecyr}

%%% Для работы шрифтов
\usepackage{xunicode,xltxtra}

%%% Для работы с русскими текстами (расстановки переносов, последовательность комманд строго обязательна)
% \usepackage{polyglossia}
% \setdefaultlanguage{russian}
%\newfontfamily{\cyrillicfontt}{GOST_B}
%\set{Times New Roman}
\setmainfont{Times New Roman}
\setromanfont{Times New Roman} 
\setsansfont{Times New Roman} 
\setmonofont{Times New Roman} 

% polyglossia only
% \newfontfamily\cyrillicfont{GOST_type_A} 
% \newfontfamily\cyrillicfontrm{GOST_type_A}
% \newfontfamily\cyrillicfonttt{GOST_type_A}
% \newfontfamily\cyrillicfontsf{GOST_type_A}
%\defaultfontfeatures{Mapping=tex-text}

%%% Для работы со сложными формулами
\usepackage{amsmath}
\usepackage{amssymb}

%%% Что бы использовать символ градуса (°) - \degree
\usepackage{gensymb}


%%% Для переноса составных слов
%\XeTeXinterchartokenstate=1
\XeTeXcharclass `\- 24
\XeTeXinterchartoks 24 0 ={\hskip\z@skip}
\XeTeXinterchartoks 0 24 ={\nobreak}

%%% Ставим подпись к рисункам. Вместо «рис. 1» будет «Рисунок 1»
\addto{\captionsrussian}{\renewcommand{\figurename}{Рисунок}}
%%% Убираем точки после цифр в заголовках
\def\russian@capsformat{%
  \def\postchapter{\@aftersepkern}%
  \def\postsection{\@aftersepkern}%
  \def\postsubsection{\@aftersepkern}%
  \def\postsubsubsection{\@aftersepkern}%
  \def\postparagraph{\@aftersepkern}%
  \def\postsubparagraph{\@aftersepkern}%
}



% Автоматически переносить на след. строку слова которые не убираются
% в строке
\sloppy

%%% Для вставки рисунков
\usepackage{graphicx}

%%% Для вставки интернет ссылок, полезно в библиографии
\usepackage{url}

%%% Подподразделы(\subsubsection) не выводим в содержании
\setcounter{tocdepth}{2}

%%% Каждый раздел (\section) с новой страницы
\let\stdsection\section
\renewcommand\section{\newpage\stdsection}

%%% В введении нумерация подразделов идёт с буквой «В» (например В.1)
\makeatletter
\renewcommand\thesubsection{\ifnum\c@section=0{В.\arabic{subsection}}\else{\arabic{section}.\arabic{subsection}}\fi}
\makeatother

\usepackage{hyperref}

\usepackage{titlesec}

\titleformat{\section}[block]{\normalfont\Large\bfseries}{\thesection}{1em}{}
\titlespacing*{\section}{20pt}{14pt}{8pt} % слева/сверху/снизу

\titleformat{\subsection}[block]{\normalfont\large\bfseries}{\thesubsection}{1em}{}
\titlespacing*{\subsection}{20pt}{12pt}{6pt} % Уменьшаем отступ сверху до 0pt и снизу до 8pt

\titleformat{\subsubsection}[block]{\normalfont\normalsize\bfseries}{\thesubsubsection}{1em}{}
\titlespacing*{\subsubsection}{20pt}{10pt}{4pt} % Уменьшаем отступ сверху до 0pt и снизу до 6pt


%%% Загружаем настройки пакета eskdx, там нужно заполнить информацию
%%% о документе - ФИО авторов, название документов и т.п.
%%% Название документа
\ESKDtitle{ Название документа }
\ESKDdocName{ Название дипломной работы  Название дипломной работы     }

\renewcommand{\ESKDcolumnXfIIname}{Руковод.}
\renewcommand{\ESKDcolumnXfIVname}{Консул.}
\renewcommand{\ESKDcolumnXfVIname}{Зав. Каф.}

\ESKDauthor{Иванов А. Б.}
\ESKDchecker{Мельников А.Б. }
\ESKDnormContr{ Н.Кнтр.~И.О. }
\ESKDapprovedBy{Поляков В.М.}

%%% Для титульника
\ESKDtitleApprovedBy{ Должность утверждающего }{ Фам. утвер. }
\ESKDtitleAgreedBy{ Должность первого согласовавшего }{ Фам. первого согл. }
\ESKDtitleAgreedBy{ Должность второго согласовавшего }{ Фам. второго согл. }
\ESKDtitleAgreedBy{ Должность третьего согласовавшего }{ Фам. третьего согл. }
\ESKDtitleDesignedBy{ Должность первого автора }{ Фам. первого автора }
\ESKDtitleDesignedBy{ Должность второго автора }{ Фам. второго автора }

\ESKDdepartment{ Ведомство }
\ESKDcompany{ Предприятие }
\ESKDclassCode{ Код по классификатору }


\ESKDdate{ 2025/04/21 }




\begin{document}


% 1) ТИТУЛЬНЫЙ ЛИСТ
\begin{titlepage}
    \centering
    {\small \textbf{МИНОБРНАУКИ РОССИИ}}\\
    {\small ФЕДЕРАЛЬНОЕ ГОСУДАРСТВЕННОЕ БЮДЖЕТНОЕ ОБРАЗОВАТЕЛЬНОЕ УЧРЕЖДЕНИЕ}\\
    {\small ВЫСШЕГО ОБРАЗОВАНИЯ}\\
    \textbf{
    «БЕЛГОРОДСКИЙ ГОСУДАРСТВЕННЫЙ ТЕХНОЛОГИЧЕСКИЙ \\
    УНИВЕРСИТЕТ им. В.Г. ШУХОВА» \\
    (БГТУ им. В.Г. Шухова) \\
    }
    
    \vfill % Первый заполнитель
    
    \raggedright
    Институт \textit{информационных технологий и управляющих систем}\\
    Кафедра \textit{программного обеспечения вычислительной техники и автоматизированных систем}\\
    Направление подготовки \textit{09.03.04 – Программная инженерия}\\
    Направленность (профиль) образовательной программы \textit{Разработка программно-информационных систем}
    
    \centering
    \vfill
    
    \textbf{ВЫПУСКНАЯ КВАЛИФИКАЦИОННАЯ РАБОТА}\\
    {\large на тему:\\[1ex]
    «\textbf{Разработка front-end Web – приложения – учебной среды с чатами и AI-анализом кода лабораторных работ}»}
    
    \vfill % Третий заполнитель
    
    \raggedright
    \begin{tabular}{@{} l l @{}}
        \textbf{Студент:}       & Бондаренко Сергей Владимирович \\
        \textbf{Зав. кафедрой}: & канд. техн. наук, доц. Поляков В.М. \\
        \textbf{Руководитель:}  & Мельников А.Б.
    \end{tabular}
    
    \vspace{2cm} 
    
    \centering
    \begin{minipage}{0.7\textwidth}
        \textbf{К защите допустить:\\
        Зав. кафедрой \underline{\hspace{4cm}} /Поляков В.М./\\
        «\underline{\hspace{1cm}}» \underline{\hspace{2cm}} 2024 г.}
    \end{minipage}
    
    \vfill
    
    \textbf{Белгород 2025 г.}
    
\end{titlepage}

\begin{titlepage}
    \centering
    {\small \textbf{МИНОБРНАУКИ РОССИИ}}\\
    {\small ФЕДЕРАЛЬНОЕ ГОСУДАРСТВЕННОЕ БЮДЖЕТНОЕ ОБРАЗОВАТЕЛЬНОЕ УЧРЕЖДЕНИЕ}\\
    {\small ВЫСШЕГО ОБРАЗОВАНИЯ}\\
    \textbf{
    «БЕЛГОРОДСКИЙ ГОСУДАРСТВЕННЫЙ ТЕХНОЛОГИЧЕСКИЙ \\
    УНИВЕРСИТЕТ им. В.Г. ШУХОВА» \\
    (БГТУ им. В.Г. Шухова) \\
    }
    
    \vfill % Первый заполнитель
    
    \raggedright
    Институт \textit{информационных технологий и управляющих систем}\\
    Кафедра \textit{программного обеспечения вычислительной техники и автоматизированных систем}\\
    Направление подготовки \textit{09.03.04 – Программная инженерия}\\
    Направленность (профиль) образовательной программы \textit{Разработка программно-информационных систем}
    
    \centering
    \vfill
    
    \textbf{ВЫПУСКНАЯ КВАЛИФИКАЦИОННАЯ РАБОТА}\\
    {\large на тему:\\[1ex]
    «\textbf{Разработка front-end Web – приложения – учебной среды с чатами и AI-анализом кода лабораторных работ}»}
    
    \vfill % Третий заполнитель
    
    \raggedright
    \begin{tabular}{@{} l l @{}}
        \textbf{Студент:}       & Бондаренко Сергей Владимирович \\
        \textbf{Зав. кафедрой}: & канд. техн. наук, доц. Поляков В.М. \\
        \textbf{Руководитель:}  & Мельников А.Б.
    \end{tabular}
    
    \vfill
    
    \centering
    \begin{minipage}{0.7\textwidth}
        \textbf{К защите допустить:\\
        Зав. кафедрой \underline{\hspace{4cm}} /Поляков В.М./\\
        «\underline{\hspace{1cm}}» \underline{\hspace{2cm}} 2024 г.}
    \end{minipage}
    
    \vfill
    
    \textbf{Белгород 2025 г.}
    
\end{titlepage}

\newpage
\ESKDthisStyle{formII}
\section*{OПРЕДЕЛЕНИЯ, СOКРAЩЕНИЯ И OБOЗНAЧЕНИЯ}

ИИ - искуственный интелект.

\tableofcontents
\ESKDcolumnII{текст}
\newpage
\ESKDthisStyle{formII}
\section*{Введение}
\addcontentsline{toc}{section}{Введение}

Развитие цифровых технологий в сфере образования значительно меняет способы взаимодействия между преподавателями и студентами, предоставляя новые возможности для обучения и обмена информацией. В условиях дистанционного и смешанного обучения особенно важной становится необходимость создания платформ, которые бы объединяли образовательные инструменты в едином пространстве. Веб-приложения, которые решают задачи взаимодействия, позволяют сократить барьеры между преподавателями и студентами, улучшить коммуникацию и повысить качество образования. Цифровая среда должна обеспечивать не только размещение учебных материалов и заданий, но и средства для общения, автоматической оценки и анализа решений с использованием современных технологий, включая искусственный интеллект.

\textbf{Актуальность} темы заключается в потребности создания интегрированной образовательной платформы, которая объединяет функции чатов, проведения занятий и автоматического анализа решений, используя возможности ИИ. На данный момент отсутствует единая система, которая бы эффективно сочетала в себе эти ключевые аспекты: возможность общения через чаты, создание заданий и автоматизированную проверку решений с помощью ИИ. Современные платформы, как правило, фрагментированы — отдельные системы для чатов, другие для размещения заданий, третьи для автоматической проверки кода, что значительно усложняет организацию учебного процесса и снижает его эффективность. Разработка интегрированного решения, которое объединило бы эти элементы, позволяет улучшить качество образовательного процесса, повысив продуктивность студентов и преподавателей, а также упростив взаимодействие и автоматизировав многие рутинные задачи.

\textbf{Целью} данной работы является разработка клиентской части образовательной платформы, которая будет включать функции взаимодействия между преподавателями и студентами, автоматизированную проверку кода, а также возможности общения в рамках чатов. Особое внимание уделяется созданию такого интерфейса, который позволит преподавателям и студентам взаимодействовать в едином пространстве, где будут доступны все образовательные инструменты и ресурсы.

\textbf{Для достижения поставленной цели необходимо решить следующие задачи:}
\begin{enumerate}
\item Проанализировать предметную область и существующие системы, выявив их сильные и слабые стороны.
\item Определить архитектурные и технологические решения, подходящие для реализации клиентской части платформы.
\item Спроектировать пользовательский интерфейс, обеспечивающий интуитивное и удобное взаимодействие для преподавателей и студентов.
\item Разработать компоненты для управления учебными структурами (институт, кафедра, группа), заданиями и чатами.
\item Интегрировать средства для автоматизированной проверки решений студентов с применением ИИ.
\item Реализовать тестирование бизнес-логики приложения для обеспечения её корректности и эффективности.
\end{enumerate}

\textbf{Структура пояснительной записки} включает следующие разделы:
\begin{itemize}
\item В первом разделе рассматриваются особенности предметной области, проводится анализ существующих решений и обоснование выбора технологий и методов проектирования. Приводится обзор существующих образовательных платформ и их недостатков, а также объясняется необходимость разработки интегрированного решения.
\item Во втором разделе описывается архитектура клиентской части приложения, структура пользовательского интерфейса, проектирование компонентов и их взаимодействие. Рассматриваются решения для реализации системы чатов, создания и проверки заданий, а также интеграции ИИ-анализа.
\item В третьем разделе приводится описание реализации: структура кода, используемые технологии (Next.js, React, TypeScript, Redux, Auth.js), описание экрана и взаимодействий, примеры реализации различных компонентов системы.
\item В заключении приводятся выводы по выполненной работе, оценивается эффективность разработанного интерфейса и функционала, а также определяются направления для дальнейшего развития и улучшения системы. Указываются перспективы внедрения ИИ в образовательные платформы для улучшения процессов оценки и взаимодействия.
\end{itemize}

\newpage
\ESKDthisStyle{formII}
\section*{Основная часть}
\addcontentsline{toc}{section}{Основная часть}


\newpage
\ESKDthisStyle{formII}
\section*{Список литературы}
\addcontentsline{toc}{section}{Список литературы}

\newpage

\ESKDthisStyle{formII}
\section*{Приложения}
\addcontentsline{toc}{section}{Приложения}

\end{document}

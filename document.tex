% !TeX program = xelatex

%%% Загружаем заголовочный файл, который хранит все настройки и все
%%% подгружаемые пакеты
\newcommand{\No}{\textnumero}

%%% Здесь выбираются необходимые графы
\documentclass[russian,utf8,pointsection,nocolumnsxix,nocolumnxxxi,nocolumnxxxii]{eskdtext}
\usepackage{fontspec}
\defaultfontfeatures{Mapping=tex-text} % Для того чтобы работали стандартные сочетания символов ---, --, << >> и т.п.

%%% Что бы работал eskdx и некоторые другие пакеты LaTeX
\usepackage{xecyr}

%%% Для работы шрифтов
\usepackage{xunicode,xltxtra}

%%% Для работы с русскими текстами (расстановки переносов, последовательность комманд строго обязательна)
% \usepackage{polyglossia}
% \setdefaultlanguage{russian}
%\newfontfamily{\cyrillicfontt}{GOST_B}
%\set{Times New Roman}
\setmainfont{Times New Roman}
\setromanfont{Times New Roman} 
\setsansfont{Times New Roman} 
\setmonofont{Times New Roman} 

% polyglossia only
% \newfontfamily\cyrillicfont{GOST_type_A} 
% \newfontfamily\cyrillicfontrm{GOST_type_A}
% \newfontfamily\cyrillicfonttt{GOST_type_A}
% \newfontfamily\cyrillicfontsf{GOST_type_A}
%\defaultfontfeatures{Mapping=tex-text}

%%% Для работы со сложными формулами
\usepackage{amsmath}
\usepackage{amssymb}

%%% Что бы использовать символ градуса (°) - \degree
\usepackage{gensymb}


%%% Для переноса составных слов
%\XeTeXinterchartokenstate=1
\XeTeXcharclass `\- 24
\XeTeXinterchartoks 24 0 ={\hskip\z@skip}
\XeTeXinterchartoks 0 24 ={\nobreak}

%%% Ставим подпись к рисункам. Вместо «рис. 1» будет «Рисунок 1»
\addto{\captionsrussian}{\renewcommand{\figurename}{Рисунок}}
%%% Убираем точки после цифр в заголовках
\def\russian@capsformat{%
  \def\postchapter{\@aftersepkern}%
  \def\postsection{\@aftersepkern}%
  \def\postsubsection{\@aftersepkern}%
  \def\postsubsubsection{\@aftersepkern}%
  \def\postparagraph{\@aftersepkern}%
  \def\postsubparagraph{\@aftersepkern}%
}



% Автоматически переносить на след. строку слова которые не убираются
% в строке
\sloppy

%%% Для вставки рисунков
\usepackage{graphicx}

%%% Для вставки интернет ссылок, полезно в библиографии
\usepackage{url}

%%% Подподразделы(\subsubsection) не выводим в содержании
\setcounter{tocdepth}{2}

%%% Каждый раздел (\section) с новой страницы
\let\stdsection\section
\renewcommand\section{\newpage\stdsection}

%%% В введении нумерация подразделов идёт с буквой «В» (например В.1)
\makeatletter
\renewcommand\thesubsection{\ifnum\c@section=0{В.\arabic{subsection}}\else{\arabic{section}.\arabic{subsection}}\fi}
\makeatother

\usepackage{hyperref}

\usepackage{titlesec}

\titleformat{\section}[block]{\normalfont\Large\bfseries}{\thesection}{1em}{}
\titlespacing*{\section}{20pt}{14pt}{8pt} % слева/сверху/снизу

\titleformat{\subsection}[block]{\normalfont\large\bfseries}{\thesubsection}{1em}{}
\titlespacing*{\subsection}{20pt}{12pt}{6pt} % Уменьшаем отступ сверху до 0pt и снизу до 8pt

\titleformat{\subsubsection}[block]{\normalfont\normalsize\bfseries}{\thesubsubsection}{1em}{}
\titlespacing*{\subsubsection}{20pt}{10pt}{4pt} % Уменьшаем отступ сверху до 0pt и снизу до 6pt


%%% Загружаем настройки пакета eskdx, там нужно заполнить информацию
%%% о документе - ФИО авторов, название документов и т.п.
%%% Название документа
\ESKDtitle{ Название документа }
\ESKDdocName{ Название дипломной работы  Название дипломной работы     }

\renewcommand{\ESKDcolumnXfIIname}{Руковод.}
\renewcommand{\ESKDcolumnXfIVname}{Консул.}
\renewcommand{\ESKDcolumnXfVIname}{Зав. Каф.}

\ESKDauthor{Иванов А. Б.}
\ESKDchecker{Мельников А.Б. }
\ESKDnormContr{ Н.Кнтр.~И.О. }
\ESKDapprovedBy{Поляков В.М.}

%%% Для титульника
\ESKDtitleApprovedBy{ Должность утверждающего }{ Фам. утвер. }
\ESKDtitleAgreedBy{ Должность первого согласовавшего }{ Фам. первого согл. }
\ESKDtitleAgreedBy{ Должность второго согласовавшего }{ Фам. второго согл. }
\ESKDtitleAgreedBy{ Должность третьего согласовавшего }{ Фам. третьего согл. }
\ESKDtitleDesignedBy{ Должность первого автора }{ Фам. первого автора }
\ESKDtitleDesignedBy{ Должность второго автора }{ Фам. второго автора }

\ESKDdepartment{ Ведомство }
\ESKDcompany{ Предприятие }
\ESKDclassCode{ Код по классификатору }


\ESKDdate{ 2025/04/21 }



\begin{document}


% 1) ТИТУЛЬНЫЙ ЛИСТ
\begin{titlepage}
    \centering
    {\small \textbf{МИНОБРНАУКИ РОССИИ}}\\
    {\small ФЕДЕРАЛЬНОЕ ГОСУДАРСТВЕННОЕ БЮДЖЕТНОЕ ОБРАЗОВАТЕЛЬНОЕ УЧРЕЖДЕНИЕ}\\
    {\small ВЫСШЕГО ОБРАЗОВАНИЯ}\\
    \textbf{
    «БЕЛГОРОДСКИЙ ГОСУДАРСТВЕННЫЙ ТЕХНОЛОГИЧЕСКИЙ \\
    УНИВЕРСИТЕТ им. В.Г. ШУХОВА» \\
    (БГТУ им. В.Г. Шухова) \\
    }
    
    \vfill % Первый заполнитель
    
    \raggedright
    Институт \textit{информационных технологий и управляющих систем}\\
    Кафедра \textit{программного обеспечения вычислительной техники и автоматизированных систем}\\
    Направление подготовки \textit{09.03.04 – Программная инженерия}\\
    Направленность (профиль) образовательной программы \textit{Разработка программно-информационных систем}
    
    \centering
    \vfill
    
    \textbf{ВЫПУСКНАЯ КВАЛИФИКАЦИОННАЯ РАБОТА}\\
    {\large на тему:\\[1ex]
    «\textbf{Разработка front-end Web – приложения – учебной среды с чатами и AI-анализом кода лабораторных работ}»}
    
    \vfill % Третий заполнитель
    
    \raggedright
    \begin{tabular}{@{} l l @{}}
        \textbf{Студент:}       & Бондаренко Сергей Владимирович \\
        \textbf{Зав. кафедрой}: & канд. техн. наук, доц. Поляков В.М. \\
        \textbf{Руководитель:}  & Мельников А.Б.
    \end{tabular}
    
    \vspace{2cm} 
    
    \centering
    \begin{minipage}{0.7\textwidth}
        \textbf{К защите допустить:\\
        Зав. кафедрой \underline{\hspace{4cm}} /Поляков В.М./\\
        «\underline{\hspace{1cm}}» \underline{\hspace{2cm}} 2025 г.}
    \end{minipage}
    
    \vfill
    
    \textbf{Белгород 2025 г.}
    
\end{titlepage}

\begin{titlepage}
    \centering
    {\small \textbf{МИНОБРНАУКИ РОССИИ}}\\
    {\small ФЕДЕРАЛЬНОЕ ГОСУДАРСТВЕННОЕ БЮДЖЕТНОЕ ОБРАЗОВАТЕЛЬНОЕ УЧРЕЖДЕНИЕ}\\
    {\small ВЫСШЕГО ОБРАЗОВАНИЯ}\\
    \textbf{
    «БЕЛГОРОДСКИЙ ГОСУДАРСТВЕННЫЙ ТЕХНОЛОГИЧЕСКИЙ \\
    УНИВЕРСИТЕТ им. В.Г. ШУХОВА» \\
    (БГТУ им. В.Г. Шухова) \\
    }
    
    \vfill % Первый заполнитель
    
    \raggedright
    Институт \textit{информационных технологий и управляющих систем}\\
    Кафедра \textit{программного обеспечения вычислительной техники и автоматизированных систем}\\
    Направление подготовки \textit{09.03.04 – Программная инженерия}\\
    Направленность (профиль) образовательной программы \textit{Разработка программно-информационных систем}
    
    \centering
    \vfill
    
    \textbf{ВЫПУСКНАЯ КВАЛИФИКАЦИОННАЯ РАБОТА}\\
    {\large на тему:\\[1ex]
    «\textbf{Разработка front-end Web – приложения – учебной среды с чатами и AI-анализом кода лабораторных работ}»}
    
    \vfill % Третий заполнитель
    
    \raggedright
    \begin{tabular}{@{} l l @{}}
        \textbf{Студент:}       & Бондаренко Сергей Владимирович \\
        \textbf{Зав. кафедрой}: & канд. техн. наук, доц. Поляков В.М. \\
        \textbf{Руководитель:}  & Мельников А.Б.
    \end{tabular}
    
    \vfill
    
    \centering
    \begin{minipage}{0.7\textwidth}
        \textbf{К защите допустить:\\
        Зав. кафедрой \underline{\hspace{4cm}} /Поляков В.М./\\
        «\underline{\hspace{1cm}}» \underline{\hspace{2cm}} 2025 г.}
    \end{minipage}
    
    \vfill
    
    \textbf{Белгород 2025 г.}
    
\end{titlepage}

\newpage
\ESKDthisStyle{formII}
\ESKDcolumnII{OПРЕДЕЛЕНИЯ, СOКРAЩЕНИЯ И OБOЗНAЧЕНИ}
\section*{OПРЕДЕЛЕНИЯ, СOКРAЩЕНИЯ И OБOЗНAЧЕНИЯ}
\addcontentsline{toc}{section}{OПРЕДЕЛЕНИЯ, СOКРAЩЕНИЯ И OБOЗНAЧЕНИЯ}

\begin{itemize}
  \item ИИ (AI) — искусственный интеллект.
  \item SPA (Single Page Application) — одностраничное веб-приложение, в котором навигация осуществляется без перезагрузки страниц.
  \item CSR (Client-Side Rendering) — отрисовка страницы на стороне клиента после загрузки минимального каркаса с сервера.
  \item SSR (Server-Side Rendering) — рендеринг веб-страниц на стороне сервера перед отправкой клиенту.
  \item SSG (Static Site Generation) — генерация статических HTML-страниц на этапе сборки приложения.
  \item JWT (JSON Web Token) — формат токена для безопасной передачи информации между сторонами в виде JSON-объекта.
  \item FSD (Feature-Sliced Design) — архитектурный подход к построению фронтенд-приложений, основанный на разделении кода по смысловым «срезам» (feature, widget, entity и др.).
  \item API (Application Programming Interface) — программный интерфейс, обеспечивающий взаимодействие между компонентами программного обеспечения.
  \item UI (User Interface) — пользовательский интерфейс.
  \item DOM (Document Object Model) — объектная модель документа, представляющая HTML-структуру страницы в виде дерева объектов.
  \item Virtual DOM — виртуальная копия DOM, используемая React для эффективного выявления и обновления изменившихся элементов.
  \item JSX — синтаксис, объединяющий JavaScript и XML-подобную разметку, применяемый в React.
  \item React — библиотека JavaScript для построения пользовательских интерфейсов на основе компонентного подхода.
  \item Next.js — фреймворк на базе React с поддержкой CSR, SSR и SSG.
  \item TypeScript — надмножество JavaScript, добавляющее статическую типизацию и расширяющее возможности языка.
  \item Node.js — серверная среда выполнения JavaScript, используемая для построения бэкенд-сервисов и выполнения скриптов на сервере.
  \item Socket.IO — библиотека для организации двусторонних WebSocket-соединений между клиентом и сервером с автоматическим фоллбеком.
  \item WebSocket — протокол для установления постоянного соединения между клиентом и сервером и обмена данными в режиме реального времени.
  \item OAuth — протокол авторизации, позволяющий сторонним приложениям получать доступ к ресурсам через выданные токены.
  \item CSRF (Cross-Site Request Forgery) — тип атаки, при которой злоумышленник вынуждает браузер пользователя выполнить нежелательный запрос от его имени; в веб-приложениях применяется защита с помощью специальных токенов.
  \item XSS (Cross-Site Scripting) — уязвимость, позволяющая внедрять скрипты стороннего происхождения на страницы; одной из мер защиты является хранение JWT в HTTP-only cookie.
  \item SSO (Single Sign-On) — единый вход, позволяющий пользователю аутентифицироваться один раз и использовать доступ ко множеству сервисов без повторной авторизации.
  \item Auth.js — библиотека для реализации аутентификации и авторизации в приложениях на Next.js (ранее известная как NextAuth.js).
  \item Redux — библиотека для централизованного управления состоянием приложений, основанная на едином хранилище и концепции «actions» и «reducers».
  \item Redux Toolkit — официальная утилита для упрощённой работы с Redux, предоставляющая набор шаблонных функций и удобных API для создания «слайсов» и сторов.
  \item Zustand — лёгкая альтернатива Redux для управления состоянием в React-приложениях с минимальным «боли молем».
  \item HTTP (Hypertext Transfer Protocol) — протокол передачи гипертекстовых данных между клиентом и сервером; используется для REST-запросов.
  \item REST (Representational State Transfer) — архитектурный стиль взаимодействия с API через HTTP-методы (GET, POST и др.).
  \item JSON (JavaScript Object Notation) — текстовый формат обмена данными, широко используемый в REST-API и для передачи полезной нагрузки токенов.
  \item HTML (HyperText Markup Language) — язык разметки веб-страниц, создающий структуру документа.
  \item CSS (Cascading Style Sheets) — язык каскадных таблиц стилей для описания внешнего вида HTML-элементов.
  \item SCSS (Sassy CSS) — расширение CSS с поддержкой переменных, вложенности и других возможностей препроцессора.
  \item CRUD (Create, Read, Update, Delete) — базовые операции над данными, которые выполняются при создании, чтении, обновлении и удалении записей.
  \item OAuth2.0 — современная версия протокола OAuth с расширенными возможностями выдачи и отзыва токенов (часто используется в связке с Auth.js для входа через сторонние сервисы).
  \item MVC (Model-View-Controller) — архитектурный паттерн разделения приложения на модель, представление и контроллер; многие упомянутые фреймворки опираются на похожие принципы.
  \item CLI (Command Line Interface) — интерфейс командной строки, используемый для генерации проекта и выполнения скриптов (например, Angular CLI или Next.js CLI).
  \item MVP (Minimum Viable Product) — минимально жизнеспособный продукт, концепция, описывающая начальную версию продукта с базовым функционалом.
  \item OAuth2 — использован для интеграции входа через аккаунты сторонних сервисов, таких как Google, Facebook и др.
  \item HTTP-only cookie — тип cookie, недоступный через JavaScript, используется для безопасного хранения JWT и защиты от XSS.
  \item DeepSeek — облачный AI-сервис для анализа кода лабораторных работ и выдачи рекомендаций по исправлению ошибок.
  \item FSD-срез (feature slice) — логически обособленный фрагмент архитектуры Feature-Sliced Design, который содержит все уровни: UI, бизнес-логику и взаимодействие с API по одной функции.
  \item Widget — компонент верхнего уровня в архитектуре FSD, объединяющий в себе несколько «фич» или отдельных элементов пользовательского интерфейса.
  \item Feature — отдельный модуль бизнес-логики в архитектуре FSD, отвечающий за конкретную функциональность приложения.
  \item Entity — модель предметной области, слой взаимодействия с API и определения типов данных в архитектуре FSD.
  \item Shared — общий слой в архитектуре FSD, содержащий утилиты, константы, типы и переиспользуемые компоненты.
  \item CRUD-операции — базовые действия (создание, чтение, обновление, удаление) с ресурсо-ориентированными данными на клиенте и сервере.
  \item UUID (Universally Unique Identifier) — универсальный уникальный идентификатор, используемый для генерации временных или локальных идентификаторов (например, при оптимистичной отправке сообщений).
  \item Optimistic UI — техника обновления пользовательского интерфейса до получения подтверждения от сервера, чтобы пользователь видел мгновенный отклик.
\end{itemize}

\newpage
\ESKDthisStyle{formII}
\tableofcontents

\newpage
\ESKDthisStyle{formII}
\section*{Введение}
\addcontentsline{toc}{section}{Введение}
\ESKDcolumnII{Введение}

Развитие цифровых технологий в сфере образования значительно меняет способы взаимодействия между преподавателями и студентами, предоставляя новые возможности для обучения и обмена информацией. В условиях дистанционного и смешанного обучения особенно важной становится необходимость создания платформ, которые бы объединяли образовательные инструменты в едином пространстве. Веб-приложения, которые решают задачи взаимодействия, позволяют сократить барьеры между преподавателями и студентами, улучшить коммуникацию и повысить качество образования. Цифровая среда должна обеспечивать не только размещение учебных материалов и заданий, но и средства для общения, автоматической оценки и анализа решений с использованием современных технологий, включая искусственный интеллект.

\textbf{Актуальность} темы заключается в потребности создания интегрированной образовательной платформы, которая объединяет функции чатов, проведения занятий и автоматического анализа решений, используя возможности ИИ. На данный момент отсутствует единая система, которая бы эффективно сочетала в себе эти ключевые аспекты: возможность общения через чаты, создание заданий и автоматизированную проверку решений с помощью ИИ. Современные платформы, как правило, фрагментированы — отдельные системы для чатов, другие для размещения заданий, третьи для автоматической проверки кода, что значительно усложняет организацию учебного процесса и снижает его эффективность. Разработка интегрированного решения, которое объединило бы эти элементы, позволяет улучшить качество образовательного процесса, повысив продуктивность студентов и преподавателей, а также упростив взаимодействие и автоматизировав многие рутинные задачи.

\textbf{Целью} данной работы является разработка клиентской части образовательной платформы, которая будет включать функции взаимодействия между преподавателями и студентами, автоматизированную проверку кода, а также возможности общения в рамках чатов. Особое внимание уделяется созданию такого интерфейса, который позволит преподавателям и студентам взаимодействовать в едином пространстве, где будут доступны все образовательные инструменты и ресурсы.

\textbf{Для достижения поставленной цели необходимо решить следующие задачи:}
\begin{enumerate}
\item Проанализировать предметную область и существующие системы, выявив их сильные и слабые стороны.
\item Определить архитектурные и технологические решения, подходящие для реализации клиентской части платформы.
\item Спроектировать пользовательский интерфейс, обеспечивающий интуитивное и удобное взаимодействие для преподавателей и студентов.
\item Разработать компоненты для управления учебными структурами (институт, кафедра, группа), заданиями и чатами.
\item Интегрировать средства для автоматизированной проверки решений студентов с применением ИИ.
\item Реализовать тестирование бизнес-логики приложения для обеспечения её корректности и эффективности.
\end{enumerate}

\textbf{Структура пояснительной записки} включает следующие разделы:
\begin{itemize}
\item В первом разделе рассматриваются особенности предметной области, проводится анализ существующих решений и обоснование выбора технологий и методов проектирования. Приводится обзор существующих образовательных платформ и их недостатков, а также объясняется необходимость разработки интегрированного решения.
\item Во втором разделе описывается архитектура клиентской части приложения, структура пользовательского интерфейса, проектирование компонентов и их взаимодействие. Рассматриваются решения для реализации системы чатов, создания и проверки заданий, а также интеграции ИИ-анализа.
\item В третьем разделе приводится описание реализации: структура кода, используемые технологии (Next.js, React, TypeScript, Redux, Auth.js), описание экрана и взаимодействий, примеры реализации различных компонентов системы.
\item В заключении приводятся выводы по выполненной работе, оценивается эффективность разработанного интерфейса и функционала, а также определяются направления для дальнейшего развития и улучшения системы. Указываются перспективы внедрения ИИ в образовательные платформы для улучшения процессов оценки и взаимодействия.
\end{itemize}

\newpage
\ESKDthisStyle{formII}
\section*{ОБЗОР И ИССЛЕДОВАНИЕ ПРЕДМЕТНОЙ ОБЛАСТИ}
\addcontentsline{toc}{section}{ОБЗОР И ИССЛЕДОВАНИЕ ПРЕДМЕТНОЙ ОБЛАСТИ}

\subsection*{Введение в предметную область}
\addcontentsline{toc}{subsection}{Введение в предметную область}

Современные образовательные процессы переживают значительные изменения под воздействием цифровых технологий. В условиях быстрого роста объемов информации и перехода на дистанционное и смешанное обучение возникает потребность в создании платформ, которые объединяют различные образовательные инструменты в единую систему. Проблемы, с которыми сталкиваются преподаватели и студенты, включают фрагментацию существующих решений: чаты для общения, отдельные системы для размещения и проверки заданий, а также инструменты для анализа решений студентов.

Существующие платформы не всегда обеспечивают интеграцию всех этих функций в одном приложении, что приводит к необходимости использования множества разных сервисов для выполнения учебных задач. В рамках образовательных процессов это усложняет взаимодействие между преподавателями и студентами, увеличивает время на организацию обучения и снижает его эффективность.

Одной из важнейших задач является создание платформы, которая объединяет все эти компоненты в одном месте, обеспечивая удобный интерфейс для студентов и преподавателей. Такая система должна включать:
\begin{itemize}
  \item возможность создания и размещения учебных заданий;
  \item автоматическое тестирование решений студентов с использованием ИИ для проверки правильности кода;
  \item чат-функциональность для общения студентов с преподавателями и внутри групп;
  \item централизованный доступ к учебным материалам.
\end{itemize}

Интеграция всех этих функций в одну платформу позволит значительно упростить организацию учебного процесса, улучшить взаимодействие между преподавателями и студентами, а также повысить качество обучения за счет автоматизации рутинных задач.
\subsection{Анализ существующих образовательных платформ}

Современные образовательные платформы, такие как Moodle, Google Classroom, Microsoft Teams для образования, а также специализированные решения, предназначенные для работы с программированием, предлагают различные функциональные возможности для взаимодействия преподавателей и студентов. Однако каждая из этих платформ имеет свои ограничения и не всегда покрывает все потребности в рамках единой системы.

\subsubsection{Moodle}
Moodle является одной из самых популярных образовательных платформ, используемых во многих учебных заведениях. Она предоставляет инструменты для размещения учебных материалов, организации тестов и заданий, а также ведения онлайн-курсов. Однако, несмотря на свои возможности, Moodle не предоставляет встроенных решений для автоматической проверки кода студентов, а также не включает в себя продвинутые механизмы общения в реальном времени, что делает её менее эффективной для динамичного взаимодействия в процессе обучения.

\subsubsection{Google Classroom}
Google Classroom предлагает простоту в использовании и позволяет интегрировать различные Google сервисы. Платформа позволяет преподавателям создавать задания, прикреплять материалы и отслеживать выполнение студентами. Однако Google Classroom не предоставляет функциональности для автоматического анализа решений, особенно в контексте программирования. Это требует интеграции с внешними инструментами, что усложняет использование системы в образовательных учреждениях.

\subsubsection{Microsoft Teams for Education}
Microsoft Teams, в отличие от Moodle и Google Classroom, активно используется для организации видеоконференций и групповых чатов. Он позволяет преподавателям и студентам взаимодействовать в реальном времени, а также интегрирует различные сервисы Microsoft 365. Однако, как и в случае с другими платформами, Microsoft Teams не предоставляет функционала для интегрированного анализа кода студентов с использованием искусственного интеллекта, что ограничивает его возможности в обучении программированию.

\subsubsection{Платформы для анализа кода}
Существуют специализированные платформы, такие как CodeSignal, Codility, LeetCode, которые позволяют преподавателям и работодателям тестировать навыки программирования студентов. Эти системы используют алгоритмы для автоматической проверки решений, однако они ограничены в функционале взаимодействия с преподавателями и студентами, а также не обеспечивают централизованный доступ к учебным материалам и заданиям.
\subsection{Проблемы существующих решений}
Основной проблемой существующих образовательных платформ является фрагментация функционала. На данный момент нет единой платформы, которая бы эффективно объединяла создание и проверку заданий, общение преподавателей и студентов, а также использовала бы технологии ИИ для автоматизированного анализа решений студентов. Это затрудняет образовательный процесс и снижает его эффективность, особенно в условиях быстро меняющихся требований дистанционного обучения.

Таким образом, для улучшения образовательного процесса существует необходимость в разработке единой интегрированной платформы, которая бы сочетала в себе все эти компоненты и обеспечивала бы максимально удобное взаимодействие для всех участников учебного процесса.
\subsection{Потребности образовательной среды}

Современные образовательные процессы предъявляют высокие требования к функциональности учебных платформ. Для эффективного взаимодействия между преподавателями и студентами необходимо создавать приложения, которые обеспечивают организационную и техническую поддержку всех ключевых элементов образовательного процесса.

\subsubsection{Доступность материалов и заданий}
Материалы и задания должны быть доступны студентам в любое время. Приложение должно обеспечивать размещение учебных ресурсов в различных форматах (текст, видео, презентации) и упрощать их поиск и использование. Это позволяет студентам готовиться к занятиям и выполнять задания без привязки ко времени, а преподавателям — быстро обновлять и дополнять учебные модули.

\subsubsection{Автоматизация проверок и оценки}
Автоматизированная проверка заданий существенно ускоряет процесс получения обратной связи. Использование искусственного интеллекта для анализа текста программы позволяет выявлять ошибки, давать подсказки и оценивать работы без участия преподавателя. Это освобождает ресурсы для индивидуальной поддержки студентов и более сложной экспертной оценки.

\subsubsection{Удобная система заданий и общения}
Приложение должно включать удобную систему создания и отслеживания заданий. Важно, чтобы преподаватели могли формулировать задания, прикреплять к ним материалы и получать результаты выполнения. Неотъемлемой частью также является возможность общения между участниками процесса~--- как в групповых, так и личных чатах, для обмена мнениями и получения поддержки.

\subsubsection{Интеграция всех процессов в одну систему}
Отдельные решения для чатов, размещения заданий и анализа текста программы создают фрагментированную среду. Необходима единая платформа, объединяющая все эти компоненты. Это упрощает взаимодействие, повышает удобство и эффективность обучения, а также снижает затраты на сопровождение и обучение работе с системой.

\subsubsection{Вывод}

Таким образом, при проектировании образовательной платформы следует учитывать потребности в постоянном доступе к материалам, автоматической проверке решений, поддержке взаимодействия и целостности функционала в рамках одного интерфейса.

\subsection{Технологии разработки клиентской части приложения}

Для реализации клиентской части платформы выбраны современные инструменты, обеспечивающие модульность, производительность, типизацию и масштабируемость интерфейса.

\subsubsection{TypeScript}
JavaScript является одним из самых популярных языков программирования для веб-разработки. Он широко используется для создания динамичных веб-страниц и приложений, поскольку позволяет работать с элементами DOM, асинхронно загружать данные и обеспечивать интерактивность пользовательских интерфейсов. Однако JavaScript имеет важный недостаток — отсутствие статической типизации. Это означает, что переменные и функции не привязываются к определённым типам данных, что может привести к ошибкам на этапе выполнения, которые трудно обнаружить в процессе разработки. Особенно это может быть проблемой в крупных приложениях, где сложно отслеживать все возможные типы данных и их изменения.

Для устранения этих проблем был разработан язык TypeScript, являющийся надмножеством JavaScript. TypeScript добавляет в JavaScript статическую типизацию, что позволяет разработчикам явно указывать типы данных для переменных и функций. Это значительно снижает вероятность ошибок и улучшает поддержку кода в будущем. Благодаря строгой типизации TypeScript помогает предотвращать баги, связанные с динамическими типами в JavaScript, и улучшает автозаполнение в редакторах кода. TypeScript распространяется как библиотека, которую можно интегрировать в проекты на JavaScript, обеспечивая совместимость с существующим кодом и позволяя постепенно внедрять типизацию без необходимости переписывать весь проект. Это особенно важно в крупных и м$ $асштабируемых приложениях, где несколько разработчиков работают с общими компонентами, и типизация помогает поддерживать консистентность кода на протяжении всего проекта.


\subsubsection{React}

React — библиотека для построения пользовательских интерфейсов, разработанная Facebook. Она широко используется в веб-разработке благодаря своей простоте, гибкости и высокой производительности. React обеспечивает декларативный стиль программирования, при котором разработчик описывает, как должен выглядеть интерфейс при заданном состоянии, а библиотека самостоятельно обновляет DOM при изменениях. Это упрощает разработку сложных и динамичных интерфейсов.

Ключевые особенности:
\begin{itemize}
\item \textbf{Компонентный подход}: Приложение разбивается на переиспользуемые и изолированные компоненты
\item \textbf{JSX-синтаксис}: Комбинация JavaScript и разметки, упрощающая написание UI
\item \textbf{Virtual DOM}: Эффективное обновление только изменённых элементов страницы
\item \textbf{Hooks API}: Современный способ управления состоянием и побочными эффектами
\end{itemize}

Преимущества для образовательных платформ:
\begin{itemize}
\item Быстрая разработка за счёт декларативности и компонентного подхода
\item Большое сообщество и развитая экосистема (Next.js, Redux, React Query и др.)
\item Поддержка SSR и SSG при использовании Next.js
\item Простая интеграция с библиотеками и сторонними сервисами
\end{itemize}

Ограничения:
\begin{itemize}
\item Отсутствие встроенной архитектуры — требует выбора и настройки дополнительных инструментов
\item Более низкий порог входа может привести к «разнообразию» архитектурных подходов в команде
\item Без Next.js не включает такие возможности как маршрутизация, SSR и API
\end{itemize}


\subsubsection{Angular}
Angular — полноценный MVC-фреймворк, предоставляющий комплексное решение для разработки enterprise-приложений. В отличие от React, Angular накладывает строгую архитектурную модель, что обеспечивает единообразие кодовой базы в крупных проектах. 

Ключевые особенности:
\begin{itemize}
\item \textbf{Двустороннее связывание данных}: Автоматическая синхронизация между моделью и представлением
\item \textbf{Инъекция зависимостей}: Встроенный механизм для управления сервисами и их зависимостями
\item \textbf{CLI-инструменты}: Генерация компонентов, сервисов и модулей через командную строку
\item \textbf{TypeScript-first}: Полная поддержка статической типизации "из коробки"
\end{itemize}

Преимущества для образовательных платформ:
\begin{itemize}
\item Строгая структура проекта для командной разработки
\item Встроенная поддержка форм с валидацией
\item Готовые решения для маршрутизации и HTTP-клиента
\end{itemize}

Ограничения:
\begin{itemize}
\item Высокий порог входа из-за сложной терминологии (декораторы, зоны, сервисы)
\item Большой размер бандла (до 500КБ в минимальной сборке)
\item Жёсткие требования к архитектуре
\end{itemize}


\subsubsection{Vue.js}

Vue.js — прогрессивный фреймворк, сочетающий подходы React и Angular. Особенно эффективен для быстрого прототипирования и проектов средней сложности.

Основные характеристики:
\begin{itemize}
\item \textbf{Реактивная система}: Автоматическое отслеживание зависимостей данных
\item \textbf{Гибкая интеграция}: Возможность использования как через CDN, так и в составе сложных SPA
\item \textbf{Single-File Components}: Объединение шаблона, логики и стилей в одном .vue-файле
\item \textbf{Переходные анимации}: Встроенная поддержка анимации состояний
\end{itemize}

Сильные стороны для учебных проектов:
\begin{itemize}
\item Понятная документация с интерактивными примерами
\item Мягкая кривая обучения для начинающих
\item Компактный размер ядра (24КБ в gzip)
\end{itemize}

Ограничения:
\begin{itemize}
\item Относительно небольшое сообщество по сравнению с React/Angular
\item Ограниченные возможности SSR без Nuxt.js
\item Меньший выбор готовых UI-библиотек
\end{itemize}


\subsubsection{Сравнительный анализ фреймворков}

\begin{table}[h]
\small
\centering
\caption{Сравнение характеристик фреймворков}
\begin{tabular}{|l|c|c|c|}
\hline
\textbf{Параметр}         & React + Next.js & Angular  & Vue.js   \\ \hline
Кривая обучения           & Средняя         & Высокая  & Средняя  \\ \hline
Сообщество                & Крупное         & Крупное  & Растущее \\ \hline
Производительность        & Высокая         & Средняя  & Высокая  \\ \hline
Гибкость архитектуры      & Высокая         & Минимальная & Средняя  \\ \hline
Поддержка SSR             & Встроенная (Next.js) & Встроенная & Nuxt.js \\ \hline
Поддержка TypeScript      & Да              & Да       & Да       \\ \hline
Готовая маршрутизация     & Да (Next.js)    & Да       & Да (Vue Router) \\ \hline
\end{tabular}
\end{table}

\textbf{Ключевые выводы:}
\begin{itemize}
\item \textbf{React + Next.js}: Предпочтителен для современных, SEO-оптимизированных приложений с гибкой архитектурой и возможностью инкрементального масштабирования
\item \textbf{Angular}: Подходит для крупных enterprise-систем с чёткими архитектурными требованиями и строгой типизацией
\item \textbf{Vue.js}: Оптимален для быстрого старта, MVP и небольших команд с ограниченным опытом
\end{itemize}

Для образовательной платформы выбран стек \textbf{React + Next.js}, поскольку он:
\begin{itemize}
\item Обеспечивает SSR и SSG «из коробки», что критично для SEO
\item Позволяет гибко комбинировать клиентскую и серверную логику
\item Имеет развитую экосистему и отличную интеграцию с библиотеками (Auth.js, Redux, Socket.IO)
\item Упрощает масштабирование и поддержку проекта в долгосрочной перспективе
\end{itemize}


\subsubsection*{Next.js}

Next.js — это популярный фреймворк для React, который значительно расширяет его возможности, предоставляя разработчикам мощные инструменты для создания высокопроизводительных веб-приложений. Одной из ключевых особенностей Next.js является поддержка рендеринга на сервере (SSR, Server-Side Rendering) и статической генерации контента (SSG, Static Site Generation). Эти подходы позволяют улучшить производительность приложений, поскольку они обеспечивают быструю загрузку страниц, оптимизированную для поисковых систем и пользователей.

С серверным рендерингом Next.js позволяет генерировать HTML на сервере для каждой страницы перед её отправкой клиенту, что обеспечивает быстрое отображение контента. Это особенно полезно для SEO, поскольку поисковые системы могут индексировать контент сразу после его загрузки. Такой подход значительно улучшает видимость веб-приложений в поисковых системах и способствует их более высокому ранжированию.

Одним из ключевых преимуществ Next.js является автоматическая разбивка кода (code splitting). Это означает, что Next.js разделяет приложение на небольшие части, которые загружаются только по мере необходимости, что помогает сократить время загрузки страниц и улучшить пользовательский опыт. Таким образом, браузер загружает только тот код, который необходим для отображения текущей страницы, а не весь код приложения.

Кроме того, Next.js поддерживает гибкие методы рендеринга, что дает разработчикам возможность выбирать наиболее подходящий способ для каждой страницы. Статическая генерация (SSG) идеально подходит для страниц, которые не изменяются часто и могут быть сгенерированы заранее, например, блоговые записи или страницы с информацией о компании. В то время как для динамических страниц, которые требуют актуализации данных на сервере при каждом запросе, можно использовать серверный рендеринг.

Next.js также упрощает настройку маршрутизации и управление данными. Встроенная система маршрутизации автоматически генерирует страницы на основе файловой структуры, что делает создание новых страниц и маршрутов простым и интуитивно понятным. Кроме того, Next.js предоставляет инструменты для работы с API, что позволяет без труда интегрировать серверную логику в приложение.

Еще одной значимой особенностью является поддержка типизации с помощью TypeScript, что делает разработку в Next.js ещё более удобной и безопасной. Комбинация TypeScript и Next.js позволяет создавать стабильные и хорошо структурированные приложения, минимизируя количество ошибок на этапе разработки.

Важно отметить, что реализацию приложения можно было бы построить и на чистом React, однако в этом случае значительная часть функциональности, такой как маршрутизация, SSR, SSG и работа с API, потребовала бы ручной настройки и подключения дополнительных библиотек. Использование Next.js избавляет от необходимости собирать всё вручную и предоставляет готовую, хорошо спроектированную архитектуру. Таким образом, Next.js становится де-факто стандартом разработки современных React-приложений. Это не просто библиотека, а фреймворк — а значит, он предлагает определённую «протоптанную дорожку», следование которой позволяет создавать более надёжные и поддерживаемые решения.

\subsubsection*{Socket.IO}

\textbf{Socket.IO} --- это JavaScript-библиотека с открытым исходным кодом, предназначенная для реализации двустороннего взаимодействия между клиентом и сервером в режиме реального времени. Основной особенностью данной технологии является использование собственного протокола поверх WebSocket с возможностью автоматического переключения на альтернативные методы связи (long polling и др.) при отсутствии поддержки WebSocket.

Преимущества использования Socket.IO:
\begin{itemize}
    \item \textbf{Гибкость}: Разработчики имеют полный контроль над архитектурой соединения, включая маршрутизацию сообщений, обработку событий, систему комнат (rooms) и пространств имён (namespaces).
    \item \textbf{Масштабируемость}: Поддержка кластеризации и горизонтального масштабирования при помощи Redis-адаптеров.
    \item \textbf{Совместимость с Node.js}: Socket.IO органично интегрируется в стек на основе Node.js, что упрощает реализацию единой инфраструктуры.
    \item \textbf{Производительность}: Низкая задержка при передаче сообщений благодаря постоянному соединению между клиентом и сервером.
    \item \textbf{Интеграция с Python}: Для серверной части на Python существует аналогичная библиотека \textit{python-socketio}, которая позволяет использовать те же возможности для реализации чатов и двустороннего общения между клиентом и сервером. Это делает возможным использование Socket.IO как на front-end (с помощью Node.js), так и на back-end (с помощью Python), обеспечивая совместимость и синхронизацию данных между клиентом и сервером.
\end{itemize}

Недостатки:
\begin{itemize}
    \item Необходимость разработки и поддержки собственной серверной инфраструктуры.
    \item Повышенная сложность при масштабировании без использования внешних инструментов (Redis, Kubernetes).
    \item Отсутствие встроенной панели мониторинга или аналитики соединений.
\end{itemize}

Socket.IO предоставляет высокий уровень кастомизации и гибкости, что делает его предпочтительным выбором в проектах, где важна точная настройка логики взаимодействия в реальном времени.

\subsubsection*{Pusher}

\textbf{Pusher} --- это облачная платформа, предоставляющая API и SDK для реализации push-уведомлений и двусторонней передачи данных в реальном времени. В отличие от Socket.IO, Pusher представляет собой managed-сервис, абстрагирующий низкоуровневые детали инфраструктуры.

Преимущества использования Pusher:
\begin{itemize}
    \item \textbf{Упрощённая интеграция}: SDK и готовые клиентские библиотеки позволяют быстро настроить соединение и передавать события.
    \item \textbf{Масштабируемость}: Обеспечивается на уровне платформы без участия разработчика.
    \item \textbf{Надёжность}: Pusher использует устойчивую облачную инфраструктуру с балансировкой нагрузки.
    \item \textbf{Аналитика и мониторинг}: Панель управления предоставляет данные о соединениях, событиях и каналах.
\end{itemize}

Ограничения:
\begin{itemize}
    \item \textbf{Платная модель}: Бесплатный тариф ограничен по числу соединений и событий, что делает использование невыгодным при росте нагрузки.
    \item \textbf{Зависимость от стороннего сервиса}: Потенциальные риски, связанные с отказоустойчивостью внешнего провайдера.
    \item \textbf{Ограниченная кастомизация}: Структура событий и поведения определяется особенностями платформы.
\end{itemize}

Pusher является удобным решением для проектов с ограниченными временными ресурсами и отсутствием внутренней серверной инфраструктуры, однако его применимость в образовательных продуктах с ограниченным бюджетом вызывает сомнения.

\subsubsection*{WebSocket в JavaScript}

JavaScript предоставляет встроенный класс \texttt{WebSocket} для организации двустороннего обмена данными между клиентом и сервером. Этот класс реализует базовую функциональность протокола WebSocket, предоставляя разработчикам простой способ обмена данными в реальном времени без необходимости использовать сторонние библиотеки. Однако, несмотря на свою доступность, использование \texttt{WebSocket} требует значительных усилий для реализации различных важных аспектов взаимодействия.

К примеру, при использовании \texttt{WebSocket} разработчик должен самостоятельно реализовать:
\begin{itemize}
    \item переподключение сокета при его падении,
    \item буферизацию сообщений в случае разрыва соединения,
    \item обработку таймаутов и ошибок,
    \item поддержку различных сред (например, Node.js и браузер),
    \item масштабирование на сервере.
\end{itemize}

Таким образом, несмотря на его доступность и гибкость, использование \texttt{WebSocket} требует написания значительного объема дополнительной логики. Это делает его менее удобным для быстрого внедрения в проект, особенно в случае масштабируемых приложений.

С учетом всех этих факторов, было принято решение отказаться от использования низкоуровневого \texttt{WebSocket} в пользу более высокоуровневых решений, таких как \texttt{Socket.IO} или \texttt{Pusher}, которые предоставляют необходимую функциональность и обрабатывают множество нюансов «из коробки», позволяя сосредоточиться на бизнес-логике приложения.

\subsubsection*{Сравнение и выбор технологии для чатов}

При сравнении библиотек Socket.IO и Pusher необходимо учитывать как технические, так и организационные аспекты. Таблица \ref{tab:chat-comparison} представляет краткое сопоставление ключевых параметров:

\begin{table}[h]
\centering
\caption{Сравнение технологий для реализации чатов в реальном времени}
\small
\label{tab:chat-comparison}
\begin{tabular}{|l|c|c|}
\hline
\textbf{Критерий} & \textbf{Socket.IO} & \textbf{Pusher} \\ \hline
Тип решения & Open-source библиотека & Облачный managed-сервис \\ \hline
Контроль над архитектурой & Полный & Ограниченный \\ \hline
Поддержка масштабирования & Через Redis и кластеризацию & Встроенная на уровне платформы \\ \hline
Простота настройки & Средняя (требует сервера) & Высокая (SDK) \\ \hline
Затраты на использование & Бесплатно & Платная модель \\ \hline
Интеграция с Node.js & Нативная & Через SDK \\ \hline
Надёжность соединения & Высокая & Высокая \\ \hline
Кастомизация протокола & Да & Нет \\ \hline
\end{tabular}
\end{table}

С учётом специфики проекта --- ограниченного бюджета, необходимости полной кастомизации и тесной интеграции с Node.js-сервером --- наилучшим выбором является использование \textbf{Socket.IO}. Данная библиотека предоставляет все необходимые механизмы для реализации масштабируемой и надёжной чат-системы, при этом позволяя оптимизировать производительность без привлечения сторонних сервисов.

Более того, благодаря открытому коду, Socket.IO не ограничивает разработчика в выборе архитектурных решений, а также обеспечивает возможность расширения функционала в будущем. В условиях ограниченных ресурсов образовательной платформы такой подход оказывается наиболее целесообразным.

\subsubsection*{Auth.js}
Auth.js --- это библиотека для реализации аутентификации и авторизации в веб-приложениях. Она является официальным решением, рекомендуемым и поддерживаемым фреймворком Next.js, что гарантирует хорошую интеграцию и поддержку всех необходимых функций. Библиотека позволяет легко подключать сторонние провайдеры аутентификации, такие как Google, Facebook и другие, а также реализовывать собственную аутентификацию с использованием базы данных. Auth.js обеспечивает надежную защиту пользовательских данных, управление сессиями, работу с токенами и предоставляет удобные API для быстрой настройки. Это решение упрощает реализацию всех ключевых механизмов безопасности, освобождая разработчиков от необходимости погружаться в тонкости реализации.

\subsubsection*{Redux}
Redux --- это библиотека для управления состоянием в приложениях, основанных на React. Она используется для централизованного хранения состояния приложения, что облегчает обмен данными между компонентами и упрощает их взаимодействие. Redux помогает избежать "проблемы пропс-дерева" в больших приложениях, когда передача данных через множество вложенных компонентов становится сложной. Хотя Redux часто используется в более сложных приложениях, в данном проекте его роль заключается в том, чтобы сделать взаимодействие между компонентами более организованным и улучшить предсказуемость состояния приложения.



\subsection{Требования к функциональности приложения}

Разрабатываемое приложение представляет собой образовательную платформу, ориентированную на университетскую среду. Основная цель~--- предоставить единое пространство для организации учебного процесса, взаимодействия между преподавателями и студентами, а также управления учебными структурами.

Каждый университет может зарегистрироваться на платформе и получить доступ к собственной административной панели. Через неё администраторы создают внутреннюю структуру, включающую институты, кафедры и учебные группы. Эти сущности служат основой для распределения доступа, назначения преподавателей и приглашения студентов.

Преподаватели, закреплённые за определёнными кафедрами, получают доступ ко всем учебным группам соответствующего подразделения. Через личную панель преподавателя реализован следующий функционал:
\begin{enumerate}
  \item Создание групповых чатов для любой группы своей кафедры;
  \item Размещение учебных материалов — как в групповых чатах, так и в личных сообщениях;
  \item Формирование и отправка заданий для студентов;
  \item Просмотр и анализ результатов выполнения заданий;
  \item Предоставление обратной связи студентам.
\end{enumerate}

Студенты, присоединённые к учебным группам, имеют доступ к следующему функционалу:
\begin{enumerate}
  \item Групповым чатам своей учебной группы;
  \item Личной переписке с преподавателями;
  \item Материалам, отправленным преподавателями;
  \item Заданиям, опубликованным в рамках их группы;
  \item Форме отправки решений и получению обратной связи.
\end{enumerate}

\subsection{Выводы}

В результате проведённого анализа можно сделать вывод о наличии устойчивого запроса на интегрированное образовательное приложение, способное решать сразу несколько ключевых задач. Современные платформы зачастую фокусируются либо на предоставлении учебных материалов, либо на коммуникации, либо на автоматизации проверки знаний, при этом разрозненность этих функций создаёт неудобства для всех участников образовательного процесса.

Потребности преподавателей включают в себя удобное управление группами и заданиями, возможность оперативной обратной связи, загрузку и распространение материалов. Студентам, в свою очередь, важно иметь стабильный и понятный доступ к заданиям, личным сообщениям и учебным ресурсам, а также возможность взаимодействовать с преподавателями и одногруппниками в привычном цифровом формате.

Предлагаемое приложение должно закрыть этот разрыв, обеспечив единую среду, в которой объединены функции управления учебным процессом, общения, публикации и проверки заданий. Такой подход позволит повысить качество образовательного взаимодействия, сократить технические барьеры и обеспечить более высокую степень вовлечённости пользователей.

Таким образом, на основании проведённого анализа подтверждается необходимость разработки новой системы, в которой ключевые элементы образовательной среды будут интегрированы в одно приложение, удовлетворяющее современным требованиям пользователей.

\newpage
\ESKDthisStyle{formII}
\section{ПРОЕКТИРОВАНИЕ АРХИТЕКТУРЫ ПРОЕКТА}
\ESKDcolumnII{ПРОЕКТИРОВАНИЕ АРХИТЕКТУРЫ ПРОЕКТА}

\subsection{Архитектура клиентской части системы}

Клиентская часть разрабатываемой системы реализована в виде одностраничного приложения (SPA) с использованием фреймворка Next.js, поддерживающего гибкий рендеринг — как на стороне клиента (CSR), так и на стороне сервера (SSR). Такое решение позволяет обеспечить как высокую производительность и отзывчивость пользовательского интерфейса, так и оптимизацию индексации содержимого поисковыми системами за счёт серверного рендеринга.

\subsubsection{Общая структура архитектуры}

Для организации структуры проекта был применён подход Feature-Sliced Design (FSD)~\cite{feature_sliced_design} — современная парадигма проектирования клиентской части приложений, ориентированная на модульность, масштабируемость и соответствие предметной области. В отличие от традиционных архитектур, основанных на технических слоях (например, разделение на компоненты, страницы или сервисы), FSD предполагает смысловое разделение приложения на функциональные модули, которые отражают реальные пользовательские сценарии и бизнес-логику.

Каждый модуль FSD отвечает за строго ограниченную область и содержит всё необходимое для своей работы: представление, поведение, взаимодействие с API и внутренние модели. Это способствует повышению читаемости и повторному использованию текста программ, а также упрощает сопровождение и развитие системы в долгосрочной перспективе.

\subsubsection{Архитектурная диаграмма}

\begin{figure}[h]
  \centering
  \includegraphics[width=0.7\linewidth]{static/fsdImage}
  \caption{Схема архитектуры клиентской части (FSD + Next.js)}
\end{figure}

На диаграмме представлены основные уровни и элементы архитектуры приложения. Следует отметить, что каждый слой в данной структуре ориентирован на строгое разграничение ответственности. Компоненты нижних уровней не имеют информации о вышестоящих слоях, что позволяет реализовать принцип инверсии зависимостей и минимизировать связанность между модулями.

\subsubsection{Слои архитектуры Feature-Sliced Design}

В таблице \ref{tab:fsd-layers} представлены слои архитектуры Feature-Sliced Design, сгруппированные по уровню абстракции и ответственности.

\begin{table}[h]
  \centering
  \caption{Слои архитектуры Feature-Sliced Design}
  \label{tab:fsd-layers}
  \begin{tabular}{|p{3cm}|p{11cm}|}
    \hline
    \textbf{Слой} & \textbf{Описание и назначение} \\ \hline
    \textit{app}      & Точка входа в приложение: глобальные стили, маршрутизация, провайдеры состояния, интеграции с внешними сервисами. \\ \hline
    \textit{pages}    & Страницы, связанные с маршрутизацией. Формируются из виджетов и не содержат бизнес-логики. \\ \hline
    \textit{widgets}  & Крупные элементы интерфейса, отражающие пользовательские сценарии, например, чат, список заданий, панель управления. \\ \hline
    \textit{features}& Изолированные пользовательские функции, такие как авторизация, отправка сообщений или регистрация. Могут включать бизнес-логику и вызовы API. \\ \hline
    \textit{entities} & Базовые предметные сущности предметной области, включающие типы, схемы, API и UI-представление. \\ \hline
    \textit{shared}   & Универсальные компоненты, утилиты и типы, переиспользуемые во всём проекте. \\ \hline
  \end{tabular}
\end{table}

\subsubsection{Концепция срезов}

Ключевым элементом архитектурного подхода Feature-Sliced Design является понятие срезов (англ. \textit{slices}). Под срезом понимается логически обособленный модуль, реализующий завершённую часть функциональности приложения. Каждый срез может содержать собственные модели данных, визуальные компоненты, бизнес-логику, а также механизмы взаимодействия с внешними источниками данных.

Ниже приведены примеры различных типов срезов и их ответственности в структуре проекта:

\begin{enumerate}
  \item \textit{features/login} — срез, реализующий сценарий авторизации пользователя;
  \item \textit{entities/task} — срез, содержащий всё, что связано с сущностью «задание»;
  \item \textit{widgets/ChatWindow} — срез, объединяющий функциональность и интерфейс чат-интерфейса;
  \item \textit{pages/home} — срез, реализующий главную страницу приложения.
\end{enumerate}

\subsubsection{Горизонтальное деление на сегменты}

Каждый срез, независимо от своего уровня, может быть дополнительно разделён на сегменты (англ. \textit{segments}) — логические подкатегории, структурирующие содержимое среза по назначению модуля. В отличие от слоёв, которые представляют вертикальную иерархию, сегменты формируют горизонтальное деление и обеспечивают внутреннюю организацию модулей.

Наиболее распространённые типы сегментов включают:
\begin{enumerate}
  \item \textit{ui} — визуальные компоненты и стили, определяющие отображение данных;
  \item \textit{model} — модели данных, хранилища состояния, типизация и бизнес-логика;
  \item \textit{api} — функции для работы с внешними сервисами, включая описание типов запросов и маппинг ответов;
  \item \textit{lib} — вспомогательные функции и библиотеки, используемые в пределах данного среза;
  \item \textit{config} — конфигурационные файлы и переключатели функциональности.
\end{enumerate}

\subsubsection*{Преимущества выбранного подхода}

Применение архитектуры Feature-Sliced Design в контексте разрабатываемого клиентского приложения позволило достичь следующих результатов:
\begin{enumerate}
  \item Чёткое разграничение обязанностей между модулями и слоями;
  \item Улучшенная масштабируемость проекта без деградации структуры;
  \item Повышенная модульность, обеспечивающая лёгкость в тестировании и повторном использовании текста программы;
  \item Создание условий для быстрой и эффективной интеграции новых членов команды в разработку;
  \item Архитектура, ориентированная на задачи и бизнес-логику, а не на технические детали.
\end{enumerate}

В совокупности данные свойства делают архитектурное решение устойчивым к росту функциональности, улучшая поддержку и развитие системы в долгосрочной перспективе.
\subsection{Проектирование интерфейсных подсистем и экранов}

Одной из ключевых задач при проектировании клиентской части является логическое и функциональное разделение интерфейса на подсистемы, каждая из которых реализует отдельный аспект пользовательского взаимодействия. Такое разделение позволяет обеспечить модульность, переиспользуемость компонентов и устойчивость к изменениям.

Проект разрабатывается в архитектуре Feature-Sliced Design, что накладывает дополнительную дисциплину на организацию экранов и компонентов: все подсистемы формируются из \texttt{entities}, \texttt{features}, \texttt{widgets} и собираются в \texttt{pages}, а общая инфраструктура — в слое \texttt{shared}.

\subsubsection{Выделение ключевых интерфейсных подсистем}

Клиентская часть разработанной платформы организована в виде набора функционально обособленных интерфейсных подсистем, каждая из которых отвечает за определённый аспект пользовательского взаимодействия и бизнес-логики. Такое разграничение позволяет повысить масштабируемость и сопровождаемость системы, а также упростить процесс тестирования и внедрения новых функций.

На основании анализа требований к функциональности приложения и сценариев использования пользователями различных ролей (администратор, преподаватель, студент), были выделены следующие ключевые подсистемы.

\begin{enumerate}
  \item \textbf{Подсистема авторизации и регистрации}\\
  Отвечает за обеспечение безопасного входа в систему, регистрацию новых пользователей и управление сессиями. Аутентификация реализована с применением библиотеки \texttt{Auth.js} и технологии JSON Web Token (JWT), что позволяет надёжно разграничивать доступ к различным разделам интерфейса в зависимости от роли пользователя.  

  Регистрация в системе представлена в виде трёх пользовательских сценариев, адаптированных под особенности образовательного процесса:
  \begin{itemize}
    \item Первый сценарий реализован для новых организаций (институтов) и сопровождается созданием административной учётной записи. На этом этапе формируется корневая структура управления учреждением.
    \item Второй и третий сценарии предназначены для регистрации преподавателей и студентов соответственно. Оба сценария доступны исключительно по индивидуальным приглашениям, что обеспечивает контроль над составом участников образовательного процесса и предотвращает несанкционированный доступ.
  \end{itemize}
  
  Подсистема тесно связана с механизмами контроля прав доступа и маршрутизации, определяя поведение интерфейса в зависимости от текущего статуса пользователя.

  \item \textbf{Подсистема управления университетом}\\
  Реализует административную логику, связанную с конфигурацией организационной структуры образовательного учреждения.
  
  Основными функциями данной подсистемы являются:
  \begin{itemize}
    \item Создание и удаление структурных единиц — институтов, кафедр, учебных групп;
    \item Управление персоналом: добавление и блокировка преподавателей и студентов;
    \item Генерация приглашений для входа новых участников на платформу с конкретной ролью;
    \item Отображение данных по структуре учреждения.
  \end{itemize}
  
  Визуально подсистема представлена в виде панели управления с множеством таблиц, форм и интерактивных элементов, обеспечивающих быстрый доступ к ключевым административным операциям. Все действия защищены авторизацией и доступны только пользователям с соответствующими правами доступа.

  \item \textbf{Подсистема работы с заданиями и отправкой решений}\\
  Данная подсистема предназначена для организации учебной деятельности: выдачи заданий преподавателями, загрузки решений студентами и автоматизированного анализа этих решений. 
  
  Основной интерфейс включает:
  \begin{itemize}
    \item Панель создания и редактирования заданий с параметрами проверки;
    \item Представление активных и завершённых заданий для студентов;
    \item Историю отправок с отображением результатов и статуса проверки.
  \end{itemize}
  
  Задания связаны с группами. Система также предоставляет базовую аналитику по результатам выполнения.

  \item \textbf{Подсистема обмена сообщениями (чаты)}\\ 
  В рамках образовательного процесса большое значение имеет возможность коммуникации. Подсистема реализует обмен сообщениями как в рамках учебной группы, так и в формате личной переписки.  
  Технически реализация основана на технологии WebSocket с использованием библиотеки \texttt{Socket.IO}, что обеспечивает мгновенную доставку сообщений и минимальную задержку при передаче данных. 
  
  Основной функционал включает:
  \begin{itemize}
    \item Подключение к соответствующим «комнатам» (группам или диалогам);
    \item Отправку и приём текстовых сообщений;
    \item Отображение истории переписки;
    \item Поддержку вложений и индикаторов прочтения.
  \end{itemize}
  
  Доступ к системе чатов осуществляется только после успешной авторизации, что исключает участие анонимных пользователей и обеспечивает безопасность переписки.

  \item \textbf{Подсистема AI-анализа решений}\\
  Одной из уникальных особенностей платформы является использование искусственного интеллекта для автоматической оценки студенческих заданий.
  
  Подсистема предназначена для получения и визуализации результатов AI-анализа, включающих:
  \begin{itemize}
    \item Оценку корректности кода;
    \item Проверку на соответствие заданию;
    \item Выявление потенциальных ошибок и некорректных конструкций;
    \item Комментарии, рекомендации и текстовые пояснения.
  \end{itemize}
  
  Результаты анализа отображаются в виде отчёта с возможностью преподавателя оставить дополнительные замечания. Таким образом, снижается нагрузка на преподавателя и повышается объективность оценивания.

\end{enumerate}

Каждая из указанных подсистем обладает чётко определёнными входными и выходными данными, а также взаимодействует с другими модулями системы. Например, подсистема работы с заданиями напрямую связана как с AI-анализом, так и с интерфейсами преподавателя и студента, а система чатов — с механизмами авторизации и маршрутизации. Такое проектирование обеспечивает гибкость, надёжность и чёткую масштабируемость клиентской архитектуры.


\subsubsection{Страницы и их структура}

Разработка интерфейсной части веб-приложения требует не только реализации функциональных компонентов, но и проектирования логически связанных экранов, отражающих ключевые сценарии взаимодействия пользователя с системой. В рамках платформы каждая страница представляет собой самостоятельный интерфейсный модуль, обслуживающий одну или несколько бизнес-задач, соответствующих определённой роли: студент, преподаватель, администратор.

Процесс формирования страниц реализован с применением маршрутизации, встроенной в фреймворк \texttt{Next.js}, что обеспечивает высокую производительность и поддержку серверного рендеринга. Страницы не только представляют визуальный уровень приложения, но и координируют работу между компонентами пользовательского интерфейса, бизнес-логикой и хранилищем состояния. 

Архитектурно страницы собираются из обособленных функциональных элементов, разработанных согласно принципам FSD: пользовательские действия реализуются в слое \texttt{features}, отображаемые сущности формируются на базе \texttt{entities}, а объединение этих блоков происходит внутри \texttt{widgets}. Такой подход позволяет повысить согласованность, переиспользуемость и модульность кода, а также снижает зависимость между различными частями интерфейса.

Ниже приведён перечень ключевых страниц, отражающих основную логику пользовательского взаимодействия.

\begin{itemize}
  \item \textbf{Страница авторизации}\\  
  Отвечает за вход пользователя в систему. Содержит форму для ввода учётных данных, а также реализует логику валидации, передачи данных на сервер, обработки ошибок и сохранения сессионного токена. После успешной авторизации пользователь перенаправляется на главную страницу, соответствующую его роли.

  \item \textbf{Страница заданий}\\
  Представляет собой ключевой интерфейс для организации и выполнения учебной деятельности. Интерфейс страницы включает:
  \begin{itemize}
    \item Список классов и учебных групп, к которым привязан пользователь;
    \item Перечень активных заданий в рамках каждой группы;
    \item Доступ к подробному описанию заданий, срокам сдачи и параметрам оценивания;
    \item Отправку решений и просмотр результатов, включая отчёты AI-анализа.
  \end{itemize}
  Для преподавателя дополнительно предоставляется интерфейс управления заданиями, а также доступа к аналитике по группам и студентам.

  \item \textbf{Административная панель института}\\
  Данная страница является основным рабочим инструментом пользователя с ролью администратора. Интерфейс включает:
  \begin{itemize}
    \item Управление иерархией образовательного учреждения (институты, кафедры, группы);
    \item Назначение и блокировка пользователей (студентов и преподавателей);
    \item Просмотр структуры учреждения в табличной форме;
    \item Генерация и отправка приглашений на регистрацию;
    \item Журнал событий и контроль активности пользователей.
  \end{itemize}
  Все действия на данной странице требуют повышенного уровня доступа и сопровождаются системой уведомлений о результатах операций.

  \item \textbf{Страница чатов}\\
  Реализует коммуникационную составляющую платформы. Пользователь получает доступ к:
  \begin{itemize}
    \item Перечню активных диалогов (личных и групповых);
    \item Истории сообщений в рамках выбранного чата;
    \item Форме для отправки сообщений и файлов;
    \item Интерактивным элементам: индикаторы доставки, статус прочтения, поиск по переписке.
  \end{itemize}
  Для преподавателей также предусмотрена возможность создания новых групповых чатов для своих учебных групп.

\end{itemize}

Все функциональные страницы приложения, за исключением экранов регистрации и входа, используют единый шаблон компоновки \texttt{AppLayout}, обеспечивающий целостность визуального восприятия и унификацию пользовательского опыта. Данный шаблон включает в себя общие элементы интерфейса — верхнюю панель навигации, боковое меню и основной контейнер для отображения содержимого, который динамически наполняется в зависимости от текущего маршрута. 

Использование общего каркаса позволяет сохранить структурную согласованность между различными разделами системы, облегчает адаптацию пользователей к интерфейсу и упрощает внедрение изменений. Кроме того, архитектурное разделение логики и представления на уровне страниц способствует инкапсуляции ответственности, а также повышает читаемость и сопровождаемость кода. В рамках маршрутизации обеспечивается централизованное управление доступом, фильтрацией и визуализацией данных с учётом ролей пользователей.

Таким образом, структура страниц приложения отражает как технические требования архитектуры, так и практическую ориентацию на удобство и эффективность работы конечных пользователей.

\subsubsection{Компоненты и принципы их структурирования}

Компонентная модель проекта выстроена на основе принципов повторного использования, инкапсуляции и чёткого разделения ответственности между уровнями абстракции. Все компоненты, применяемые в рамках клиентского интерфейса, условно делятся на два основных класса: общие (универсальные) и специфические (бизнес-ориентированные).

\begin{itemize}
  \item \textbf{Общие компоненты} (\texttt{shared/ui}) представляют собой переиспользуемые элементы пользовательского интерфейса, не зависящие от предметной области. К ним относятся кнопки, поля ввода, модальные окна, индикаторы загрузки, элементы навигации, уведомления и другие базовые визуальные элементы. Такие компоненты широко применяются на всех уровнях интерфейса и не содержат бизнес-логики.
  
  \item \textbf{Специфические компоненты}, разрабатываемые в слоях \texttt{entities} и \texttt{widgets}, предназначены для реализации прикладной логики и отображения конкретных сущностей системы. Примерами являются компоненты отображения сообщений в чате, карточек заданий, панели управления преподавателя, таблиц пользователей и др. Они обладают внутренним состоянием и часто включают обращение к хранилищу или API.
\end{itemize}

Такое структурное разграничение существенно упрощает масштабирование проекта, облегчает поддержку и повторное использование элементов, а также способствует разделению труда между разработчиками.

\subsubsection{Распределение логики по слоям архитектуры}

Функциональная логика клиентской части системы строго распределяется по слоям архитектуры Feature-Sliced Design, что обеспечивает высокую модульность и инкапсуляцию поведения. Каждому слою соответствует свой уровень ответственности:

\begin{itemize}
  	\item В слое \texttt{entities} сосредоточена модель предметной области: типизация, структура сущностей, атомарные компоненты отображения, такие как \texttt{Registration}, \texttt{Department}, \texttt{Group}. Данный слой реализует описание и базовое представление данных без привязки к конкретным действиям пользователя.
  
	\item Слой \texttt{features} содержит реализацию отдельных действий, составляющих пользовательские сценарии: отправка сообщений, регистрация, загрузка задания, подтверждение действия и т.д. Эти модули инкапсулируют конкретные шаги взаимодействия пользователя с интерфейсом, часто включая локальное состояние и вызовы к API. \texttt{Features} могут быть использованы многократно и комбинироваться для построения более сложных сценариев.
	
	\item Слой \texttt{widgets} представляет собой реализацию полноценных пользовательских сценариев — законченных интерфейсных блоков, решающих определённую задачу. Примеры: интерфейс чата, панель с заданиями, административный модуль управления группами. Каждый виджет объединяет несколько фич и сущностей, обеспечивая завершённую и логически связанную единицу поведения.
	
	\item Слой \texttt{pages} выполняет роль точки входа и финальной сборки пользовательских сценариев. Здесь происходит выбор и компоновка виджетов в зависимости от маршрута, роли пользователя и контекста сессии. Кроме того, на уровне страниц задаются глобальные обёртки, обеспечиваются ограничения доступа, инициализируются загрузки данных и подключаются необходимые провайдеры. Таким образом, \texttt{pages} являются связующим слоем между навигацией и пользовательским опытом.
\end{itemize}

Такое строгое распределение обязанностей по слоям позволяет исключить дублирование логики, минимизировать связанность между модулями и обеспечить чёткую иерархию ответственности.

\subsubsection{UX-решения и пользовательские сценарии}

Для повышения удобства и доступности платформы, особенно в условиях использования её разными категориями пользователей, были реализованы следующие решения в области пользовательского опыта (UX):

\begin{itemize}
  \item \textbf{Централизованная навигация} — через универсальный макет, включающий боковую и верхнюю панели, интерфейс остаётся единообразным и интуитивно понятным вне зависимости от текущего маршрута.
  \item \textbf{Toast-уведомления} — реализация мгновенной обратной связи при выполнении действий: успешная отправка формы, ошибка сети, получение новых сообщений.
  \item \textbf{Обработка пустых состояний и ошибок} — предусмотрены интерфейсы для ситуаций отсутствия данных, ошибок загрузки или недоступности сервера.
 \end{itemize}

В результате, пользователь получает предсказуемый и непрерывный опыт взаимодействия с системой вне зависимости от своей роли и уровня подготовки.

\subsubsection*{Вывод}

Проектирование интерфейсной части приложения основывается на чётком структурном и функциональном разграничении компонентов, ориентированном на принципы модульности и масштабируемости. Использование архитектуры Feature-Sliced Design позволяет изолировать бизнес-логику, визуальные компоненты и маршрутизацию, что делает интерфейс легко расширяемым и сопровождаемым.

Реализованная организация интерфейса, объединяющая единый шаблон компоновки, повторно используемые компоненты и специфические бизнес-модули, способствует формированию целостного пользовательского опыта. Выбранные UX-решения обеспечивают удобство и логичность навигации, а также высокую отзывчивость системы при взаимодействии с пользователем.


Интерфейсная часть проекта построена на модульной архитектуре, основанной на бизнес-функциях. Подсистемы выделены логически, а их реализация изолирована в независимые модули, что повышает удобство поддержки, расширения и переиспользования компонентов.

\subsection{Проектирование взаимодействия с сервером и WebSocket}

Клиентская часть приложения активно взаимодействует с сервером для получения и отправки данных, а также поддерживает постоянное соединение с помощью WebSocket в рамках подсистемы обмена сообщениями. При проектировании механизма взаимодействия были учтены требования безопасности, стабильности соединения, обработки ошибок, а также необходимость автоматического обновления сессионных данных пользователя.

\subsubsection{Аутентификация и управление токенами}

Для обеспечения защищённого доступа к функциональности платформы используется система авторизации с применением JSON Web Token (JWT). Управление сессией пользователя реализовано через библиотеку \texttt{Auth.js}, которая выполняет роль промежуточного слоя между клиентом и системой хранения токенов.

Поскольку access token не хранится в открытом виде на стороне клиента, его получение и обновление возможны только через специальные механизмы обращения к серверу. При первичном входе пользователя токен сохраняется в защищённой cookie. Для последующего взаимодействия клиенту необходимо перед каждым сетевым запросом удостовериться в актуальности токена.

В рамках проектирования был реализован механизм автоматической валидации access token. Каждый раз перед отправкой запроса выполняется проверка срока его действия. В случае, если срок истёк, инициируется обращение к серверу с целью его обновления через \texttt{Auth.js}. После получения нового токена он автоматически обновляется в cookie, и запрос выполняется повторно.

Дополнительно, учитывая асинхронную природу работы клиента, был предусмотрен механизм защиты от многократного обновления токена в случае параллельных запросов. Реализована единая точка обращения к логике проверки и обновления токена. Если несколько запросов запускаются одновременно, и access token требует обновления, то все они получают результат единого процесса рефреша, исключая избыточные сетевые обращения. Это позволяет сократить нагрузку на сервер и избежать конфликтов при замене токенов.

\subsubsection{Унифицированная функция отправки запросов}

Для стандартизации сетевого взаимодействия была разработана функция \texttt{sendRequest}, инкапсулирующая всю логику подготовки и отправки запросов к серверу. Она реализует следующие функции:
\begin{itemize}
  \item Преобразование данных из формата JavaScript в JSON (\texttt{JSON.stringify});
  \item Добавление заголовков, включая авторизационный \texttt{Authorization: Bearer};
  \item Обработка возможных ошибок и повторная попытка в случае обновления токена;
  \item Поддержка различных HTTP-методов.
\end{itemize}

Данный подход позволяет централизованно управлять всей логикой сетевого взаимодействия и снижает вероятность ошибок при интеграции новых клиентских модулей.

\subsubsection{Взаимодействие через WebSocket}

Для реализации функциональности обмена сообщениями и других сценариев, требующих обновления данных в режиме реального времени, в клиентской части приложения применяется технология WebSocket. В отличие от традиционного HTTP-взаимодействия, WebSocket обеспечивает постоянное двустороннее соединение между клиентом и сервером, позволяя оперативно обмениваться событиями без необходимости постоянного опроса сервера.

В рамках проекта используется библиотека \texttt{Socket.IO}, которая предоставляет высокоуровневую обёртку над стандартным WebSocket-протоколом и значительно упрощает реализацию клиентского взаимодействия. Преимуществами \texttt{Socket.IO} являются:
\begin{itemize}
  \item Поддержка автоматического переподключения при обрыве соединения;
  \item Передача структурированных событий с именами и аргументами;
  \item Интеграция с механизмами авторизации и middleware;
  \item Гибкость при работе с пространствами имён и комнатами;
  \item Совместимость с fallback-транспортами (в случае недоступности WebSocket).
\end{itemize}

На стороне клиента был разработан и реализован специализированный хук \texttt{useSocket}, инкапсулирующий всю логику работы с соединением. Он обеспечивает:
\begin{itemize}
  \item Инициализацию соединения с сервером по заданному адресу;
  \item Отправку и приём событий с типизированной структурой данных;
  \item Автоматическую подписку и отписку от необходимых каналов;
  \item Управление жизненным циклом подключения;
  \item Обработку ошибок и логирование сетевых событий.
\end{itemize}

Для каждого подключённого пользователя создаётся уникальное пространство взаимодействия, определяемое его ролью и идентификатором. Это позволяет реализовать маршрутизацию сообщений между конкретными участниками чата (включая как личные, так и групповые диалоги), а также централизованно управлять доступом к отдельным коммуникационным потокам.

Особое внимание в проектировании было уделено вопросу авторизации в рамках WebSocket-сессии. В момент установления соединения клиент передаёт access token в параметрах подключения. На стороне сервера выполняется проверка подлинности токена, после чего соединение активируется. Однако, с учётом ограниченного срока действия access token, реализована логика автоматического переподключения. При обнаружении истечения токена:
\begin{enumerate}
  \item Инициируется обновление токена через заранее определённый механизм;
  \item Закрывается текущее соединение;
  \item После получения нового токена создаётся новое подключение с обновлёнными параметрами.
\end{enumerate}

Таким образом, WebSocket-подсистема функционирует устойчиво и прозрачно для конечного пользователя, обеспечивая бесперебойную передачу сообщений даже в условиях потери соединения или истечения сессии. Благодаря использованию \texttt{Socket.IO} удалось добиться высокой надёжности, расширяемости и лёгкости сопровождения реализации в рамках модульной архитектуры клиентской части.

\subsubsection{Вывод}

Взаимодействие клиентской части с сервером реализовано с учётом требований к безопасности, надёжности и масштабируемости. Реализованные механизмы автоматического обновления токенов, централизованная отправка запросов и продуманная интеграция WebSocket-соединения позволяют обеспечить стабильную работу интерфейса и корректную обработку всех пользовательских сценариев в режиме реального времени.

\subsection{Интеграция ИИ-модуля DeepSeek в архитектуру проекта}

\subsubsection{Контейнеризация DeepSeek}
Для обеспечения независимого жизненного цикла и лёгкой масштабируемости ИИ-компонента DeepSeek развёртывается в виде изолированного Docker-контейнера. Такой подход позволяет:
\begin{itemize}
	\item Быстро запускать и останавливать сервис без влияния на основное приложение.
	\item Поддерживать разные версии DeepSeek параллельно, экспериментируя с обновлениями моделей.
	\item Мигрировать между хостами и облачными средами с минимальными изменениями конфигурации.
\end{itemize}
Контейнер содержит все необходимые зависимости: предобученные трансформерные модели, библиотеки для анализа AST и семантической обработки, а также механизмы формирования отчётов. При этом фронтенд остаётся полностью изолированным от этих деталей — ему достаточно знать только адрес и формат запросов.

\subsubsection{Взаимодействие клиентской части с DeepSeek}
Клиентская часть приложения, реализованная на React и Next.js, отправляет HTTP-запросы через единый API-шлюз. Основная логика взаимодействия упрощена до следующих действий:
\begin{itemize}
	\item Преподаватель выбирает работу студента и нажатием кнопки инициирует анализ.
	\item Фронтенд направляет корректно сформированный запрос к API, где указаны идентификатор работы и необходимые параметры оценки.
	\item По готовности отчёта фронтенд периодически проверяет статус анализа и загружает результаты в привычном формате.
	\item Все шаги интегрированы в единый пользовательский поток: преподаватель не видит инфраструктуры контейнеров, а получает только готовый отчёт.
\end{itemize}
Такая схема взаимодействия гарантирует простоту клиентской логики и минимальную связность с внутренним устройством ИИ-сервиса.

\subsubsection{Преимущества контейнеризированного подхода}
Контейнеризация DeepSeek даёт следующие ключевые плюсы:
\begin{itemize}
	\item \textbf{Изоляция нагрузки:} анализ кода выполняется в отдельном окружении, не влияя на отзывчивость пользовательского интерфейса.
	\item \textbf{Горизонтальное масштабирование:} при необходимости обрабатывать большое число запросов можно запускать несколько инстансов контейнера.
	\item \textbf{Упрощённое сопровождение:} обновление ИИ-компонента сводится к выпуску нового Docker-образа без правок в фронтенде.
	\item \textbf{Гибкость развертывания:} контейнеры можно запускать как локально, так и в облаке, сохраняя одинаковую конфигурацию.
\end{itemize}


\subsubsection{Вывод}
Таким образом, архитектура остаётся прозрачной на уровне клиентского приложения, а DeepSeek надёжно инкапсулирован и готов к дальнейшему эволюционному развитию.

\subsection{Обеспечение безопасности клиентской части}

Безопасность пользовательского взаимодействия является важнейшей составляющей архитектуры клиентской части платформы. В условиях, когда доступ к различным модулям приложения осуществляется на основе ролей, а взаимодействие с данными сопровождается отображением пользовательского контента, особое внимание уделяется как управлению доступом, так и защите от потенциальных атак, включая межсайтовое выполнение скриптов (XSS). В данной подсистеме реализован комплекс механизмов, направленных на защиту данных и поведения интерфейса со стороны клиента.

\subsubsection{Механизм разграничения доступа с использованием Middleware}

Одним из ключевых компонентов обеспечения безопасности клиентской части является система контроля доступа на основе промежуточного слоя — \texttt{middleware}. В рамках архитектуры \texttt{Next.js}, middleware представляет собой функцию, исполняемую при каждом запросе к защищённым маршрутам. Она позволяет перехватывать обращения к страницам до их рендеринга и на этой стадии выполнять необходимые проверки: наличие токена, его валидность, а также права пользователя.

В контексте реализуемой платформы, при обращении пользователя к любой защищённой странице клиентская логика через middleware извлекает JWT-токен из cookies и дешифрует его содержимое, получая так называемый payload — полезную нагрузку токена. Внутри неё содержится вся необходимая информация о пользователе: уникальный идентификатор, срок действия сессии и, что особенно важно, роль в системе (\texttt{admin}, \texttt{teacher}, \texttt{student}).

На основе этой информации middleware выполняет следующие действия:
\begin{itemize}
  \item Если пользователь не авторизован (отсутствует валидный токен) — происходит автоматический редирект на страницу входа.
  \item Если пользователь авторизован, но не обладает достаточными правами — осуществляется перенаправление на главную страницу или отображается сообщение об отказе в доступе.
  \item Если пользователь обладает необходимой ролью — доступ к ресурсу предоставляется, и страница загружается с соответствующим контентом.
\end{itemize}

Таким образом, механизм \texttt{middleware} обеспечивает надёжную фильтрацию обращений к различным частям интерфейса, предотвращая несанкционированный доступ и обеспечивая соответствие поведения приложения политике разграничения прав.

\subsubsection{Роль и защита при работе с форматируемым текстом}

Дополнительным вектором потенциальной угрозы в клиентских приложениях является отображение форматируемого текста, особенно если пользователь имеет возможность редактировать его содержимое. В таких случаях возрастает риск внедрения вредоносных скриптов, замаскированных под обычный HTML.

Для решения данной задачи в проекте используется специализированная библиотека \texttt{tiptap} — расширяемый редактор форматированного текста, основанный на \texttt{ProseMirror}. Одним из ключевых преимуществ \texttt{tiptap} является контроль над тем, какие HTML-теги и атрибуты допускаются к интерпретации и отображению. Таким образом, даже если пользователь попытается вставить опасный HTML-код (например, \texttt{<script>} или инъекцию с обработчиком событий), редактор проигнорирует или удалит такие элементы на этапе парсинга.

Технически это реализуется следующим образом:
\begin{itemize}
  \item При вводе текстов редактор не сохраняет «сырые» HTML-строки, а формирует безопасное представление контента на основе строго описанных схем;
  \item При рендеринге текста из базы или состояния редактор отображает только те элементы, которые были описаны как допустимые;
  \item Расширения (extensions), добавляемые к \texttt{tiptap}, позволяют точно контролировать список разрешённых действий и структур (например, разрешить только \texttt{<strong>}, \texttt{<em>}, \texttt{<ul>} и \texttt{<code>}).
\end{itemize}

Таким образом, даже при наличии активной формы редактирования форматируемого текста, пользовательская среда остаётся защищённой от внедрения опасного контента. Редактор фактически выступает в роли фильтра, строго ограничивающего возможный набор HTML-инструкций.

\subsubsection{Вывод}

Комплекс реализованных решений позволяет эффективно защитить клиентскую часть приложения как от внешнего вмешательства, так и от ошибочного доступа со стороны пользователей. Механизм middleware обеспечивает надёжную проверку подлинности сессии и прав доступа до загрузки страниц, а редактор \texttt{tiptap} гарантирует безопасность при работе с форматируемым текстом. Такое сочетание архитектурных и прикладных решений способствует созданию устойчивой и безопасной пользовательской среды.

\subsection{Покрытие бизнес-логики юнит-тестами}

Наш подход к обеспечению надёжности клиентского приложения фокусируется на обязательном юнит-тестировании бизнес-логики при помощи Jest. Тестирование UI-компонентов считается вторичным: написание и поддержка сравнений HTML-вывода часто оказывается более трудоёмким и хрупким, чем простая визуальная валидация. Визуальный осмотр интерфейса преподавателем или дизайнером даёт более быстрый и надёжный результат без лишних накладных расходов.

Основные принципы нашего подхода:
\begin{enumerate}
  \item Юнит-тесты покрывают функции, отвечающие за валидацию данных, расчёт оценок и другие критичные механизмы, гарантируя корректность работы независимо от изменений UI;
  \item Модульные тесты интерфейсов не используются: динамика верстки и частые мелкие правки приводят к избыточным провалам тестов и дополнительным усилиям на их поддержку;
  \item Благодаря отказу от snapshot-тестирования HTML структура текста программы остаётся гибкой, а команда освобождает время на развитие функциональности вместо постоянной правки тестов;
  \item Автоматический запуск тестов бизнес-логики при каждом пуше позволяет мгновенно обнаруживать регрессии и поддерживать стабильность продукта;
  \item Для окончательной валидации интерфейса используется ручной осмотр ключевых страниц после сборки, что даёт уверенность в корректности отображения без сложных технических средств.
\end{enumerate}

Такой подход обеспечивает надёжность самой логики приложения и упрощает работу с UI: вместо громоздких автоматизированных тестов на вёрстку мы применяем человеческую экспертизу для финальной проверки внешнего вида и пользовательского опыта.

\newpage
\ESKDthisStyle{formII}
\section{ПРОГРАММНАЯ РЕАЛИЗАЦИЯ}
\ESKDcolumnII{ПРОГРАММНАЯ РЕАЛИЗАЦИЯ}

\setcounter{figure}{0} 
\makeatletter
  \renewcommand{\thefigure}{3.\arabic{figure}}
\makeatother

\subsection{Архитектура по Feature-Sliced Design}

Feature-Sliced Design (FSD) — это гибкий набор рекомендаций и подходов по логической организации клиентского части программы, основанных на выделении независимых функциональных слоёв и зон ответственности. В отличие от строгих стандартов, архитектура FSD предоставляет разработчикам свободу выбора конкретных решений, сохраняя при этом единый общий каркас структуры. Такая архитектура повышает читаемость, масштабируемость и тестируемость приложения, а также упрощает командную разработку и поддержку приложения.

\subsubsection{Слой \textit{shared}}

Слой \textit{shared} служит хранилищем нижнего уровня для общих и переиспользуемых компонентов, утилит и ресурсов, не зависящих от конкретной бизнес-логики и состоит из следующих частей:

\begin{enumerate}
  \item UI-компоненты общего назначения включают простые React-компоненты (кнопки, лоадеры, таблицы, модальные окна), не содержащие бизнес-логику и используемые в различных контекстах.
  \item Провайдеры контекста, такие как \textit{DragAndDropFilesProvider} и \textit{EnterKeyHandlerProvider}, обеспечивают единообразную работу с событиями и состояниями по всему приложению.
  \item Утилиты и хелперы включают функции для форматирования дат и чисел, генерации уникальных идентификаторов, работы с \textit{localStorage} хранилищем и другие вспомогательные средства.
  \item Кастомные хуки общего назначения (\textit{useSearchParamsListener}, \textit{useWindowSize}, \textit{usePreviousValue} и др.) позволяют сократить дублирование текста программы и упрощают работу с состоянием.
  \item SVG-иконки и графика подключаются через SVGR плагин, что обеспечивает единообразие отображения и управление атрибутами SVG.
  \item Компоненты навигации и управления состоянием, включая \textit{PaginationComponent}, \textit{Breadcrumbs}, \textit{Tabs} и другие, обеспечивают взаимодействие с URL-параметрами и организацию структуры интерфейса.
\end{enumerate}
\subsubsection{Слой \textit{entities}}

Слой \textit{entities} отвечает за интеграцию с внешними сервисами.
\begin{enumerate}
  \item \textit{api/}: тонкий слой-абстракция над HTTP-клиентами (\textit{fetch}/\textit{axios}), где функции названы в соответствии с операциями Swagger/OpenAPI (например, \textit{getUserProfile}, \textit{createOrder});
  \item \textit{types/}: TypeScript-интерфейсы и типы для запросов и ответов (например, \textit{UserProfileResponseType}, \textit{OrderCreateRequestBodyType});
  \item \textit{models/}: классы и mapper-функции для преобразования сырых данных из API в удобные объекты;
  \item \textit{services/}: обёртки для работы с локальным кэшем (IndexedDB, \textit{localStorage}) и реализации retry-логики и таймаутов.
\end{enumerate}

\subsubsection{Слои \textit{features}, \textit{widgets}, \textit{pages}}

Главные рабочие слои приложения, отвечающие за реализацию конкретной функциональности:
\begin{enumerate}
  \item Слой \textit{pages} отвечает за маршрутизация и верхний уровень страниц, описывающий пути, guards для доступа, асинхронную загрузку данных и выбор виджетов.
  \item Слой \textit{widgets} содержит презентационные и «умные» компоненты по паттерну Smart/Presentational. Презентационный компонент (\textit{Presentational Component}) отвечает исключительно за UI и принимает данные через пропсы, не взаимодействуя напрямую с API интерфейсом или глобальным состоянием. Умный хук (\textit{Smart Hook}) инкапсулирует бизнес-логику и передаёт необходимые данные в презентационные компоненты.
  \item Слой \textit{features} реализует бизнес-логику и управление состоянием с помощью собственных хуков, редьюсеров и слайсов Redux. В рамках этого слоя используются следующие структуры: файл \textit{hooks.ts} содержит главный хук-фабрику (например, \textit{useRegistrationUniversity}); файл \textit{schema.ts} описывает схемы форм; файл \textit{utils.ts} содержит вспомогательные функции и селекторы; файл \textit{store.ts} подключает функциональность к библиотек Redux или Redux Toolkit.
\end{enumerate}

\subsubsection{Пример виджета RegistrationUniversity}

В листинге~\ref{lst:registration-university-widget} представлен презентационный компонент, отвечающий за отображение формы регистрации университета.

\begin{lstlisting}[breaklines=true,caption=RegistrationUniversityWidget,label=lst:registration-university-widget]
  // Presentation-компонент
  export function RegistrationUniversityWidget() {
    const { formDataRef, isError, setIsError, onSubmit, isLoading } =
      useRegistrationUniversity();

    return (
      <div>...</div>
    );
  }
\end{lstlisting}

В листинге~\ref{lst:use-registration-university} представлена функция, содержащий логику взаимодействия пользователя с формой регистрации университета и обработки отправки запроса на регистрацию.

\begin{lstlisting}[breaklines=true,caption=useRegistrationUniversity,label=lst:use-registration-university]
  // Smart-компонент: хук-фабрика
  export function useRegistrationUniversity() {
    const formDataRef = useRef<RegistrationData>();
    const [isError, setIsError] = useState<string[]>([]);
    const [isLoading, setIsLoading] = useState(false);
    const router = useRouter();

    const onSubmit = async () => {
      ...
    };
    return { formDataRef, isError, setIsError, onSubmit, isLoading };
  }
\end{lstlisting}

\subsubsection{Соглашения по именованию и структуре}

Для поддержания единого стиля и предсказуемости структуры проекта:
\begin{enumerate}
  \item Имена функций и типов для взаимодействия с серверной частью совпадают;
  \item Компоненты и хуки получают префиксы по зоне ответственности (\textit{RegistrationForm}, \textit{useFilesUpload} и т. д.);
  \item В каждом каталоге \textit{features/FeatureName} обязателен минимум файлов: \textit{index.ts} и сегмент;
  \item Названия типов и функций для взаимодействия с серверной частью должны содержать информацию о типе запроса и типе данных (тело запроса или возвращаемые данные).
\end{enumerate}

% Раздел 3.2: Собственная UI-библиотека и генерация форм
\subsection{Собственная UI-библиотека и генерация форм}

\subsubsection{Общая идея и мотивация}
Для обеспечения единого стилистического и функционального каркаса клиентского приложения была разработана собственная библиотека компонентов и утилит, распространяемая через npm-пакет. В её составе присутствуют:
\begin{itemize}
%  \item SCSS-миксины для реализации адаптивных сеток и гибкой типографики,
  \item набор готовых UI-компонентов (например, \textit{Button}, \textit{Tag}, \textit{ScrollProvider}), не содержащих бизнес-логику,
  \item провайдеры глобальных состояний и контекстов (например, для управления скроллом или обработкой событий клавиатуры),
  \item унифицированный генератор форм \textit{FormBuilder}, позволяющий описывать структуру и поведение сложных форм через декларативную схему.
\end{itemize}
Применение данной библиотеки ускоряет процессы разработки и упрощает поддержку интерфейса, так как все ключевые решения собраны в централизованном модуле с единым API и консистентной документацией.

\subsubsection{Компонент \textit{FormBuilder}: концепция и API}
Ключевым элементом библиотеки является компонент \textit{FormBuilder}. Он реализует маршрутизацию данных и событий между декларативной схемой формы и её полями. Основные принципы работы:
\begin{enumerate}
  \item Пользователь задаёт схему параметров формы, представляющую собой массив объектов с полем \textit{type} и соответствующим набором \textit{props}.
  \item \textit{FormBuilder} инициализирует внутреннее состояние формы и передаёт каждому полю текущие значения и функции обработки изменений.
  \item При срабатывании события изменения значения поле уведомляет \textit{FormBuilder}, который обновляет общую модель данных и вызывает коллбэк \textit{onChange}.
\end{enumerate}

Схема формы описывается функцией, возвращающей массив объектов. Каждый объект содержит:

\begin{itemize}
  \item \textit{type}: идентификатор типа элемента (например, \textit{input\_field}, \textit{array\_fields}),
  \item \textit{props}: набор свойств, необходимых для рендеринга и обработки (имя поля, текст метки, дополнительные параметры).
\end{itemize}

Пример описания схемы формы приведён в листинге~\ref{lst:form-scheme}.

\begin{lstlisting}[caption=Пример описания схемы формы,label=lst:form-scheme]
export function inviteTeacherScheme(): FORM_BUILDER_SCHEMA {
  return [
    {
      type: 'input_field',
      props: {
        name: 'email',
        labelText: 'Email'
      }
    },
    {
      type: 'input_field',
      props: {
        type: 'select',
        name: 'department_id',
        ownerInputComponent: <DepartmentSelectField />
      }
    }
  ];
}
\end{lstlisting}

Пример использования компонента \textit{FormBuilder} приведён в листинге~\ref{lst:formbuilder-usage}.

\begin{lstlisting}[caption=Использование \textit{FormBuilder},label=lst:formbuilder-usage]
<FormBuilder schema={inviteTeacherScheme()} 
			 onChange={onChangeFormData}/>
\end{lstlisting}

Компонент \textit{FormBuilder} автоматически распределяет данные между полями и собирает итоговый объект формы, передавая его через \textit{onChange}.

\subsubsection{Типы элементов схемы и их поведение}
Ниже приведены ключевые типы схем, поддерживаемые \textit{FormBuilder}, и описание их функциональности.


Схема \textit{INPUT\_FIELD\_SCHEMA} отвечает за отображение и управление единичным полем ввода.
\begin{itemize}
  \item \textit{name}: ключ в итоговом объекте данных,
  \item \textit{labelText}: отображаемая метка поля,
  \item \textit{hintText} (опционально): текст подсказки под полем,
  \item \textit{type} (опционально): уточняет тип поля (например, \textit{select}, \textit{datetime}),
  \item \textit{ownerInputComponent} (опционально): пользовательский компонент, принимающий \textit{value}, \textit{onChange}, \textit{isError}, \textit{onBlur}.
\end{itemize}

Пример использования схемы представлен в листинге~\ref{lst:input-fields-schema}.

\begin{lstlisting}[caption={Пример INPUT\_FIELD\_SCHEMA},label={lst:input-fields-schema}]
const schema: INPUT_FIELD_SCHEMA = {
  type: 'input_field',
  props: {
    name: 'username',
    labelText: 'Имя пользователя',
    hintText: 'Введите ваш логин'
  }
};
\end{lstlisting}

Схема \textit{ARRAY\_FIELDS\_SCHEMA} предназначена для формирования массива однотипных входных полей. Все вложенные \textit{input\_field} будут собраны в массив.

\begin{itemize}
  \item \textit{name}: имя массива в итоговой модели,
  \item \textit{children}: массив схем элементов, входящих в каждую запись.
\end{itemize}

Пример схемы показан в листинге~\ref{lst:array-fields-schema}.

\begin{lstlisting}[caption={Пример \textit{ARRAY\_FIELDS\_SCHEMA}},label={lst:array-fields-schema}]
const schema: ARRAY_FIELDS_SCHEMA = [
  {
    type: 'array_fields',
    props: {
      name: 'subjects',
      children: [
        {
          type: 'input_field',
          props: {
            name: 'subjectName',
            labelText: 'Название предмета'
          }
        }
      ]
    }
  }
];
\end{lstlisting}

Схема \textit{FORM\_WRAPPER\_SCHEMA} служит для группировки полей без сброса счётчика массивов. Внутренние поля записываются как вложенный объект.

\begin{itemize}
  \item \textit{name}: имя ключа в итоговом объекте,
  \item \textit{children}: схема вложенных элементов.
\end{itemize}

Пример схемы показан в листинге~\ref{lst:form-wrapper-schema}.

\begin{lstlisting}[caption={Пример \textit{FORM\_WRAPPER\_SCHEMA}},label={lst:form-wrapper-schema}]
const schema: FORM_WRAPPER_SCHEMA = [{
  type: 'form_wrapper',
  props: {
    name: 'teacherInfo',
    children: [
      {
        type: 'input_field',
        props: { name: 'firstName', labelText: 'Имя' }
      },
      {
        type: 'input_field',
        props: { name: 'lastName', labelText: 'Фамилия' }
      }
    ]
  }
}];
\end{lstlisting}

Схема \textit{BLOCK\_WRAPPER\_SCHEMA} позволяет визуально группировать элементы без изменения структуры данных: поля в нём обрабатываются как часть текущего массива или объекта. Пример схемы показан в листинге~\ref{lst:block-wrapper-schema}.

\begin{lstlisting}[caption={Пример \textit{BLOCK\_WRAPPER\_SCHEMA}},label={lst:block-wrapper-schema}]
const schema: BLOCK_WRAPPER_SCHEMA = [
  {
    type: 'block_wrapper',
    props: {
      children: [
        {
          type: 'input_field',
          props: {
            name: 'code',
            labelText: 'Код'
          }
        }
      ]
    }
  }
];
\end{lstlisting}

Схема \textit{REACT\_NODE\_SCHEMA} предназначена для вставки произвольного React-элемента в форму. Пример схемы показан в листинге~\ref{lst:react-node-schema}.

\begin{lstlisting}[caption={Пример \textit{REACT\_NODE\_SCHEMA}},label={lst:react-node-schema}]
const schema: REACT_NODE_SCHEMA = [
  {
    type: 'react_node',
    props: {
      node: <CustomSeparator />
    }
  }
];
\end{lstlisting}

\subsubsection{Преимущества и выводы}
Использование компонента \textit{FormBuilder} существенно снижает сложность создания многоуровневых форм:
\begin{itemize}
  \item консистентность API при описании разных типов полей,
  \item возможность единообразной валидации и управления ошибками,
  \item лёгкость расширения — добавление новых типов полей или обёрток сводится к регистрации нового \textit{type} и соответствующего рендерера,
  \item повышение читаемости кода: структура формы полностью отражена в схеме без дублирования логики в компонентах.
\end{itemize}

% Конец раздела 3.2

\subsubsection{Работа с токеном JWT}
Для передачи и обновления токена JWT в сессии используется функция обратного вызова \textit{jwt}, принимающий параметр \textit{trigger}, позволяющий определить сценарий обработки. Логика работы примерно следующая: при первом входе пользователя (\textit{trigger = signIn}) в токен записываются поля \textit{access\_token}, \textit{refresh\_token} и прочие метаданные; при последующих запросах проверяется срок жизни \textit{access\_token}, и при необходимости инициируется процесс его обновления (\textit{trigger = update}). Если же токен ещё валиден, возвращается неизменённая структура. Пример реализации приведён в листинге~\ref{lst:jwt-callback}.

\begin{lstlisting}[caption={Функция обратоного вызова jwt}, label={lst:jwt-callback}]
	async jwt({ token, trigger, user, session }): Promise<JWT> {
		// Срабатывает при первичной аутентификации (signIn)
		if (trigger === 'signIn') {
			return { ...user, error: null };
		}
		// Если токена ещё нет (например, при восстановлении сессии из куки)
		if (token == null) {
			return { ...session?.user, error: 'another' } as JWT;
		}
		// При запросе обновления (trigger = 'update')
		if (trigger === 'update') {
			return await refreshingProcess(token);
		}
		// Во всех остальных случаях (токен валиден), возвращаем прежнее состояние
		return { ...token, error: null };
	},
\end{lstlisting}

Ниже на рисунке~\ref{fig:auth-refresh} показана вся последовательная диаграмма, иллюстрирующая проверку срока жизни JWT токена на клиенте и, при необходимости, получение нового JWT токена по токену обновления. Эта схема помогает понять, как именно библиотека Auth.js взаимодействует с сервером для устойчивого хранения и своевременного обновления токенов без лишних повторных запросов.

\begin{figure}[h]
    \centering
    \includegraphics[width=0.9\textwidth]{static/diagrams/AuthRefresh.png}
    \caption{Схема процесса проверки и обновления JWT токена через токен обновления}
    \label{fig:auth-refresh}
\end{figure}

На рисунке~\ref{fig:auth-refresh} можно выделить два основных этапа: аутентификация пользователя и процесс использования токена.

При входе пользователя в систему происходит следующая последовательность действий:
\begin{enumerate}
    \item Пользователь вводит поля \textit{username} и \textit{password} в приложение;
    \item Модуль Auth.js формирует объект с учётными данными и отправляет запрос на сервер (метод \textit{login});
    \item Сервер возвращает токен доступа и токен обновления;
    \item Модуль Auth.js сохраняет полученный JWT токен в защищённые cookie данные.
\end{enumerate}

После успешной аутентификации при каждом запросе выполняется проверка срока жизни токена и, при необходимости, его обновление.  Ниже приведены основные этапы этой проверки и процесса обновления:
\begin{enumerate}
    \item При каждом запросе клиент проверяет поле \textit{exp} (срок жизни) в токене доступа.
    \item Если текущий момент времени превысил время жизни токена доступа, начинается подготовка к обновлению токена.
    \item В случае истечения срока действия токена доступа модуль Auth.js отправляет токен обновления на сервер. Сервер возвращает новый токен доступа и новый токен обновления. После этого модуль Auth.js заменяет старый токен в cookie браузера на новый.
    \item Если же токен доступа ещё валиден, модуль Auth.js просто использует существующий токен и не делает дополнительных запросов (промежуточные уведомления внизу диаграммы).
\end{enumerate}

Таким образом, схема на рисунке~\ref{fig:auth-refresh} демонстрирует, что клиент всегда сначала пробует воспользоваться существующим токеном доступа, проверяя его валидность. Только если проверка не проходит, выполняется последовательность обновления, благодаря чему повышается отказоустойчивость и исключается ситуация «гонки» при параллельных запросах на обновление.

Важным дополнением к этой концепции является механизм предотвращения <<гонки состояний>> при одновременном запросе нескольких API-методов, обнаруживающих, что токен доступа просрочен. В таких случаях на клиенте сохраняется единственный промис обновления, который переиспользуется всеми последующими запросами до получения ответа от сервера. Пример реализации этого механизма приведён в листинге~\ref{lst:race-condition}.

\begin{lstlisting}[caption={Механизм предотвращения гонки состояний при рефреше токена}, label={lst:race-condition}]
	export type RefreshPromiseStateType = Promise<JWT | null> | null;
	let tokenPromiseState: RefreshPromiseStateType = null;
	export const setTokenPromiseState = (
		promise: RefreshPromiseStateType
	): void => {
		tokenPromiseState = promise;
	};
	export const getTokenPromiseState = (): RefreshPromiseStateType => {
		return tokenPromiseState;
	};

	let tokenState: JWT | null = null;
	export const setTokenState = (newTokenState: JWT | null): void => {
		tokenState = newTokenState;
	};
	export const getTokenState = (): JWT | null => {
		return tokenState;
	};

	let timeoutState: NodeJS.Timeout | null = null;
\end{lstlisting}

На основе приведённой логики обеспечивается централизованная обработка авторизации и обновления токенов без дублирования текста программмы в разных частях приложения. Кроме того, использование одной общей очереди запросов к серверу для обновления JWT токена предотвращает нежелательные состояния гонки и лишние обращения к серверу.

% Раздел 3.4: Модуль «Регистрация и вход»
\subsection{Модуль «Регистрация и вход»}

В системе предусмотрены три сценария регистрации: для университета, студентов и преподавателей. В URL-параметрах \textit{invite\_id} передаются данные приглашения для последующей валидации и передачи на бэкенд.

Выбор формы регистрации осуществляется в зависимости от типа пользователя, переданного через параметры URL. Пример логики выбора виджета приведён в листинге~\ref{lst:registration_widget_select}.

\begin{lstlisting}[caption={Выбор виджета регистрации по типу}, label={lst:registration_widget_select}]
const getForm = () => {
    const type = getSearchParams(REGISTRATION_TYPE_PARAM_NAME) as RegistrationTypesType;
    const inviteId = getInviteId();
    switch (type) {
        case 'teacher':
            return <RegistrationTeacherWidget inviteId={inviteId} />;
        case 'student':
            return <RegistrationStudentWidget inviteId={inviteId} />;
        case 'university':
        default:
            return <RegistrationUniversityWidget />;
    }
};
\end{lstlisting}
На рисунке~\ref{fig:registration_university} показан интерфейс страницы регистрации университета. Пользователь должен ввести название учебного заведения, фамилию, имя и отчество контактного лица, email администрации университета, а также задать и подтвердить пароль. После заполнения всех обязательных полей и нажатия кнопки «Зарегистрировать университет» инициируется отправка данных на сервер.

\begin{figure}[h]
    \centering
    \includegraphics[width=0.5\textwidth]{static/presintation/RegPage.png} % Путь фейковый, замените на реальный при компиляции
    \caption{Страница регистрации университета: поля для названия института, контактного лица (ФИО), email администрации, пароля и подтверждения пароля.}
    \label{fig:registration_university}
\end{figure}

Листинг~\ref{lst:registration_teacher_widget} демонстрирует структуру виджета регистрации преподавателя. Здесь реализована обработка валидности приглашения и логика отображения состояния формы в зависимости от результата асинхронной проверки.

\begin{lstlisting}[caption={Виджет регистрации преподавателя}, label={lst:registration_teacher_widget}]
export function RegistrationTeacherWidget({ inviteId }: RegistrationPropsType) {
    const { initData, onSubmit, isError, setIsError, formDataRef } =
        useRegistrationStudentAndTeacher<typeof registerTeacher>({
            inviteId,
            registrationRequest: registerTeacher
        });

    if (initData === undefined) {
        return 'Loading';
    }

    if (initData === null) {
        // Неверное приглашение или оно истекло
        return 'Error';
    }

    // Дальнейшая логика отображения формы регистрации преподавателя
    ...
}
\end{lstlisting}

\begin{figure}[h]
    \centering
    \includegraphics[width=0.5\textwidth]{static/presintation/LoginPage.png} % Путь фейковый, замените на реальный при компиляции
    \caption{Экран входа в систему: поля для ввода Email и пароля, а также кнопка «Войти».}
    \label{fig:login_page}
\end{figure}

На рисунке~\ref{fig:login_page} представлен экран входа в систему. Для авторизации пользователь вводит Email и пароль, после чего нажимает кнопку «Войти». В случае успешной авторизации система перенаправляет его в личный кабинет; при некорректных учётных данных отображается сообщение об ошибке.

% Конец раздела 3.4

% Раздел 3.5: Административная панель университета
\subsection{Административная панель университета}
Административная панель университета разделена на несколько секций, каждая из которых отвечает за управление соответствующими данными: институтами, кафедрами, группами, преподавателями и студентами. Также предусмотрён модуль для формирования и отправки приглашений.

\subsubsection{Страница списка с таблицей данных}
На страницах списка отображается компонент \texttt{Table} с переданной конфигурацией \texttt{header} и \texttt{body}, а также обрабатываются события клика по строкам для навигации к деталям записи:
\begin{lstlisting}[caption={Компонент страницы списка институтов}]
export function AdminInstitutePage() {
    const { data, body, onClickRow, header } = useListAdminInstitute();

    return (
        <div className={AdminListPageStyle.AdminListPage}>
            <div className={AdminListPageStyle.table}>
                <Table
                    header={header}
                    body={body}
                    onClickRow={onClickRow}
                />
            </div>
            <PaginationComponent totalCount={data.total_count} />
        </div>
    );
}
\end{lstlisting}

\subsubsection{Страница просмотра/редактирования записи}
На странице детализации отображаются текущие данные и предоставляется возможность их изменения через генератор форм и кнопку сохранения. Также реализована операция удаления записи:
\begin{lstlisting}[caption={Компонент страницы детализации института}]
export function AdminInstituteDetailPage({ id }: { id: number }) {
    const { onUpdate, onDelete, onChangeFormData, data } = useDetailAdminInstitute(id);

    return (
        <div className={AdminDetailPageStyle.AdminDetailPage}>
            <div className={AdminDetailPageStyle.content}>
                <h3 className={AdminDetailPageStyle.header}>
                   	Данные института
                </h3>
                <FormBuilder<InstitutePatchType>
                    formDataDefault={data}
                    clearForm={true}
                    onChange={onChangeFormData}
                    schema={instituteSchema()}
                />
                <Button
                    size="large"
                    hierarchy="primary"
                    text="Сохранить"
                    width="fill"
                    onClick={onUpdate}
                />
            </div>
            <div className={AdminDetailPageStyle.action}>
                <h3 className={AdminDetailPageStyle.header}>
                   	Операции
                </h3>
                <Button
                    size="small"
                    hierarchy="primary"
                    warning={true}
                    iconLeft={<Trash01SVG />}
                    text="Удалить"
                    onClick={onDelete}
                    width="fill"
                />
            </div>
        </div>
    );
}
\end{lstlisting}

\subsubsection{Страница создания новой записи}
Компонент создания использует \texttt{FormBuilder} с соответствующей схемой и отправляет форму по событию нажатия кнопки:
\begin{lstlisting}[caption={Компонент страницы создания института}]
export function AdminInstituteCreatePage() {
    const { onChangeFormData, onSend } = useCreateAdminInstitute();

    return (
        <div className={AdminDetailPageStyle.AdminDetailPage}>
            <div className={AdminDetailPageStyle.content}>
                <h1 className={AdminDetailPageStyle.header}>
                   	Создание института
                </h1>
                <FormBuilder<InstitutePostType>
                    schema={instituteSchema()}
                    onChange={onChangeFormData}
                />
                <Button
                    onClick={onSend}
                    text="Создать"
                    size="large"
                    width="fill"
                />
            </div>
        </div>
    );
}
\end{lstlisting}

\subsubsection{Модальные виджеты для создания приглашений}
Для формирования приглашений используются два модальных компонента, управляемые состоянием \texttt{isActiveModalWindow}:
\begin{lstlisting}[caption={Выбор модального окна приглашения}]
const getModalWindow = () => {
    switch (isActiveModalWindow) {
        case 'teacher':
            return <CreateTeacherInvites onClose={() => setIsActiveModalWindow(undefined)} />;
        case 'students':
            return <CreateStudentsInvites onClose={() => setIsActiveModalWindow(undefined)} />;
    }
};
\end{lstlisting}

Кнопки для вызова модальных окон:
\begin{lstlisting}[caption={Кнопки вызова модальных приглашений}]
<ActionField
    title="Пригласить преподавателя"
    subtitle="Сформировать ссылку для регистрации преподавателя"
    onClick={() => setIsActiveModalWindow('teacher')}
/>
<ActionField
    title="Пригласить студентов"
    subtitle="Сформировать ссылку для регистрации студентов"
    onClick={() => setIsActiveModalWindow('students')}
/>
\end{lstlisting}

% Конец раздела 3.5
\subsubsection{Последовательная диаграмма работы модуля Classrooms}
Ниже приведена последовательная диаграмма, иллюстрирующая полный жизненный цикл взаимодействия преподавателя, студента, UI, бэкенда и AI-сервиса при работе с виртуальными классами и заданиями. На рисунке~\ref{fig:classroom-flow} показаны все основные этапы: создание класса, создание задания с указанием промта для автопроверки, получение списков заданий студентом, отправка решения студентом, автоматическая проверка через AI и выставление оценки преподавателем.


На рисунке~\ref{fig:classroom-flow} можно выделить следующие ключевые этапы:

\begin{enumerate}
    \item \textbf{Преподаватель создаёт класс:}
    \begin{itemize}
        \item Преподаватель открывает форму создания класса в своем интерфейсе (UI: Преподаватель).
        \item UI отправляет запрос на бэкенд с данными нового класса.
        \item Бэкенд возвращает подтверждение успешного создания (например, ID нового класса).
        \item UI отображает преподавателю сообщение «Класс создан».
    \end{itemize}

    \item \textbf{Преподаватель создаёт задание с возможностью задать промт для AI-проверки:}
    \begin{itemize}
        \item Преподаватель переходит в форму создания задания, указывая вместе с условием текста задания промт для AI-проверки.
        \item UI отправляет запрос на бэкенд с данными задания и промтом.
        \item Бэкенд возвращает подтверждение успешного сохранения задания.
        \item UI отображает преподавателю сообщение «Задание создано».
    \end{itemize}

    \item \textbf{Студент получает список доступных заданий:}
    \begin{itemize}
        \item Студент открывает интерфейс (UI: Студент) и запрашивает список заданий для конкретного класса.
        \item UI отправляет запрос на бэкенд с ID класса.
        \item Бэкенд возвращает массив доступных заданий.
        \item UI отображает студенту список заданий.
    \end{itemize}

    \item \textbf{Студент отправляет решение на проверку:}
    \begin{itemize}
        \item Студент открывает форму отправки решения, выбирая конкретное задание.
        \item UI отправляет файлы решения и метаданные (например, ID задания, ID студента) на бэкенд.
        \item Бэкенд возвращает подтверждение успешной загрузки решения.
        \item UI отображает студенту сообщение «Решение отправлено».
    \end{itemize}

    \item \textbf{Автопроверка через AI:}
    \begin{itemize}
        \item Бэкенд получает файлы решения и ранее заданный промт к заданию.
        \item Бэкенд отправляет файлы и промт во внешний AI-сервис.
        \item AI-сервис выполняет анализ кода (например, проверку корректности, стилевых нарушений и т. д.) и возвращает результат вместе с комментариями.
        \item Бэкенд сохраняет результат автопроверки и передаёт его UI обоим ролям:
        \begin{itemize}
            \item UI Студента: отображается результат автопроверки (оценка AI, комментарии).
            \item UI Преподавателя: отображается результат автопроверки (для последующей ручной проверки и выставления итоговой оценки).
        \end{itemize}
    \end{itemize}

    \item \textbf{Преподаватель ставит оценку, и студент получает её:}
    \begin{itemize}
        \item Преподаватель открывает форму выставления оценки (UI: Преподаватель) для конкретного решения.
        \item UI отправляет в бэкенд оценку и комментарий преподавателя.
        \item Бэкенд сохраняет оценку, возвращает подтверждение сохранения.
        \item UI отображает преподавателю сообщение «Оценка сохранена», а UI Студента — обновлённую финальную оценку.
    \end{itemize}
\end{enumerate}

Таким образом, последовательная диаграмма на рисунке~\ref{fig:classroom-flow} демонстрирует весь цикл взаимодействий: от создания класса и задания преподавателем до получения студентом финальной оценки после автопроверки и ручного выставления оценки преподавателем.

\subsubsection{Последовательная диаграмма работы модуля Chats}
Модуль «Chats» обеспечивает обмен сообщениями в реальном времени между пользователями системы (преподавателями и студентами) посредством WebSocket-соединения. Ниже на рисунке~\ref{fig:chats-flow} приведена последовательная диаграмма, демонстрирующая ключевые этапы работы чата: открытие приложения, выбор беседы, отправка и приём сообщений, а также обработку истечения JWT и повторное подключение.

\begin{figure}[H]
    \centering
    \includegraphics[width=0.9\textwidth]{static/diagrams/Chats.png}
    \caption{Схема взаимодействия клиента (UI: Боковая панель и UI: Чат), AuthJS (Next.js), бэкенда и WebSocket при работе модуля «Chats».}
    \label{fig:chats-flow}
\end{figure}

На рисунке~\ref{fig:chats-flow} выделены следующие этапы:

\begin{enumerate}
    \item \textbf{Открытие приложения:}
    \begin{itemize}
        \item Пользователь переходит в раздел «Чаты». UI боковой панели запрашивает и отображает список доступных бесед (из кэша или по HTTP).
        \item При загрузке страницы «Чат» UI извлекает \texttt{access\_token} из защищённой cookie (AuthJS).
        \item После получения токена устанавливается WebSocket-соединение с сервером, передавая JWT в заголовке. Бэкенд подтверждает подключение.
    \end{itemize}

    \item \textbf{Выбор чата в боковой панели:}
    \begin{itemize}
        \item Пользователь кликает на одну из бесед (параметр \texttt{chatId}). Боковая панель передаёт этот \texttt{chatId} компоненту «Чат».
        \item UI «Чат» выполняет HTTP GET-запрос к эндпоинту \texttt{/chats/\{chatId\}/messages?page=1} для получения первой страницы сообщений.
        \item Бэкенд возвращает список сообщений, которые UI отображает в области истории сообщений.
    \end{itemize}

    \item \textbf{Отправка нового сообщения:}
    \begin{itemize}
        \item Пользователь пишет текст сообщения и нажимает «Отправить» в UI «Чат».
        \item UI «Чат» эмиттит событие \texttt{send\_message} по WebSocket, передавая объект \{\texttt{chat: chatId, message: {…}}\}.
        \item Бэкенд принимает событие, сохраняет сообщение и возвращает подтверждение приёма.
        \item UI «Чат» временно отображает отправленное сообщение (optimistic UI) со статусом “отправляется” до получения фактического сообщения от сервера.
    \end{itemize}

    \item \textbf{Получение нового сообщения (\texttt{new\_message}):}
    \begin{itemize}
        \item Когда любой участник (в том числе другой пользователь) отправляет сообщение, сервер по WebSocket рассылает событие \texttt{new\_message} всем подписанным участникам данного чата.
        \item UI «Чат» получает событие \texttt{new\_message \{ chat, message \}} и добавляет новое сообщение в конец списка истории.
        \item Боковая панель получает обновлённый \texttt{chatId} для обновления списка бесед (например, переставить текущую беседу наверх) и отображает актуальный список чатов с учётом новых сообщений.
    \end{itemize}

    \item \textbf{Редактирование сообщения (\texttt{edit\_message}):}
    \begin{itemize}
        \item Если пользователь (автор сообщения) редактирует ранее отправленное сообщение, UI «Чат» отправляет серверу по WebSocket событие \texttt{edit\_message \{ chat, message \}}.
        \item Сервер обновляет текст сообщения и рассылает событие \texttt{edit\_message} всем участникам беседы.
        \item UI «Чат» находит сообщение по его \texttt{id} или \texttt{local\_id} и обновляет текст.
        \item После этого боковая панель получает сигнал «передать chatId» для обновления порядка бесед и отображает обновлённый список чатов.
    \end{itemize}

    \item \textbf{Отметка сообщения как прочитанного (\texttt{read\_message}):}
    \begin{itemize}
        \item Когда пользователь прокручивает историю и дожидается видимости новых сообщений, UI «Чат» отправляет событие \texttt{read\_message \{ chat, message \}} по WebSocket.
        \item Сервер обновляет статус сообщения на «прочитано» и уведомляет других участников через событие \texttt{read\_message}.
        \item UI «Чат» и боковая панель получают это событие, обновляют статус соответствующего сообщения и обновляют отображение списка чатов (например, убрать бейдж «новых сообщений»).
    \end{itemize}

    \item \textbf{Обработка истечения JWT и переподключение:}
    \begin{itemize}
        \item Если при работе WebSocket возникает ошибка авторизации (например, \texttt{access\_token} истёк), UI «Чат» получает событие \texttt{connect\_error} от библиотеки Socket.IO.
        \item Компонент проверяет поле \texttt{exp(access\_token)} локально. При истечении:
        \begin{enumerate}
            \item UI «Чат» отправляет \texttt{refresh\_token} на AuthJS/Next.js, получая новый \texttt{access\_token}.
            \item После получения нового токена выполняется повторное подключение WebSocket, передавая обновлённый JWT.
            \item Сервер WebSocket подтверждает новый сеанс подключения.
        \end{enumerate}
        \item Боковая панель и UI «Чат» возобновляют подписку на события и продолжают обмен сообщениями без потери данных.
    \end{itemize}
\end{enumerate}

Таким образом, последовательная диаграмма на рисунке~\ref{fig:chats-flow} отражает полный цикл работы модуля «Chats»: от открытия приложения и установления защищённого соединения до приёма, отправки, редактирования и пометки сообщений, а также автоматического обновления JWT и переподключения WebSocket. Все события обрабатываются как в компоненте UI «Чат», так и в боковой панели, чтобы гарантировать актуальность списка бесед и статусов сообщений.

\newpage
\ESKDthisStyle{formII}
\section*{ЗАКЛЮЧЕНИЕ}
\addcontentsline{toc}{section}{ЗАКЛЮЧЕНИЕ}
\ESKDcolumnII{ЗАКЛЮЧЕНИЕ}

В рамках проведённой дипломной работы было улучшено и упрощено взаимодействие преподавателей и студентов в образовательном процессе университета посредством разработки клиентской части интегрированного образовательного приложения. Основная цель заключалась в повышении качества опыта работы преподавателей и обучения студентов, обеспечивая единое пространство для коммуникаций, автоматизированной проверки заданий и управления учебными ресурсами.

Для достижения поставленной цели были выполнены следующие задачи:
\begin{enumerate}
  \item Проанализирована предметная область и существующие образовательные системы, выявлены их сильные и слабые стороны;
  \item Определены архитектурные и технологические решения (Next.js, React, TypeScript, Redux, Auth.js, Socket.IO) для реализации гибкой, надёжной и удобной клиентской части приложения;
  \item Спроектирован пользовательский интерфейс, обеспечивающий интуитивное и удобное взаимодействие для преподавателей и студентов (навигация, адаптивная вёрстка, единые UI-компоненты);
  \item Разработаны компоненты для управления учебными структурами (институты, кафедры, группы), заданиями и чатами, включая административную панель, что упростило работу преподавателей при ведении учебного процесса;
  \item Интегрированы средства автоматизированной проверки решений студентов с применением ИИ (DeepSeek) в модуле виртуальных классов, что повысило качество и скорость проверки лабораторных работ;
  \item Реализовано тестирование бизнес-логики клиентского приложения с использованием Jest, что подтвердило корректность и стабильность основных функций.
\end{enumerate}

Таким образом, все поставленные задачи выполнены, а основная цель достигнута: предложенные архитектурные решения обеспечили удобство работы преподавателей и улучшили опыт обучения студентов. Разработанный интерфейс продемонстрировал эффективность в упрощении коммуникаций, автоматизации рутинных задач и повышении качества образовательного процесса.

С практической точки зрения, данное клиентское приложение обладает следующими преимуществами:
\begin{enumerate}
  \item Быстрая разработка за счёт переиспользования компонентов и генерации форм, что позволяет преподавателям оперативно адаптировать интерфейс под учебный процесс;
  \item Надёжная безопасность благодаря механизму управления сессиями (JWT, Auth.js) и авторизации;
  \item Удобство сопровождения за счёт модульной структуры и чёткого разделения ответственности, что облегчает расширение и поддержку проекта;
  \item Расширяемость и готовность к новым сценариям благодаря гибкой конфигурации интерфейса, что позволяет быстро внедрять новые образовательные инструменты;
  \item Высокая устойчивость системы, подтверждённая результатами тестирования бизнес-логики, что гарантирует стабильность работы приложения для всех участников образовательного процесса.
\end{enumerate}

Перспективными направлениями дальнейшего развития являются:
\begin{enumerate}
  \item Расширение функционала анализа текста программ на основе искусственного интеллекта за счёт надстроек для автоматической проверки студенческих работ по заданным паттернам;
  \item Добавление офлайн-режима работы с последующей синхронизацией изменений при восстановлении связи, что улучшит опыт студентов в условиях нестабильного интернета;
  \item Разработка мобильных клиентских приложений для повышения доступности приложения и удобства работы преподавателей и студентов на разных устройствах;
  \item Внедрение системы аналитики пользовательской активности для оптимизации интерфейса, оценки эффективности учебного процесса и принятия обоснованных решений по улучшению образовательной среды.
\end{enumerate}

Итоги работы подтверждают, что предложенные решения действительно упростили работу преподавателей, улучшили опыт обучения студентов и повысили качество образовательного процесса. Открываются новые возможности для дальнейших исследований и совершенствования системы.

\newpage
\ESKDthisStyle{formII}
\section*{СПИСОК ЛИТЕРАТУРЫ}
\ESKDcolumnII{СПИСОК ЛИТЕРАТУРЫ}

\printbibliography



\newpage

\ESKDthisStyle{formII}
\ESKDcolumnII{ПРИЛОЖЕНИЕ A}
\section*{ПРИЛОЖЕНИЕ A}
\addcontentsline{toc}{section}{ПРИЛОЖЕНИЕ A}

\setcounter{figure}{0} 
\makeatletter
  \renewcommand{\thefigure}{A.\arabic{figure}}
\makeatother


\begin{figure}[H]
\centering
\includegraphics[width=0.5\linewidth]{static/useCaseDiagramm}
\caption{Диаграмма вариантов использования системы для различных ролей пользователей.}
\label{fig:usecasediagramm}
\end{figure}

\begin{figure}[H]
    \centering
    \includegraphics[width=0.9\textwidth]{static/diagrams/Chats.png}
    \caption{Схема взаимодействия клиента (UI: Боковая панель и UI: Чат), AuthJS (Next.js), бэкенда и WebSocket при работе модуля «Chats».}
    \label{fig:chats-flow}
\end{figure}


\begin{figure}[H]
    \centering
    \includegraphics[width=0.9\textwidth]{static/diagrams/Classroom.png}
    \caption{Схема взаимодействия преподавателя, студента, UI, бэкенда и AI при работе с виртуальными классами}
    \label{fig:classroom-flow}
\end{figure}

\ESKDthisStyle{formII}
\ESKDcolumnII{ПРИЛОЖЕНИЕ Б}
\section*{ПРИЛОЖЕНИЕ Б}
\addcontentsline{toc}{section}{ПРИЛОЖЕНИЕ Б}

Исходный код фронтенда доступен в GitHub-репозитории:
\href{https://github.com/BeSmileV/unichat-front}{https://github.com/BeSmileV/unichat-front}

\end{document}

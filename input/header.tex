%%% Здесь выбираются необходимые графы
\documentclass[russian,utf8,pointsection,nocolumnsxix,nocolumnxxxi,nocolumnxxxii]{eskdtext}
\usepackage{fontspec}
\defaultfontfeatures{Mapping=tex-text}

%%% Чтобы работал eskdx и другие пакеты
\usepackage{xecyr}

%%% Шрифты и кодировки
\usepackage{xunicode,xltxtra}
\usepackage{listings}
\usepackage{longtable}
\usepackage{caption}

%%% Русский текст
\setmainfont{Times New Roman}
\setromanfont{Times New Roman}
\setsansfont{Times New Roman}
\setmonofont{Times New Roman}

%%% Математика
\usepackage{amsmath,amssymb}

%%% Символ градуса
\usepackage{gensymb}

%%% Перенос составных слов
\XeTeXcharclass `\- 24
\XeTeXinterchartoks 24 0 ={\hskip\z@skip}
\XeTeXinterchartoks 0 24 ={\nobreak}

%%% Подпись «Рисунок» вместо «рис. 1»
\addto{\captionsrussian}{\renewcommand{\figurename}{Рисунок}}

%%% Убираем точки после цифр в заголовках
\def\russian@capsformat{%
	\def\postchapter{\@aftersepkern}%
	\def\postsection{\@aftersepkern}%
	\def\postsubsection{\@aftersepkern}%
	\def\postsubsubsection{\@aftersepkern}%
	\def\postparagraph{\@aftersepkern}%
	\def\postsubparagraph{\@aftersepkern}%
}

\sloppy

%%% Графика, ссылки, ToC
\usepackage{graphicx,float,url}
\setcounter{tocdepth}{3}

%%% Гиперссылки
\usepackage{hyperref}

%%% Заголовки разделов
\usepackage{titlesec}
\titleformat{\section}[block]{\normalfont\Large\bfseries}{\thesection}{1em}{}
\titlespacing*{\section}{20pt}{14pt}{8pt}

\titleformat{\subsection}[block]{\normalfont\large\bfseries}{\thesubsection}{1em}{}
\titlespacing*{\subsection}{20pt}{12pt}{6pt}

\titleformat{\subsubsection}[block]{\normalfont\normalsize\bfseries}{\thesubsubsection}{1em}{}
\titlespacing*{\subsubsection}{20pt}{10pt}{4pt}

%%% === Патч нумерации subsubsection ===
\usepackage{etoolbox}               % подтягиваем \preto

% 1) объявляем счётчик
\newcounter{subsubsectcount}

% 2) перед каждым \section: разрыв страницы + сброс нашего счётчика
\preto\section{%
	\clearpage
	\setcounter{subsubsectcount}{0}%
}

% 3) сохраняем оригинальную команду
\let\origsubsubsection\subsubsection

% 4) переопределяем логику
\renewcommand{\subsubsection}[1]{%
	\stepcounter{subsubsectcount}%
	\ifnum\value{subsubsectcount}>6
	% после 5-й — звёздочная версия (без номера и без ToC)
	\origsubsubsection*{#1}%
	\else
	% первые 5 — как обычно
	\origsubsubsection{#1}%
	\fi
}
%%% ====================================

%%% 1) Подключаем пакеты (у вас уже есть)
\usepackage{listings}
\usepackage{xcolor}
\usepackage{fontspec}
\setmonofont{JetBrains Mono}

%%% 1) Базовый стиль
\lstdefinestyle{custom}{
	basicstyle=\ttfamily\small,
	frame=single,
	backgroundcolor=\color{gray!10},
	keywordstyle=\bfseries\color{blue!70!black},
	commentstyle=\itshape\color{gray!60},
	stringstyle=\color{red!70!black},
	numbers=left,
	numberstyle=\tiny\color{gray!50},
	xleftmargin=2em,
	xrightmargin=2em,
	showstringspaces=false,
}

%%% 2) Язык TypeScript/TSX
\lstdefinelanguage{TypeScript}{
	sensitive=true,          % регистрозависимый
	breaklines=true,         % перенос длинных строк
	morekeywords={%
		abstract,any,as,asserts,async,await,boolean,break,case,catch,class,const,continue,%
		debugger,declare,default,delete,do,else,enum,export,extends,false,finally,for,%
		from,function,get,if,implements,import,in,instanceof,interface,let,module,namespace,%
		never,new,null,number,of,package,private,protected,public,readonly,require,return,%
		set,static,string,super,switch,symbol,this,throw,true,try,type,typeof,undefined,%
		unique,unknown,var,void,while,with,yield%
	},
	morecomment=[l]{//},
	morecomment=[s]{/*}{*/},
	morestring=[b]",
	morestring=[b]',
	alsoletter={<,>,/},
	moredelim=**[is][\color{blue!50!black}]{<}{>},
	literate={%
		{<}{{$<$}}1
		{>}{{$>$}}1
		{/}{{$/$}}1
	}%
}  % <-- убедитесь, что закрывающая } на месте

%%% 3) Стиль для TS/TSX
\lstdefinestyle{tscustom}{
style=custom,
language=TypeScript,
tabsize=2,
breakatwhitespace=false,
columns=fullflexible,
keepspaces=true,
breakautoindent=true,
breakindent=1em,
postbreak=\mbox{\textcolor{gray}{$\hookrightarrow$}\space},
}

%%% 4) Устанавливаем tscustom по умолчанию
\lstset{style=tscustom}




